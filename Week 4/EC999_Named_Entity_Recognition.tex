\documentclass{beamer}\usepackage[]{graphicx}\usepackage[]{color}
%% maxwidth is the original width if it is less than linewidth
%% otherwise use linewidth (to make sure the graphics do not exceed the margin)
\makeatletter
\def\maxwidth{ %
  \ifdim\Gin@nat@width>\linewidth
    \linewidth
  \else
    \Gin@nat@width
  \fi
}
\makeatother

\definecolor{fgcolor}{rgb}{0.345, 0.345, 0.345}
\newcommand{\hlnum}[1]{\textcolor[rgb]{0.686,0.059,0.569}{#1}}%
\newcommand{\hlstr}[1]{\textcolor[rgb]{0.192,0.494,0.8}{#1}}%
\newcommand{\hlcom}[1]{\textcolor[rgb]{0.678,0.584,0.686}{\textit{#1}}}%
\newcommand{\hlopt}[1]{\textcolor[rgb]{0,0,0}{#1}}%
\newcommand{\hlstd}[1]{\textcolor[rgb]{0.345,0.345,0.345}{#1}}%
\newcommand{\hlkwa}[1]{\textcolor[rgb]{0.161,0.373,0.58}{\textbf{#1}}}%
\newcommand{\hlkwb}[1]{\textcolor[rgb]{0.69,0.353,0.396}{#1}}%
\newcommand{\hlkwc}[1]{\textcolor[rgb]{0.333,0.667,0.333}{#1}}%
\newcommand{\hlkwd}[1]{\textcolor[rgb]{0.737,0.353,0.396}{\textbf{#1}}}%
\let\hlipl\hlkwb

\usepackage{framed}
\makeatletter
\newenvironment{kframe}{%
 \def\at@end@of@kframe{}%
 \ifinner\ifhmode%
  \def\at@end@of@kframe{\end{minipage}}%
  \begin{minipage}{\columnwidth}%
 \fi\fi%
 \def\FrameCommand##1{\hskip\@totalleftmargin \hskip-\fboxsep
 \colorbox{shadecolor}{##1}\hskip-\fboxsep
     % There is no \\@totalrightmargin, so:
     \hskip-\linewidth \hskip-\@totalleftmargin \hskip\columnwidth}%
 \MakeFramed {\advance\hsize-\width
   \@totalleftmargin\z@ \linewidth\hsize
   \@setminipage}}%
 {\par\unskip\endMakeFramed%
 \at@end@of@kframe}
\makeatother

\definecolor{shadecolor}{rgb}{.97, .97, .97}
\definecolor{messagecolor}{rgb}{0, 0, 0}
\definecolor{warningcolor}{rgb}{1, 0, 1}
\definecolor{errorcolor}{rgb}{1, 0, 0}
\newenvironment{knitrout}{}{} % an empty environment to be redefined in TeX

\usepackage{alltt}
\usetheme{default}
%\usetheme{Malmoe}

\title[EC999: Quantitative Text Analysis]{EC999: Named Entity Recognition} \def\newblock{\hskip .11em plus .33em minus .07em}


\def\Tiny{\fontsize{10pt}{10pt}\selectfont}
\def\smaller{\fontsize{8pt}{8pt}\selectfont}

\institute[Warwick]{University of Chicago \& University of Warwick}
\author[Thiemo Fetzer]{Thiemo Fetzer}

 \date{\today}

\usepackage{natbib}
\usepackage{amsmath}
\usepackage{hyperref}
\usepackage{graphicx}
\usepackage{graphics}

\usepackage{amsfonts}
\usepackage{amssymb}
\usepackage{pdfpages}
\usepackage{natbib}
\usepackage{hyperref}
%\usepackage{enumitem}
 \usepackage{pgffor}
\usepackage{booktabs,caption,fixltx2e}
\usepackage[flushleft]{threeparttable}
\usepackage{verbatim} 
\usepackage{cancel}
\newcommand\xxcancel[1]{\xcancel{#1}\vphantom{#1}}

\usepackage{mathtools,xparse}

\newenvironment{Description}
               {\list{}{\labelwidth=0pt \itemindent-\leftmargin
                        \let\makelabel\Descriptionlabel
                        % or whatever
               }}
               {\endlist}
\newcommand*\Descriptionlabel[1]{%
  \hspace\labelsep
  \normalfont%  reset current font setting
  \color{blue}\bfseries\sffamily% or whatever 
  #1}


\setbeamersize{text margin left = 16pt, text margin right = 16pt}
\newcommand{\code}[1]{\texttt{#1}}

\newenvironment<>{algorithm}[1][\undefined]{%
\begin{actionenv}#2%
\ifx#1\undefined%
   \def\insertblocktitle{Algorithm}%
\else%
   \def\insertblocktitle{Algorithm ({\em#1})}%
\fi%
\par%
\mode<presentation>{%
  \setbeamercolor{block title}{fg=white,bg=yellow!50!black}
  \setbeamercolor{block body}{fg=black,bg=yellow!20}
}%
\usebeamertemplate{block begin}\em}
{\par\usebeamertemplate{block end}\end{actionenv}}


\newenvironment<>{assumption}[1][\undefined]{%
\begin{actionenv}#2%
\ifx#1\undefined%
   \def\insertblocktitle{Assumption}%
\else%
   \def\insertblocktitle{Assumption ({\em#1})}%
\fi%
\par%
\mode<presentation>{%
  \setbeamercolor{block title}{fg=white,bg=blue!50!black}
  \setbeamercolor{block body}{fg=black,bg=blue!20}
}%
\usebeamertemplate{block begin}\em}
{\par\usebeamertemplate{block end}\end{actionenv}}

%changing spacing between knitr code and output
\usepackage{etoolbox} 
\makeatletter 
\preto{\@verbatim}{\topsep=0pt \partopsep=0pt } 
\makeatother
\renewenvironment{knitrout}{\setlength{\topsep}{0mm}}{}
\IfFileExists{upquote.sty}{\usepackage{upquote}}{}
\begin{document}



\AtBeginSection[]
{
 \begin{frame}<beamer>
 \frametitle{Plan}
 \tableofcontents[currentsection]
 \end{frame}
}
\maketitle
 

%%%%%%%%%%%%%%%%%%%%%%%%%%%%


%%%%%%%%%%%%%%%%%%%%%%%%%%%%%%%%%%%%%%%%%%%%%%%%%%%%%%%%%%
\begin{frame}[fragile]{Named Entity Recognition in Information Retrieval}

\begin{itemize}

\item Information retrieval systems extract clear, factual information

\item Can think of trying to answer questions such as: Who? What? Where? When? 

\item Named Entities are very important subtask to many information retrieval tasks.

\item Named Entity recognition systems are build to \emph{identify} and \emph{classify} named entites of specific types. 

\item Typically they separate: Person, Organization, Date, Location,...


\item NER is useful for:

\begin{enumerate}
\item Named entities can be indexed, linked off, etc.
\item Sentiment can be attributed to companies or products
\item A lot of IE relations are associations between named entities
\item For question answering, answers are often named entities.

\end{enumerate}
\end{itemize}

\end{frame}
%%%%%%%%%%%%%%%%%%%%%%%%%%%%%%%%%%%%%%%%%%%%%%%%%%%%%%%%%%%%%%%%%%%%%%%%%



%%%%%%%%%%%%%%%%%%%%%%%%%%%%%%%%%%%%%%%%%%%%%%%%%%%%%%%%%%
\begin{frame}[fragile]{Named Entity Identification}

The decision by the independent MP \textbf{Andrew Wilkie} to withdraw
his support for the minority \textbf{Labor} government sounded
dramatic but it should not further threaten its stability. When,
after the \textbf{2010} election, \textbf{Wilkie}, \textbf{Rob Oakeshott}, \textbf{Tony Windsor}
and the \textbf{Greens} agreed to support \textbf{Labor}, they gave just two
guarantees: confidence and supply.

\end{frame}
%%%%%%%%%%%%%%%%%%%%%%%%%%%%%%%%%%%%%%%%%%%%%%%%%%%%%%%%%%%%%%%%%%%%%%%%



%%%%%%%%%%%%%%%%%%%%%%%%%%%%%%%%%%%%%%%%%%%%%%%%%%%%%%%%%%
\begin{frame}[fragile]{Named Entity Categorization}

The decision by the independent MP \textbf{Andrew Wilkie - PERSON} to withdraw
his support for the minority \textbf{Labor - ORGANIZATION} government sounded
dramatic but it should not further threaten its stability. When,
after the \textbf{2010 - DATE} election, \textbf{Wilkie - PERSON}, \textbf{Rob Oakeshott - PERSON}, \textbf{Tony Windsor - PERSON}
and the \textbf{Greens - ORGANIZATION} agreed to support \textbf{Labor - ORGANIZATION}, they gave just two
guarantees: confidence and supply.

\end{frame}
%%%%%%%%%%%%%%%%%%%%%%%%%%%%%%%%%%%%%%%%%%%%%%%%%%%%%%%%%%%%%%%%%%%%%%%%




%%%%%%%%%%%%%%%%%%%%%%%%%%%%%%%%%%%%%%%%%%%%%%%%%%%%%%%%%%
\begin{frame}[fragile]{Named Entity Recognition}

\begin{itemize}

\item Crucial for Information Extraction, Question Answering and
Information Retrieval
\item[] $\Rightarrow$ Up to 10\% of a newswire text may consist of proper names, dates, times,...

\item Relational information is built on top of Named Entities
\item Many web pages tag various entities, with links to bio
or topic pages, etc.
\item[] $\Rightarrow$ knowledge graphs build on top of Wikipedia use NER parsing of articles.

\item \url{https://www.wikidata.org/wiki/Q22686}

\end{itemize}

\end{frame}
%%%%%%%%%%%%%%%%%%%%%%%%%%%%%%%%%%%%%%%%%%%%%%%%%%%%%%%%%%%%%%%%%%%%%%%%


%%%%%%%%%%%%%%%%%%%%%%%%%%%%%%%%%%%%%%%%%%%%%%%%%%%%%%%%%%
\begin{frame}[fragile]{Three Approaches to Named Entity Recognition}

\begin{enumerate}

\item Rule based retrieval: e.g. through regular expressions
\item[] $\Rightarrow$ remember the email vs telefon number extraction regexes?

\item Classifiers
\item[] In this course we will introduce a range, especially Naive Bayes, logistic regression, ...


\item Sequence models
\item[] Hiden Markov Models, ...


\end{enumerate}

\end{frame}
%%%%%%%%%%%%%%%%%%%%%%%%%%%%%%%%%%%%%%%%%%%%%%%%%%%%%%%%%%%%%%%%%%%%%%%%



%%%%%%%%%%%%%%%%%%%%%%%%%%%%%%%%%%%%%%%%%%%%%%%%%%%%%%%%%%
\begin{frame}[fragile]{Rule-based retrieval}

\begin{center}
\includegraphics[scale=0.5]{figures/02_create_new_ical01.jpg}
\end{center}

Rule based extraction of dates using regular expressions

\end{frame}
%%%%%%%%%%%%%%%%%%%%%%%%%%%%%%%%%%%%%%%%%%%%%%%%%%%%%%%%%%%%%%%%%%%%%%%%




%%%%%%%%%%%%%%%%%%%%%%%%%%%%%%%%%%%%%%%%%%%%%%%%%%%%%%%%%%
\begin{frame}[fragile]{Rule-based retrieval}

\begin{itemize}
\item  If extracting from automatically generated web pages, simple
regex patterns usually work as behind the scenes a common template forces a layout.

\item[] HTML page
\item[] \code{<b>(.*?)</b>}

\item For certain restricted, common types of entities in unstructured
text, simple regex patterns also usually work.
\item[] Finding (US) phone numbers
\item[] \code{([2-9][0-9]\{2\})[- .]([0-9]\{3\})[- .]([0-9]\{4\})}

\end{itemize}
\end{frame}
%%%%%%%%%%%%%%%%%%%%%%%%%%%%%%%%%%%%%%%%%%%%%%%%%%%%%%%%%%%%%%%%%%%%%%%%

%%%%%%%%%%%%%%%%%%%%%%%%%%%%%%%%%%%%%%%%%%%%%%%%%%%%%%%%%%
\begin{frame}[fragile]{Rule-based retrieval}

\begin{itemize}
\item  Can leverage POS tagging to define a set of rules to extract training data for a classifier.

\item Remember pairs of "NN - NN" extracted in collocation discovery exercise?


\item Determining which person holds what office in what organization
 \item[] [person] , [office] of [org]
 
\item[] This defines a sequence of nouns that could be extracted using a regex searching for [NN], [NN] of [NN].

\end{itemize}
\end{frame}
%%%%%%%%%%%%%%%%%%%%%%%%%%%%%%%%%%%%%%%%%%%%%%%%%%%%%%%%%%%%%%%%%%%%%%%%



%%%%%%%%%%%%%%%%%%%%%%%%%%%%%%%%%%%%%%%%%%%%%%%%%%%%%%%%%%
\begin{frame}[fragile]{Machine Learning Approach to Named Entity Recognition}

\textbf{Training}
\begin{enumerate}
\item  Collect a set of representative training documents 

\item Label each token for its entity class or other (O) 

\item Design feature extractors appropriate to the text and classes 

\item. Train a sequence classifier to predict the labels from the data [more next week]

\end{enumerate}

\textbf{Testing}

\begin{enumerate}
\item Receive a set of testing documents 
\item Run sequence model inference to label each token 
\item Appropriately output the recognized entities 

\end{enumerate}

\end{frame}
%%%%%%%%%%%%%%%%%%%%%%%%%%%%%%%%%%%%%%%%%%%%%%%%%%%%%%%%%%%%%%%%%%%%%%%%




%%%%%%%%%%%%%%%%%%%%%%%%%%%%%%%%%%%%%%%%%%%%%%%%%%%%%%%%%%
\begin{frame}[fragile]{Features for sequence labeling}


\begin{itemize}

\item Words, current word, previous word, 

\item Parts of words such as indicators for suffixes common to names of places or locations.

\item Part of speech tags 

\item Shapes of words

\end{itemize}

\end{frame}
%%%%%%%%%%%%%%%%%%%%%%%%%%%%%%%%%%%%%%%%%%%%%%%%%%%%%%%%%%%%%%%%%%%%%%%%


%%%%%%%%%%%%%%%%%%%%%%%%%%%%%%%%%%%%%%%%%%%%%%%%%%%%%%%%%%
\begin{frame}[fragile]{MEMM inference in systems}

\begin{itemize}

\item Conditional Markov Model the classifiers makes a decision at a time conditional on the evidence in the neighborhood.

\item Typically use a type of logistic regression classifier.

\item Example for data used for classifier

\end{itemize}

\begin{tabular}{ccc}
$w$ & POS $p$ & NER \\ 
Three & CD & O \\ 
ISIL & NNP & ORG \\ 
terrorists & NNS & O  \\
were & VBD & O \\ 
killed & VBN & O  \\
in & IN & O \\ 
Rakka & NNP & \textbf{LOC}  \\
. & . & O 
\end{tabular}

Features at word $w_n = \text{Rakka}$: $\{w_{n-1} = "\text{in}", w_n = \text{Rakka}, w_{n+1} = ".", p_{n-1}="IN", p_{n}="NNP",...\}$.

\end{frame}
%%%%%%%%%%%%%%%%%%%%%%%%%%%%%%%%%%%%%%%%%%%%%%%%%%%%%%%%%%%%%%%%%%%%%%%%


%%%%%%%%%%%%%%%%%%%%%%%%%%%%%%%%%%%%%%%%%%%%%%%%%%%%%%%%%%
\begin{frame}[fragile]{Named Entity Recognition in R}

\begin{itemize}

\item There are a couple of built named entity recognition modules built into easy access NLP packages.

\item Most notably are OpenNLP and the Stanford NLP, both require \code{rJava} package and a Java installation on the operating system.
\item[] \code{install.packages('rJava')} and installation of \code{Java JDK 1.2 or higher}

\item I will also show how to access purpose built classifiers built for very specific tasks can be made accessible to work in R.

\end{itemize}


\end{frame}
%%%%%%%%%%%%%%%%%%%%%%%%%%%%%%%%%%%%%%%%%%%%%%%%%%%%%%%%%%%%%%%%%%%%%%%%



%%%%%%%%%%%%%%%%%%%%%%%%%%%%%%%%%%%%%%%%%%%%%%%%%%%%%%%%%%
\begin{frame}[fragile]{OpenNLP}

The Apache OpenNLP library is a machine learning based toolkit for the processing of natural language text.

\includegraphics[scale=0.5]{figures/onlplogo.jpg}

It supports the most common NLP tasks, such as tokenization, sentence segmentation, part-of-speech tagging, named entity extraction, chunking, parsing, and coreference resolution. These tasks are usually required to build more advanced text processing services

\url{https://opennlp.apache.org}
\end{frame}
%%%%%%%%%%%%%%%%%%%%%%%%%%%%%%%%%%%%%%%%%%%%%%%%%%%%%%%%%%%%%%%%%%%%%%%%



%%%%%%%%%%%%%%%%%%%%%%%%%%%%%%%%%%%%%%%%%%%%%%%%%%%%%%%%%%
\begin{frame}[fragile]{Opening a pipeline between OpenNLP and R}

\begin{knitrout}\tiny
\definecolor{shadecolor}{rgb}{0.969, 0.969, 0.969}\color{fgcolor}\begin{kframe}
\begin{alltt}
\hlcom{# install.packages(c('NLP', 'openNLP')) install.packages('openNLPmodels.en', repos =}
\hlcom{# 'http://datacube.wu.ac.at/', type = 'source')}
\hlkwd{library}\hlstd{(NLP)}
\hlkwd{library}\hlstd{(openNLP)}
\hlkwd{library}\hlstd{(openNLPmodels.en)}

\hlstd{TRUMP} \hlkwb{<-} \hlkwd{readLines}\hlstd{(}\hlstr{"../../Data/Speeches/trump-foreignpolicy.txt"}\hlstd{)}
\hlstd{TRUMP} \hlkwb{<-} \hlkwd{paste}\hlstd{(TRUMP,} \hlkwc{collapse} \hlstd{=} \hlstr{" "}\hlstd{)}
\hlstd{TRUMP} \hlkwb{<-} \hlkwd{gsub}\hlstd{(}\hlstr{"([A-Z]\{4,\})"}\hlstd{,} \hlstr{""}\hlstd{, TRUMP)}
\hlcom{# create instance}
\hlstd{word_ann} \hlkwb{<-} \hlkwd{Maxent_Word_Token_Annotator}\hlstd{()}
\hlstd{sent_ann} \hlkwb{<-} \hlkwd{Maxent_Sent_Token_Annotator}\hlstd{()}

\hlstd{TRUMP_annotate} \hlkwb{<-} \hlkwd{annotate}\hlstd{(TRUMP,} \hlkwd{list}\hlstd{(sent_ann, word_ann))}

\hlkwd{head}\hlstd{(TRUMP_annotate)}
\end{alltt}
\begin{verbatim}
##  id type     start end features
##   1 sentence     1  10 constituents=<<integer,3>>
##   2 sentence    12  34 constituents=<<integer,7>>
##   3 sentence    36  68 constituents=<<integer,9>>
##   4 sentence    70 168 constituents=<<integer,19>>
##   5 sentence   170 220 constituents=<<integer,10>>
##   6 sentence   222 327 constituents=<<integer,22>>
\end{verbatim}
\begin{alltt}
\hlcom{# combine Annotations with text}
\hlstd{TRUMP_doc} \hlkwb{<-} \hlkwd{AnnotatedPlainTextDocument}\hlstd{(TRUMP, TRUMP_annotate)}
\end{alltt}
\end{kframe}
\end{knitrout}
\end{frame}
%%%%%%%%%%%%%%%%%%%%%%%%%%%%%%%%%%%%%%%%%%%%%%%%%%%%%%%%%%%%%%%%%%%%%%%%%



%%%%%%%%%%%%%%%%%%%%%%%%%%%%%%%%%%%%%%%%%%%%%%%%%%%%%%%%%%
\begin{frame}[fragile]{OpenNLP Named Entity Recognition types}
OpenNLP can find dates, locations, money, organizations, percentages, people, and times. 

\begin{knitrout}\tiny
\definecolor{shadecolor}{rgb}{0.969, 0.969, 0.969}\color{fgcolor}\begin{kframe}
\begin{alltt}
\hlcom{# need to create instances of different types annotators supported values for kind are}
\hlcom{# 'date', 'location', 'money', 'organization', 'percentage', 'person', 'misc'}
\hlstd{person_ann} \hlkwb{<-} \hlkwd{Maxent_Entity_Annotator}\hlstd{(}\hlkwc{kind} \hlstd{=} \hlstr{"person"}\hlstd{)}
\hlstd{location_ann} \hlkwb{<-} \hlkwd{Maxent_Entity_Annotator}\hlstd{(}\hlkwc{kind} \hlstd{=} \hlstr{"location"}\hlstd{)}
\hlstd{organization_ann} \hlkwb{<-} \hlkwd{Maxent_Entity_Annotator}\hlstd{(}\hlkwc{kind} \hlstd{=} \hlstr{"organization"}\hlstd{)}

\hlcom{# pipeline for processing, NER annotators require sentence and word boundaries as inputs}
\hlstd{pipeline} \hlkwb{<-} \hlkwd{list}\hlstd{(sent_ann, word_ann, location_ann, organization_ann, person_ann)}
\hlstd{TRUMP_annotate} \hlkwb{<-} \hlkwd{annotate}\hlstd{(TRUMP, pipeline)}
\hlstd{TRUMP_doc} \hlkwb{<-} \hlkwd{AnnotatedPlainTextDocument}\hlstd{(TRUMP, TRUMP_annotate)}

\hlstd{TRUMP_doc}
\end{alltt}
\begin{verbatim}
## <<AnnotatedPlainTextDocument>>
## Metadata:  0
## Annotations:  1, length(s): 5475
## Content:  chars: 25389
\end{verbatim}
\end{kframe}
\end{knitrout}
\end{frame}
%%%%%%%%%%%%%%%%%%%%%%%%%%%%%%%%%%%%%%%%%%%%%%%%%%%%%%%%%%%%%%%%%%%%%%%%%


%%%%%%%%%%%%%%%%%%%%%%%%%%%%%%%%%%%%%%%%%%%%%%%%%%%%%%%%%%
\begin{frame}[fragile]{OpenNLP Extract Named Entities from Annotated Document}
\begin{knitrout}\tiny
\definecolor{shadecolor}{rgb}{0.969, 0.969, 0.969}\color{fgcolor}\begin{kframe}
\begin{alltt}
\hlcom{# Extract entities from an AnnotatedPlainTextDocument}
\hlstd{entities} \hlkwb{<-} \hlkwa{function}\hlstd{(}\hlkwc{doc}\hlstd{,} \hlkwc{kind}\hlstd{) \{}
    \hlstd{s} \hlkwb{<-} \hlstd{doc}\hlopt{$}\hlstd{content}
    \hlstd{a} \hlkwb{<-} \hlkwd{annotations}\hlstd{(doc)[[}\hlnum{1}\hlstd{]]}
    \hlkwa{if} \hlstd{(}\hlkwd{hasArg}\hlstd{(kind)) \{}
        \hlstd{k} \hlkwb{<-} \hlkwd{sapply}\hlstd{(a}\hlopt{$}\hlstd{features, `[[`,} \hlstr{"kind"}\hlstd{)}
        \hlstd{s[a[k} \hlopt{==} \hlstd{kind]]}
    \hlstd{\}} \hlkwa{else} \hlstd{\{}
        \hlstd{s[a[a}\hlopt{$}\hlstd{type} \hlopt{==} \hlstr{"entity"}\hlstd{]]}
    \hlstd{\}}
\hlstd{\}}

\hlkwd{sort}\hlstd{(}\hlkwd{table}\hlstd{(}\hlkwd{entities}\hlstd{(TRUMP_doc,} \hlkwc{kind} \hlstd{=} \hlstr{"person"}\hlstd{)),} \hlkwc{decreasing} \hlstd{=} \hlnum{TRUE}\hlstd{)[}\hlnum{1}\hlopt{:}\hlnum{10}\hlstd{]}
\end{alltt}
\begin{verbatim}
## 
##   Hillary Clinton           Clinton           Hillary Hillary Clinton’s              Bill 
##                23                 9                 8                 6                 3 
##    Bernie Sanders      Bill Clinton          Clintons   Abraham Lincoln    Bill Clinton’s 
##                 2                 2                 2                 1                 1
\end{verbatim}
\begin{alltt}
\hlkwd{sort}\hlstd{(}\hlkwd{table}\hlstd{(}\hlkwd{entities}\hlstd{(TRUMP_doc,} \hlkwc{kind} \hlstd{=} \hlstr{"location"}\hlstd{)),} \hlkwc{decreasing} \hlstd{=} \hlnum{TRUE}\hlstd{)[}\hlnum{1}\hlopt{:}\hlnum{10}\hlstd{]}
\end{alltt}
\begin{verbatim}
## 
## United States         China       America         Libya          Iran          Iraq 
##            13             9             5             5             3             3 
##   Middle East         Egypt        Israel      New York 
##             3             2             2             2
\end{verbatim}
\end{kframe}
\end{knitrout}
\end{frame}
%%%%%%%%%%%%%%%%%%%%%%%%%%%%%%%%%%%%%%%%%%%%%%%%%%%%%%%%%%%%%%%%%%%%%%%%%


%%%%%%%%%%%%%%%%%%%%%%%%%%%%%%%%%%%%%%%%%%%%%%%%%%%%%%%%%%
\begin{frame}[fragile]{OpenNLP Handling}

\begin{itemize}

\item OpenNLP handling in R is a bit cumbersome, have to set up pipelines etc...

\item Could write a custom function to perform a specific NLP task and return vector of , e.g. named entities.

\item For SENNA, this type of processing pipeline set up is explained here: \url{http://www.trfetzer.com/a-simple-pipeline-to-access-senna-nlp-toolkit-through-r/}

\item Next present StanfordNLP package, which is easier to work with in R.

\end{itemize}

\end{frame}
%%%%%%%%%%%%%%%%%%%%%%%%%%%%%%%%%%%%%%%%%%%%%%%%%%%%%%%%%%%%%%%%%%%%%%%%%




%%%%%%%%%%%%%%%%%%%%%%%%%%%%%%%%%%%%%%%%%%%%%%%%%%%%%%%%%%
\begin{frame}[fragile]{Stanford CoreNLP}

Stanford CoreNLP integrates many of Stanford’s NLP tools, including the part-of-speech (POS) tagger, the named entity recognizer (NER), the parser, the coreference resolution system, sentiment analysis, bootstrapped pattern learning, and the open information extraction tools.
\begin{center}
\includegraphics[scale=0.4]{figures/CoreNLP.png}
\end{center}
\url{http://stanfordnlp.github.io/CoreNLP/}

\end{frame}
%%%%%%%%%%%%%%%%%%%%%%%%%%%%%%%%%%%%%%%%%%%%%%%%%%%%%%%%%%%%%%%%%%%%%%%%%


%%%%%%%%%%%%%%%%%%%%%%%%%%%%%%%%%%%%%%%%%%%%%%%%%%%%%%%%%%
\begin{frame}[fragile]{Stanford CoreNLP}
\begin{center}
\includegraphics[scale=0.3]{figures/Xi-Jinping.png}
\end{center}
\url{http://stanfordnlp.github.io/CoreNLP/}
\end{frame}
%%%%%%%%%%%%%%%%%%%%%%%%%%%%%%%%%%%%%%%%%%%%%%%%%%%%%%%%%%%%%%%%%%%%%%%%%



%%%%%%%%%%%%%%%%%%%%%%%%%%%%%%%%%%%%%%%%%%%%%%%%%%%%%%%%%%
\begin{frame}[fragile]{CoreNLP R-package}
\begin{knitrout}\tiny
\definecolor{shadecolor}{rgb}{0.969, 0.969, 0.969}\color{fgcolor}\begin{kframe}
\begin{alltt}
\hlcom{# install relevant packages devtools::install_github('statsmaths/coreNLP') this installs}
\hlcom{# the full stanford corenlp coreNLP::downloadCoreNLP()}
\hlkwd{library}\hlstd{(rJava)}
\hlkwd{library}\hlstd{(coreNLP)}
\hlcom{# creating an instance (opening the pipeline)}
\hlkwd{initCoreNLP}\hlstd{()}
\hlstd{str} \hlkwb{<-} \hlkwd{c}\hlstd{(}\hlstr{"President George Bush spoke to the troops on board of an aircraft carrier in the Gulf of Mexico."}\hlstd{)}
\hlstd{output} \hlkwb{<-} \hlkwd{annotateString}\hlstd{(str)}
\hlkwd{getToken}\hlstd{(output)}
\end{alltt}
\begin{verbatim}
##    sentence id     token     lemma CharacterOffsetBegin CharacterOffsetEnd POS      NER
## 1         1  1 President President                    0                  9 NNP        O
## 2         1  2    George    George                   10                 16 NNP   PERSON
## 3         1  3      Bush      Bush                   17                 21 NNP   PERSON
## 4         1  4     spoke     speak                   22                 27 VBD        O
## 5         1  5        to        to                   28                 30  TO        O
## 6         1  6       the       the                   31                 34  DT        O
## 7         1  7    troops    troops                   35                 41 NNS        O
## 8         1  8        on        on                   42                 44  IN        O
## 9         1  9     board     board                   45                 50  NN        O
## 10        1 10        of        of                   51                 53  IN        O
## 11        1 11        an         a                   54                 56  DT        O
## 12        1 12  aircraft  aircraft                   57                 65  NN        O
## 13        1 13   carrier   carrier                   66                 73  NN        O
## 14        1 14        in        in                   74                 76  IN        O
## 15        1 15       the       the                   77                 80  DT        O
## 16        1 16      Gulf      Gulf                   81                 85 NNP LOCATION
## 17        1 17        of        of                   86                 88  IN LOCATION
## 18        1 18    Mexico    Mexico                   89                 95 NNP LOCATION
## 19        1 19         .         .                   95                 96   .        O
##    Speaker
## 1     PER0
## 2     PER0
## 3     PER0
## 4     PER0
## 5     PER0
## 6     PER0
## 7     PER0
## 8     PER0
## 9     PER0
## 10    PER0
## 11    PER0
## 12    PER0
## 13    PER0
## 14    PER0
## 15    PER0
## 16    PER0
## 17    PER0
## 18    PER0
## 19    PER0
\end{verbatim}
\end{kframe}
\end{knitrout}


\end{frame}
%%%%%%%%%%%%%%%%%%%%%%%%%%%%%%%%%%%%%%%%%%%%%%%%%%%%%%%%%%%%%%%%%%%%%%%%%

%%%%%%%%%%%%%%%%%%%%%%%%%%%%%%%%%%%%%%%%%%%%%%%%%%%%%%%%%%
\begin{frame}[fragile]{Other CoreNLP features}
\begin{knitrout}\tiny
\definecolor{shadecolor}{rgb}{0.969, 0.969, 0.969}\color{fgcolor}\begin{kframe}
\begin{alltt}
\hlkwd{getDependency}\hlstd{(output)}
\end{alltt}
\begin{verbatim}
##    sentence governor dependent     type governorIdx dependentIdx govIndex depIndex
## 1         1     ROOT     spoke     root           0            4       NA        4
## 2         1     Bush President compound           3            1        3        1
## 3         1     Bush    George compound           3            2        3        2
## 4         1    spoke      Bush    nsubj           4            3        4        3
## 5         1   troops        to     case           7            5        7        5
## 6         1   troops       the      det           7            6        7        6
## 7         1    spoke    troops  nmod:to           4            7        4        7
## 8         1    board        on     case           9            8        9        8
## 9         1   troops     board  nmod:on           7            9        7        9
## 10        1  carrier        of     case          13           10       13       10
## 11        1  carrier        an      det          13           11       13       11
## 12        1  carrier  aircraft compound          13           12       13       12
## 13        1    board   carrier  nmod:of           9           13        9       13
## 14        1     Gulf        in     case          16           14       16       14
## 15        1     Gulf       the      det          16           15       16       15
## 16        1    spoke      Gulf  nmod:in           4           16        4       16
## 17        1   Mexico        of     case          18           17       18       17
## 18        1     Gulf    Mexico  nmod:of          16           18       16       18
## 19        1    spoke         .    punct           4           19        4       19
\end{verbatim}
\begin{alltt}
\hlkwd{getSentiment}\hlstd{(output)}
\end{alltt}
\begin{verbatim}
##   id sentimentValue sentiment
## 1  1              1  Negative
\end{verbatim}
\end{kframe}
\end{knitrout}
\end{frame}
%%%%%%%%%%%%%%%%%%%%%%%%%%%%%%%%%%%%%%%%%%%%%%%%%%%%%%%%%%%%%%%%%%%%%%%%%


%%%%%%%%%%%%%%%%%%%%%%%%%%%%%%%%%%%%%%%%%%%%%%%%%%%%%%%%%%
\begin{frame}[fragile]{NER for Lazy People: OpenNLP Access via Web API}
\begin{knitrout}\tiny
\definecolor{shadecolor}{rgb}{0.969, 0.969, 0.969}\color{fgcolor}\begin{kframe}
\begin{alltt}
\hlstd{url} \hlkwb{<-} \hlstr{"http://www.huffingtonpost.com/entry/house-republicans-ethics_us_586bdb14e4b0de3a08f99e66?6ztihpvi"}
\hlstd{api} \hlkwb{<-} \hlstr{"http://juicer.herokuapp.com/api/article?url="}
\hlstd{target} \hlkwb{<-} \hlkwd{paste}\hlstd{(api, url,} \hlkwc{sep} \hlstd{=} \hlstr{""}\hlstd{)}

\hlstd{raw.data} \hlkwb{<-} \hlkwd{readLines}\hlstd{(target,} \hlkwc{warn} \hlstd{=} \hlstr{"F"}\hlstd{)}

\hlstd{rd} \hlkwb{<-} \hlkwd{fromJSON}\hlstd{(raw.data)}
\hlstd{dat} \hlkwb{<-} \hlstd{rd}\hlopt{$}\hlstd{article}

\hlstd{ENTITIES} \hlkwb{<-} \hlkwd{data.frame}\hlstd{(}\hlkwd{do.call}\hlstd{(}\hlstr{"rbind"}\hlstd{, dat}\hlopt{$}\hlstd{entities))}

\hlstd{ENTITIES[}\hlnum{1}\hlopt{:}\hlnum{10}\hlstd{, ]}
\end{alltt}
\begin{verbatim}
##            type                                                      text frequency
## 1        Person                                                     Trump         1
## 2  Organization Campaign for Accountability , Citizens for Responsibility         1
## 3        Person                                              Nancy Pelosi         1
## 4  Organization                                                     House         3
## 5  Organization                                           People 's House         1
## 6        Person                                                 Goodlatte         1
## 7      Location                                                Washington         2
## 8        Person                                                 Paul Ryan         1
## 9      Location                                                WASHINGTON         1
## 10       Person                                                    Pelosi         1
\end{verbatim}
\end{kframe}
\end{knitrout}

$\Rightarrow$ this works off the Stanford NLP NER module, which we will also directly use in R. 

\end{frame}
%%%%%%%%%%%%%%%%%%%%%%%%%%%%%%%%%%%%%%%%%%%%%%%%%%%%%%%%%%%%%%%%%%%%%%%%%


%%%%%%%%%%%%%%%%%%%%%%%%%%%%%%%%%%%%%%%%%%%%%%%%%%%%%%%%%%
\begin{frame}[fragile]{NER for Lazy People (2): MonkeyLearn}

\begin{itemize}

\item Can leverage existing extractors made available on MonkeyLearn platform.

\item MonkeyLearn is a platform through which classifiers for very specific tasks are shared by other data scientists. 

\item They produce benchmark estimates on the performance of extractors and classifiers that allows you to compare different products.

\item This is very useful resource for evaluation of your own proto-types against other classifiers.
\end{itemize}

\end{frame}
%%%%%%%%%%%%%%%%%%%%%%%%%%%%%%%%%%%%%%%%%%%%%%%%%%%%%%%%%%%%%%%%%%%%%%%%%


%%%%%%%%%%%%%%%%%%%%%%%%%%%%%%%%%%%%%%%%%%%%%%%%%%%%%%%%%%
\begin{frame}[fragile]{NER for Lazy People (2): MonkeyLearn}

\begin{center}
\includegraphics[scale=0.4]<1>{figures/monkeylearn.png}
\includegraphics[scale=0.4]<2>{figures/monkeylearn3.png}\\

\url{http://monkeylearn.com/}
\end{center}

\end{frame}
%%%%%%%%%%%%%%%%%%%%%%%%%%%%%%%%%%%%%%%%%%%%%%%%%%%%%%%%%%%%%%%%%%%%%%%%%



%%%%%%%%%%%%%%%%%%%%%%%%%%%%%%%%%%%%%%%%%%%%%%%%%%%%%%%%%%
\begin{frame}[fragile]{NER for Lazy People (3): MonkeyLearn}
\begin{knitrout}\tiny
\definecolor{shadecolor}{rgb}{0.969, 0.969, 0.969}\color{fgcolor}\begin{kframe}
\begin{alltt}
\hlcom{# set up an account on Monkeylearn and obtain an API reference 100k calls per month are for}
\hlcom{# free devtools::install_github('ropensci/monkeylearn') set API key as environment variable}
\hlcom{# Sys.setenv(MONKEYLEARN_KEY='')}

\hlkwd{library}\hlstd{(}\hlstr{"monkeylearn"}\hlstd{)}
\hlstd{text} \hlkwb{<-} \hlstr{"In the 19th century, the major European powers had gone to great lengths to maintain a balance of power throughout Europe, resulting in the existence of a complex network of political and military 
alliances throughout the continent by 1900. These had started in 1815, with the Holy Alliancebetween Prussia, Russia, and Austria. 
Then, in October 1873, German Chancellor Otto von Bismarck negotiated the League of the Three Emperors (German: Dreikaiserbund) between the monarchs of Austria-Hungary, Russia and Germany."}
\hlstd{output} \hlkwb{<-} \hlkwd{monkeylearn_extract}\hlstd{(}\hlkwc{request} \hlstd{= text,} \hlkwc{extractor_id} \hlstd{=} \hlstr{"ex_isnnZRbS"}\hlstd{)}
\hlstd{output}
\end{alltt}
\begin{verbatim}
##   count      tag            entity                         text_md5
## 1     1 LOCATION            Europe 97b50d3cf5012f5ba4aa2d40117da521
## 2     1 LOCATION           Prussia 97b50d3cf5012f5ba4aa2d40117da521
## 3     1 LOCATION   Austria-Hungary 97b50d3cf5012f5ba4aa2d40117da521
## 4     1 LOCATION           Austria 97b50d3cf5012f5ba4aa2d40117da521
## 5     1 LOCATION           Germany 97b50d3cf5012f5ba4aa2d40117da521
## 6     1   PERSON Otto von Bismarck 97b50d3cf5012f5ba4aa2d40117da521
## 7     2 LOCATION            Russia 97b50d3cf5012f5ba4aa2d40117da521
\end{verbatim}
\end{kframe}
\end{knitrout}
\end{frame}
%%%%%%%%%%%%%%%%%%%%%%%%%%%%%%%%%%%%%%%%%%%%%%%%%%%%%%%%%%%%%%%%%%%%%%%%%

\end{document}

