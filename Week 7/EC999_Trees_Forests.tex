\documentclass{beamer}\usepackage[]{graphicx}\usepackage[]{color}
%% maxwidth is the original width if it is less than linewidth
%% otherwise use linewidth (to make sure the graphics do not exceed the margin)
\makeatletter
\def\maxwidth{ %
  \ifdim\Gin@nat@width>\linewidth
    \linewidth
  \else
    \Gin@nat@width
  \fi
}
\makeatother

\definecolor{fgcolor}{rgb}{0.345, 0.345, 0.345}
\newcommand{\hlnum}[1]{\textcolor[rgb]{0.686,0.059,0.569}{#1}}%
\newcommand{\hlstr}[1]{\textcolor[rgb]{0.192,0.494,0.8}{#1}}%
\newcommand{\hlcom}[1]{\textcolor[rgb]{0.678,0.584,0.686}{\textit{#1}}}%
\newcommand{\hlopt}[1]{\textcolor[rgb]{0,0,0}{#1}}%
\newcommand{\hlstd}[1]{\textcolor[rgb]{0.345,0.345,0.345}{#1}}%
\newcommand{\hlkwa}[1]{\textcolor[rgb]{0.161,0.373,0.58}{\textbf{#1}}}%
\newcommand{\hlkwb}[1]{\textcolor[rgb]{0.69,0.353,0.396}{#1}}%
\newcommand{\hlkwc}[1]{\textcolor[rgb]{0.333,0.667,0.333}{#1}}%
\newcommand{\hlkwd}[1]{\textcolor[rgb]{0.737,0.353,0.396}{\textbf{#1}}}%
\let\hlipl\hlkwb

\usepackage{framed}
\makeatletter
\newenvironment{kframe}{%
 \def\at@end@of@kframe{}%
 \ifinner\ifhmode%
  \def\at@end@of@kframe{\end{minipage}}%
  \begin{minipage}{\columnwidth}%
 \fi\fi%
 \def\FrameCommand##1{\hskip\@totalleftmargin \hskip-\fboxsep
 \colorbox{shadecolor}{##1}\hskip-\fboxsep
     % There is no \\@totalrightmargin, so:
     \hskip-\linewidth \hskip-\@totalleftmargin \hskip\columnwidth}%
 \MakeFramed {\advance\hsize-\width
   \@totalleftmargin\z@ \linewidth\hsize
   \@setminipage}}%
 {\par\unskip\endMakeFramed%
 \at@end@of@kframe}
\makeatother

\definecolor{shadecolor}{rgb}{.97, .97, .97}
\definecolor{messagecolor}{rgb}{0, 0, 0}
\definecolor{warningcolor}{rgb}{1, 0, 1}
\definecolor{errorcolor}{rgb}{1, 0, 0}
\newenvironment{knitrout}{}{} % an empty environment to be redefined in TeX

\usepackage{alltt}
\usetheme{default}
%\usetheme{Malmoe}

\title[EC999: Quantitative Text Analysis]{Trees and Forests} \def\newblock{\hskip .11em plus .33em minus .07em}


\def\Tiny{\fontsize{10pt}{10pt}\selectfont}
\def\smaller{\fontsize{8pt}{8pt}\selectfont}

\institute[Warwick]{University of Chicago \& University of Warwick}
\author[Thiemo Fetzer]{Thiemo Fetzer}

 \date{\today}



%changing spacing between knitr code and output
\usepackage{etoolbox} 
\makeatletter 
\preto{\@verbatim}{\topsep=0pt \partopsep=0pt } 
\makeatother
\renewenvironment{knitrout}{\setlength{\topsep}{0mm}}{}



\usepackage{natbib}
\usepackage{amsmath}
\usepackage{hyperref}
\usepackage{graphicx}
\usepackage{graphics}

\usepackage{amsfonts}
\usepackage{amssymb}
\usepackage{pdfpages}
\usepackage{natbib}
\usepackage{hyperref}
%\usepackage{enumitem}
 \usepackage{pgffor}
\usepackage{booktabs,caption,fixltx2e}
\usepackage[flushleft]{threeparttable}
\usepackage{verbatim} 
\usepackage{cancel}
\newcommand\xxcancel[1]{\xcancel{#1}\vphantom{#1}}
\newcommand{\code}{\texttt}

\usepackage{mathtools,xparse}
 

\setbeamersize{text margin left = 16pt, text margin right = 16pt}


\newenvironment{Description}
               {\list{}{\labelwidth=0pt \itemindent-\leftmargin
                        \let\makelabel\Descriptionlabel
                        % or whatever
               }}
               {\endlist}
\newcommand*\Descriptionlabel[1]{%
  \hspace\labelsep
  \normalfont%  reset current font setting
  \color{blue}\bfseries\sffamily% or whatever 
  #1}


\newenvironment<>{algorithm}[1][\undefined]{%
\begin{actionenv}#2%
\ifx#1\undefined%
   \def\insertblocktitle{Algorithm}%
\else%
   \def\insertblocktitle{Algorithm ({\em#1})}%
\fi%
\par%
\mode<presentation>{%
  \setbeamercolor{block title}{fg=white,bg=yellow!50!black}
  \setbeamercolor{block body}{fg=black,bg=yellow!20}
}%
\usebeamertemplate{block begin}\em}
{\par\usebeamertemplate{block end}\end{actionenv}}
\IfFileExists{upquote.sty}{\usepackage{upquote}}{}
\begin{document}



\AtBeginSection[]
{
 \begin{frame}<beamer>
 \frametitle{Plan}
 \tableofcontents[currentsection]
 \end{frame}
}
\maketitle
 

%%%%%%%%%%%%%%%%%%%%%%%%%%%%


%%%%%%%%%%%%%%%%%%%%%%%%%%%%%%%%%%%%%%%%%%%%%%%%%%%%%%%%%%
\begin{frame}{Regression Trees}

\begin{itemize}

\item Linear regression is a global model, where there is a single predictive formula
holding over the entire data-space. Very difficult to capture or adequately model non-linearities, e.g. interactions.  

\item The idea of Regression Trees is to sub-divide, or partition, the
feature space into smaller regions, where the interactions are explicit.  

\item Divide the feature space, that is the $\mathbf{X}$ into  $J$ distinct regions that do not overlap, call them $R_1,...,R_J$. 
\item For every test observation $x_j$ that falls into a region $R_i$, the prediction $\hat{f}(x_j)$ is simply the mean of the training observations $y_i$ that fall into the region $R_i$

$$\hat{y}_{R_j} = 1/n_j \sum_{i \in R_j}{y_i}$$

\item[] where $n_j$ is number of training observations that fall into $R_j$.
\end{itemize}
 
\end{frame}
%%%%%%%%%%%%%%%%%%%%%%%%%%%%%%%%%%%%%%%%%%%%%%%%%%%%%%%%%%


%%%%%%%%%%%%%%%%%%%%%%%%%%%%%%%%%%%%%%%%%%%%%%%%%%%%%%%%%%
\begin{frame}[fragile,shrink]{Regression Trees: Hedonic Pricing Example}

\begin{knitrout}\tiny
\definecolor{shadecolor}{rgb}{0.969, 0.969, 0.969}\color{fgcolor}\begin{kframe}
\begin{alltt}
\hlkwd{library}\hlstd{(tree)}
\hlkwd{set.seed}\hlstd{(}\hlnum{1312}\hlstd{)}
\hlstd{SAMPLE}\hlkwb{<-}\hlkwd{sample}\hlstd{(}\hlnum{1}\hlopt{:}\hlkwd{nrow}\hlstd{(HOUSES),}\hlnum{18000}\hlstd{)}
\hlstd{TRAINING}\hlkwb{<-}\hlstd{HOUSES[SAMPLE]}
\hlstd{res}\hlkwb{<-}\hlkwd{tree}\hlstd{(soldprice}\hlopt{~}\hlstd{builtyryr}\hlopt{+}\hlstd{baths,} \hlkwc{data}\hlstd{=TRAINING)}
\hlkwd{plot}\hlstd{(res)}
\hlkwd{text}\hlstd{(res)}
\end{alltt}
\end{kframe}

{\centering \includegraphics[width=\maxwidth]{figures/knitr-treehouses-1} 

}



\end{knitrout}

\end{frame}
%%%%%%%%%%%%%%%%%%%%%%%%%%%%%%%%%%%%%%%%%%%%%%%%%%%%%%%%%%


 
 %%%%%%%%%%%%%%%%%%%%%%%%%%%%%%%%%%%%%%%%%%%%%%%%%%%%%%%%%%
\begin{frame}[fragile,shrink]{Top Down Greedy Binary Recursive Splitting}

Define two half planes as $R_1(j,s) = \{X | X_j < s \}$ and $R_2(j,s) = \{ X| X_j \geq s\}$.
 \begin{algorithm}[Binary Recursive Splitting]
   \begin{enumerate}
   \item Initialize with empty tree $\mathcal{T}$
   
   \item For each  $j=1,...,k$, find the $s$ that minimizes
      $$\min_s RSS_j = \sum_{i: x_i \in R_1(j,s)}{(y_i - \hat{y}_{R_1})^2} + \sum_{i \in R_2(j,s)}{(y_i - \hat{y}_{R_2})^2}$$
    
    \item Choose the $j$ and corresponding optimal $\tilde{s}$  at which the value of $RSS_j$ is lowest and split the data at this point into two half planes $R_1,R_2$.

    \item For each new half plane, repeat (2)-(4) so long as cutting criterion is not reached.

   \end{enumerate}
\end{algorithm}
 
\end{frame}
%%%%%%%%%%%%%%%%%%%%%%%%%%%%%%%%%%%%%%%%%%%%%%%%%%%%%%%%%%
 


%%%%%%%%%%%%%%%%%%%%%%%%%%%%%%%%%%%%%%%%%%%%%%%%%%%%%%%%%%
\begin{frame}[fragile,shrink]{Non-Linearities}

\begin{knitrout}\tiny
\definecolor{shadecolor}{rgb}{0.969, 0.969, 0.969}\color{fgcolor}\begin{kframe}
\begin{alltt}
\hlkwd{library}\hlstd{(tree)}
\hlkwd{set.seed}\hlstd{(}\hlnum{1312}\hlstd{)}
\hlstd{SAMPLE}\hlkwb{<-}\hlkwd{sample}\hlstd{(}\hlnum{1}\hlopt{:}\hlkwd{nrow}\hlstd{(HOUSES),}\hlnum{18000}\hlstd{)}
\hlstd{TRAINING}\hlkwb{<-}\hlstd{HOUSES[SAMPLE]}
\hlstd{res}\hlkwb{<-}\hlkwd{tree}\hlstd{(soldprice}\hlopt{~}\hlstd{builtyryr}\hlopt{+}\hlstd{baths,} \hlkwc{data}\hlstd{=TRAINING)}
\hlkwd{plot}\hlstd{(res)}
\hlkwd{text}\hlstd{(res)}
\end{alltt}
\end{kframe}

{\centering \includegraphics[width=\maxwidth]{figures/knitr-treehousesnonlinear-1} 

}



\end{knitrout}

\end{frame}
%%%%%%%%%%%%%%%%%%%%%%%%%%%%%%%%%%%%%%%%%%%%%%%%%%%%%%%%%%


%%%%%%%%%%%%%%%%%%%%%%%%%%%%%%%%%%%%%%%%%%%%%%%%%%%%%%%%%%
\begin{frame}{Regression Trees can generate non-linear models}

\begin{itemize}

\item In the above example, its easy to show that we have interactive effects.

\item We can, in fact, write the tree as a sequence of and interaction between different indicator functions. We have $X_1$ = Bathrooms, $X_2$ = Built Year
\begin{itemize}
\item[] $I_1 = \mathbf{I}(X_1 \leq 2.375)$
\item[] $I_2 = \mathbf{I}(X_1 \leq 1.375)$
\item[] $I_3 = \mathbf{I}(X_2 \leq 1939.5)$
\item[] $I_4 = \mathbf{I}(X_2 \leq 1957.5)$
\end{itemize}
 \item Our regression tree is a model of the form:
\begin{eqnarray*}
 Y &=&  \beta_0 \times \mathbf{I}(X_1 \leq 2.375) \times  \mathbf{I}(X_2 \leq 1939.5)  \\
  & + & \beta_1 \times  \mathbf{I}(X_1 \leq 2.375) \times \mathbf{I}(X_2 > 1939.5) \times \mathbf{I}(X_1 \leq 1.375) \\ 
  & + & \beta_2 \times  \mathbf{I}(X_1 \leq 2.375) \times \mathbf{I}(X_2 > 1939.5) \times \mathbf{I}(X_1 > 1.375) \\
  & + & \beta_3 \times  \mathbf{I}(X_1 > 2.375) \times \mathbf{I}(X_2 < 1957.5) \\
  & + & \beta_4 \times  \mathbf{I}(X_1 > 2.375) \times \mathbf{I}(X_2 < 1957.5) 
\end{eqnarray*}

 \end{itemize}
\end{frame}
%%%%%%%%%%%%%%%%%%%%%%%%%%%%%%%%%%%%%%%%%%%%%%%%%%%%%%%%%%


%%%%%%%%%%%%%%%%%%%%%%%%%%%%%%%%%%%%%%%%%%%%%%%%%%%%%%%%%%
\begin{frame}[fragile,shrink]{Regression Trees: Hedonic Pricing Example}
\begin{knitrout}\tiny
\definecolor{shadecolor}{rgb}{0.969, 0.969, 0.969}\color{fgcolor}\begin{kframe}
\begin{alltt}
\hlkwd{partition.tree}\hlstd{(res)}
\hlkwd{points}\hlstd{(TRAINING[,} \hlkwd{c}\hlstd{(}\hlstr{"baths"}\hlstd{,}\hlstr{"builtyryr"}\hlstd{),}\hlkwc{with}\hlstd{=F],} \hlkwc{cex}\hlstd{=}\hlnum{.4}\hlstd{)}
\end{alltt}
\end{kframe}

{\centering \includegraphics[width=\maxwidth]{figures/knitr-partitionhouses-1} 

}



\end{knitrout}
\end{frame}
%%%%%%%%%%%%%%%%%%%%%%%%%%%%%%%%%%%%%%%%%%%%%%%%%%%%%%%%%%

%%%%%%%%%%%%%%%%%%%%%%%%%%%%%%%%%%%%%%%%%%%%%%%%%%%%%%%%%%
\begin{frame}{How to build a tree? Binary Recursive Splitting}

\begin{itemize}

\item As before, we want to minimize prediction error RSS, computed as

$$RSS = \sum_{j=1}^J \sum_{i \in R_j}(y_i - \hat{y}_{R_j})^2$$

\item There is an uncountably large ways that you could split the predictor space into    $R_1,...,R_J$ boxes.

\item We follow an algorithmic approach known as \emph{greedy recursive binary splitting}.

\item At each step, identify the variable $j$ and the split point $s$ such that the resulting reduction in training RSS is the largest.

\item Perform this step iteratively on the sub trees until you are left with a mininum number of observations in each subtree (there are alternative stopping criteria)
\end{itemize}
 
\end{frame}
%%%%%%%%%%%%%%%%%%%%%%%%%%%%%%%%%%%%%%%%%%%%%%%%%%%%%%%%%%
 


 %%%%%%%%%%%%%%%%%%%%%%%%%%%%%%%%%%%%%%%%%%%%%%%%%%%%%%%%%%
\begin{frame}{Bias vs Variance for Large Trees}

\begin{itemize}

\item In the extreme case, each terminal node will contain only a single observation.

\item This obviously results in bad prediction performance on the test set due to high variance, as we are overfitting the data.

\item So we could do two things:

\begin{itemize}

\item Require each cut to result in a significant reduction in RSS. Why is that not a good idea?

\item Build a very large tree, and then remove (or combine) regions - \textbf{tree pruning}

\end{itemize}
\end{itemize}

\end{frame}
%%%%%%%%%%%%%%%%%%%%%%%%%%%%%%%%%%%%%%%%%%%%%%%%%%%%%%%%%%
 

 %%%%%%%%%%%%%%%%%%%%%%%%%%%%%%%%%%%%%%%%%%%%%%%%%%%%%%%%%%
\begin{frame}{Pruning of Trees}

\begin{itemize}

\item Starting out from a large tree $\mathcal{T}$, we could ``prune'' the tree, i.e. remove and combine regions from the bottom and creating subtrees.

\item We want to select a subtree that has lowest test error. For any subtree, we could estimate test error using cross-validation, but this is (often) not feasible due to large number of trees (like best subset selection)

\item \emph{Cost complexity pruning} considers a sequence of trees that are a function of a tuning parameter $\alpha$.

$$\sum_{m=1}^{|T|} \sum_{x_i \in R_m} (y_i - \hat{y}_{R_m})^2 + \alpha |T|$$

\item[] $|T|$ is the number of terminal nodes in a tree, i.e. the number of boxes, $R_m$ is rectangle pertaining to terminal node $m$ and $\hat{y}_{R_m}$ is mean of training observations in $R_m$.

\end{itemize}

\end{frame}
%%%%%%%%%%%%%%%%%%%%%%%%%%%%%%%%%%%%%%%%%%%%%%%%%%%%%%%%%%
 
 
 %%%%%%%%%%%%%%%%%%%%%%%%%%%%%%%%%%%%%%%%%%%%%%%%%%%%%%%%%%
\begin{frame}[fragile,shrink]{Complex "Bushy" Tree}
\begin{knitrout}\tiny
\definecolor{shadecolor}{rgb}{0.969, 0.969, 0.969}\color{fgcolor}\begin{kframe}
\begin{alltt}
\hlstd{res}\hlkwb{<-}\hlkwd{tree}\hlstd{(soldprice}\hlopt{/}\hlnum{1000}\hlopt{~} \hlstd{sqft}\hlopt{+}\hlstd{baths}\hlopt{+}\hlstd{builtyryr}\hlopt{+}\hlstd{distgas2009}\hlopt{+}\hlstd{distgas2010}\hlopt{+}\hlstd{lat}\hlopt{+}\hlstd{lon}\hlopt{+}\hlkwd{factor}\hlstd{(NAME)}\hlopt{+}\hlkwd{factor}\hlstd{(year),} \hlkwc{data}\hlstd{=TRAINING,}
          \hlkwc{control}\hlstd{=}\hlkwd{tree.control}\hlstd{(}\hlkwd{nrow}\hlstd{(TRAINING),}\hlkwc{mincut}\hlstd{=}\hlnum{2}\hlstd{,}\hlkwc{minsize}\hlstd{=}\hlnum{5}\hlstd{,}\hlkwc{mindev}\hlstd{=}\hlnum{0.0005}\hlstd{))}
\hlkwd{plot}\hlstd{(res)}
\hlkwd{text}\hlstd{(res,} \hlkwc{cex}\hlstd{=}\hlnum{.5}\hlstd{)}
\end{alltt}
\end{kframe}

{\centering \includegraphics[width=\maxwidth]{figures/knitr-bushytree-1} 

}



\end{knitrout}
\end{frame}
%%%%%%%%%%%%%%%%%%%%%%%%%%%%%%%%%%%%%%%%%%%%%%%%%%%%%%%%%%
  
 
 


 %%%%%%%%%%%%%%%%%%%%%%%%%%%%%%%%%%%%%%%%%%%%%%%%%%%%%%%%%%
\begin{frame}[fragile,shrink]{Pruning a "Bushy" Tree}
\begin{figure}[h]
\begin{center}$
\begin{array}{c}
\includegraphics[scale=.5]<1>{figures/knitr-bushytreepruned-1.pdf} 
\includegraphics[scale=.5]<2>{figures/knitr-bushytreepruned-3.pdf} 
\includegraphics[scale=.5]<3>{figures/knitr-bushytreepruned-5.pdf} 
\includegraphics[scale=.5]<4>{figures/knitr-bushytreepruned-6.pdf} 
\end{array}$
\end{center}
\end{figure}
\end{frame}
%%%%%%%%%%%%%%%%%%%%%%%%%%%%%%%%%%%%%%%%%%%%%%%%%%%%%%%%%%
  
  
%%%%%%%%%%%%%%%%%%%%%%%%%%%%%%%%%%%%%%%%%%%%%%%%%%%%%%%%%%
\begin{frame}{Cross Validation Error, Test Error and Training Error}

\begin{itemize}

\item Hence the idea is to vary $\alpha$ and have the computer compute different subtrees, one for each value of $\alpha$.

\item As $\alpha \rightarrow 0$, we obtain the unpruned tree.

\item As $\alpha \rightarrow \infty$, we obtain a tree with a single terminal node.

\item How to decide on optimal $\alpha$? Two approaches...


\begin{itemize}

\item We can perform a \emph{validation set} approach to compute test and training error and based on the minimum point of test error, choose 

\item Alternatively, we can obtain a sequence of subtrees and then compute the K-fold cross validation error.

\end{itemize}

\item Present each in turn for our example

\end{itemize}

\end{frame}
%%%%%%%%%%%%%%%%%%%%%%%%%%%%%%%%%%%%%%%%%%%%%%%%%%%%%%%%%%
 
 
%%%%%%%%%%%%%%%%%%%%%%%%%%%%%%%%%%%%%%%%%%%%%%%%%%%%%%%%%%
\begin{frame}[fragile,shrink]{Validation Set Approach: Training and Test MSE}

\begin{knitrout}\tiny
\definecolor{shadecolor}{rgb}{0.969, 0.969, 0.969}\color{fgcolor}

{\centering \includegraphics[width=\maxwidth]{figures/knitr-treevalidationset-1} 

}



\end{knitrout}

\end{frame}
%%%%%%%%%%%%%%%%%%%%%%%%%%%%%%%%%%%%%%%%%%%%%%%%%%%%%%%%%%
 
 
 
%%%%%%%%%%%%%%%%%%%%%%%%%%%%%%%%%%%%%%%%%%%%%%%%%%%%%%%%%%
\begin{frame}[fragile,shrink]{10 Fold Cross Validation}

\begin{knitrout}\tiny
\definecolor{shadecolor}{rgb}{0.969, 0.969, 0.969}\color{fgcolor}

{\centering \includegraphics[width=\maxwidth]{figures/knitr-treecrossvalidationset-1} 

}



\end{knitrout}

\end{frame}
%%%%%%%%%%%%%%%%%%%%%%%%%%%%%%%%%%%%%%%%%%%%%%%%%%%%%%%%%%
 
 
 
 
 %%%%%%%%%%%%%%%%%%%%%%%%%%%%%%%%%%%%%%%%%%%%%%%%%%%%%%%%%%
\begin{frame}{More Common Pattern: CV, Validation Set and Training Error}
 \begin{figure}[h]
\begin{center}$
\begin{array}{c}
\includegraphics[scale=.5]<1>{figures/common-cv-training-testerror.png} 
\end{array}$
\end{center}
\end{figure}
 
 
\end{frame}
%%%%%%%%%%%%%%%%%%%%%%%%%%%%%%%%%%%%%%%%%%%%%%%%%%%%%%%%%%
 
 
 %%%%%%%%%%%%%%%%%%%%%%%%%%%%%%%%%%%%%%%%%%%%%%%%%%%%%%%%%%
\begin{frame}[fragile,shrink]{Building Regression Trees}
 \begin{algorithm}[Building Regression Trees]

  \begin{enumerate}

\item Use recursive binary splitting to grow a large tree on the training data, stopping only when each terminal node has fewer than some minimum number of observations.
\item Apply cost complexity pruning to the large tree in order to obtain a sequence of best subtrees, as a function of $\alpha$
\item Use K-fold cross-validation to choose  $\alpha$. That is, divide the training observations into K folds. For each k = 1, . . . , K:
\begin{enumerate}
\item Repeat Steps 1 and 2 on all but the kth fold of the training data. 
\item Evaluate the mean squared prediction error on the data in the left-out kth fold, as a function of  $\alpha$
\end{enumerate}

\item Average the results for each value of  $\alpha$, and pick α to minimize the average error.
\item Return the subtree from Step 2 that corresponds to the chosen value of  $\alpha$.

 \end{enumerate}
 \end{algorithm}

\end{frame}
%%%%%%%%%%%%%%%%%%%%%%%%%%%%%%%%%%%%%%%%%%%%%%%%%%%%%%%%%%
 

 %%%%%%%%%%%%%%%%%%%%%%%%%%%%%%%%%%%%%%%%%%%%%%%%%%%%%%%%%%
\begin{frame}{Regression Trees versus Linear Models}

\begin{itemize}

\item We can write a Regression Tree predictive model as

$$ f(X) = \sum_{i=1}^M{c_m \mathbf{I}(X \in R_m)}$$ 

\item The basis functions are indicator functions and $c_m$ is the estimated average of the response variable. 

\item It becomes clear from the previous examples, that regression trees are highly non-linear as they allow interactive effects between variables. How?

\item Rectangular partitioning of predictor space may be adequate, but in some cases, the partition may be non-linear.

\end{itemize}

\end{frame}
%%%%%%%%%%%%%%%%%%%%%%%%%%%%%%%%%%%%%%%%%%%%%%%%%%%%%%%%%%
 

%%%%%%%%%%%%%%%%%%%%%%%%%%%%%%%%%%%%%%%%%%%%%%%%%%%%%%%%%%
\begin{frame} {Regression Trees Pros and Cons}

\begin{itemize}
\item Trees are very easy to explain to people. In fact, they are even easier to explain than linear regression!
\item Some people believe that decision trees more closely mirror human decision-making than do the regression approaches seen so far.

\item Trees can easily handle qualitative predictors without the need to create dummy variables.

\item Unfortunately, trees generally do not have the same level of predictive accuracy as some of the other regression methods.

\item Additionally, trees can be very non-robust. In other words, a small change in the data can cause a large change in the final estimated tree.

\end{itemize}

Next, we explore Bagging and Boosting as an approach to aggregate many regression trees into Forests.
\end{frame}
%%%%%%%%%%%%%%%%%%%%%%%%%%%%%%%%%%%%%%%%%%%%%%%%%%%%%%%%%%




\section{Classification Trees}
%%%%%%%%%%%%%%%%%%%%%%%%%%%%%%%%%%%%%%%%%%%%%%%%%%%%%%%%%%
\begin{frame}{Classification Trees}

\begin{itemize}

\item The only fundamental difference between regression trees and classification trees is that our response varibale is not numeric, but categorical.


\item As such, our objective function is not to minimize RSS, but rather to minimize classification error rate or a measure of \emph{node purity}.

\item Our \emph{decision rule} on labelling a new observation will require us to label a test observation $\mathbf{x}_j$ falling into region $R_i$ as having the label $k$ if label $k$ is the most commonly occuring among the training observations falling into region $R_i$.

\end{itemize}
 
\end{frame}
%%%%%%%%%%%%%%%%%%%%%%%%%%%%%%%%%%%%%%%%%%%%%%%%%%%%%%%%%%


%%%%%%%%%%%%%%%%%%%%%%%%%%%%%%%%%%%%%%%%%%%%%%%%%%%%%%%%%%
\begin{frame}{Objective Function}

\begin{itemize}

\item That is, we assign to a new observation the \textbf{modal} class of our training data that falls in region $R_i$

\item What is our expected "misclassification error rate" (our alternative to RSS)?

$$ E = 1 - \max_{k}( \hat{p}_{jk})$$

\item[] where $\hat{p}_{jk}$ represents the proportion of training observations in the $j$-th region that are from the $k$th class.

\end{itemize}
 
\end{frame}
%%%%%%%%%%%%%%%%%%%%%%%%%%%%%%%%%%%%%%%%%%%%%%%%%%%%%%%%%%

%%%%%%%%%%%%%%%%%%%%%%%%%%%%%%%%%%%%%%%%%%%%%%%%%%%%%%%%%%
\begin{frame}{Alternative Objective Function}

\begin{itemize}

\item It turns out that misclassification error rate is often not preferred as it produces splits with high node impurity.

\item Rather than using misclassification error as a measure of quality of a particular split, there are two alternatives, in particular the \emph{Gini Index}

$$ G  = \sum_{k=1}^{K} \hat{p}_{jk} (1-\hat{p}_{jk})$$

\item as a measure for total variance of a split region $j$, the \emph{Gini index} takes on a small value if all of the $\hat{p}_{jk}$'s are close to zero or one. [see if there is a single class, this is a bell shaped curve]

\item Its clear that this captures a notion of \emph{node purity} - a small value of $G$ indicates that a node $j$ contains predominantly observations from a single class.

\end{itemize}
 
\end{frame}
%%%%%%%%%%%%%%%%%%%%%%%%%%%%%%%%%%%%%%%%%%%%%%%%%%%%%%%%%%

 
%%%%%%%%%%%%%%%%%%%%%%%%%%%%%%%%%%%%%%%%%%%%%%%%%%%%%%%%%%
\begin{frame}{Alternative Objective Function (2)}

\begin{itemize}

\item An alternative to the Gini index is \emph{cross-entropy}, given by

$$ D  = - \sum_{k=1}^{K} \hat{p}_{jk} \log{\hat{p}_{jk}}$$

\item Using L'Hospital rule, one can show that $\lim_{p \rightarrow 0} D \rightarrow 0$ and  $\lim_{p \rightarrow 1} D \rightarrow 0$ 

\item Therefore, like the Gini index, the cross-entropy will take on a small value if the $j$-th node is pure. In fact, it turns out that the Gini index and the cross-entropy are quite similar numerically.

\end{itemize}
 
\end{frame}
%%%%%%%%%%%%%%%%%%%%%%%%%%%%%%%%%%%%%%%%%%%%%%%%%%%%%%%%%%

 
 
%%%%%%%%%%%%%%%%%%%%%%%%%%%%%%%%%%%%%%%%%%%%%%%%%%%%%%%%%%
\begin{frame}[fragile,shrink]{Spam Classification Example}
\begin{knitrout}\tiny
\definecolor{shadecolor}{rgb}{0.969, 0.969, 0.969}\color{fgcolor}\begin{kframe}
\begin{alltt}
\hlkwd{library}\hlstd{(tree)}
 \hlstd{email} \hlkwb{<-} \hlkwd{read.csv}\hlstd{(}\hlstr{"R/spam.csv"}\hlstd{)}
\hlstd{email}\hlopt{$}\hlstd{spam} \hlkwb{<-} \hlkwd{factor}\hlstd{(email}\hlopt{$}\hlstd{spam,}\hlkwc{levels}\hlstd{=}\hlkwd{c}\hlstd{(}\hlnum{0}\hlstd{,}\hlnum{1}\hlstd{),}\hlkwc{labels}\hlstd{=}\hlkwd{c}\hlstd{(}\hlstr{"important"}\hlstd{,}\hlstr{"spam"}\hlstd{))}
\hlstd{validation}\hlkwb{<-}\hlkwd{sample}\hlstd{(}\hlnum{1}\hlopt{:}\hlkwd{nrow}\hlstd{(email),} \hlnum{500}\hlstd{)}
\hlstd{res}\hlkwb{<-}\hlkwd{tree}\hlstd{(spam}\hlopt{~}\hlstd{.,} \hlkwc{data}\hlstd{=email[}\hlopt{-}\hlstd{validation])}

\hlkwd{plot}\hlstd{(res)}
\hlkwd{text}\hlstd{(res,} \hlkwc{cex}\hlstd{=}\hlnum{0.6}\hlstd{)}
\end{alltt}
\end{kframe}

{\centering \includegraphics[width=\maxwidth]{figures/knitr-decisiontree-1} 

}



\end{knitrout}
\end{frame}
%%%%%%%%%%%%%%%%%%%%%%%%%%%%%%%%%%%%%%%%%%%%%%%%%%%%%%%%%%


%%%%%%%%%%%%%%%%%%%%%%%%%%%%%%%%%%%%%%%%%%%%%%%%%%%%%%%%%%
\begin{frame}[fragile,shrink]{Test Error}
\begin{knitrout}\tiny
\definecolor{shadecolor}{rgb}{0.969, 0.969, 0.969}\color{fgcolor}\begin{kframe}
\begin{alltt}
\hlstd{pred}\hlkwb{<-}\hlkwd{predict}\hlstd{(res, email[validation,],} \hlkwc{type}\hlstd{=}\hlstr{"class"}\hlstd{)}

\hlkwd{table}\hlstd{(pred, email[validation,]}\hlopt{$}\hlstd{spam)}
\end{alltt}
\begin{verbatim}
##            
## pred        important spam
##   important       282   26
##   spam             20  172
\end{verbatim}
\begin{alltt}
\hlkwd{sum}\hlstd{(}\hlkwd{diag}\hlstd{(}\hlnum{2}\hlstd{)} \hlopt{*} \hlkwd{table}\hlstd{(pred, email[validation,]}\hlopt{$}\hlstd{spam))}\hlopt{/}\hlnum{500}
\end{alltt}
\begin{verbatim}
## [1] 0.908
\end{verbatim}
\end{kframe}
\end{knitrout}
\end{frame}
%%%%%%%%%%%%%%%%%%%%%%%%%%%%%%%%%%%%%%%%%%%%%%%%%%%%%%%%%%




%%%%%%%%%%%%%%%%%%%%%%%%%%%%%%%%%%%%%%%%%%%%%%%%%%%%%%%%%%
\begin{frame}[fragile,shrink]{Controlling the Size of the Tree}
\begin{knitrout}\tiny
\definecolor{shadecolor}{rgb}{0.969, 0.969, 0.969}\color{fgcolor}\begin{kframe}
\begin{alltt}
\hlstd{res}\hlkwb{<-}\hlkwd{tree}\hlstd{(spam}\hlopt{~}\hlstd{.,} \hlkwc{data}\hlstd{=email[}\hlopt{-}\hlstd{validation],}\hlkwc{mincut}\hlstd{=}\hlnum{2}\hlstd{,}\hlkwc{minsize}\hlstd{=}\hlnum{5}\hlstd{,}\hlkwc{mindev}\hlstd{=}\hlnum{0.0005}\hlstd{)}

\hlkwd{plot}\hlstd{(res)}
\hlkwd{text}\hlstd{(res,} \hlkwc{cex}\hlstd{=}\hlnum{0.5}\hlstd{)}
\end{alltt}
\end{kframe}

{\centering \includegraphics[width=3in]{figures/knitr-decisiontreebushy-1} 

}



\end{knitrout}
\end{frame}
%%%%%%%%%%%%%%%%%%%%%%%%%%%%%%%%%%%%%%%%%%%%%%%%%%%%%%%%%%

%%%%%%%%%%%%%%%%%%%%%%%%%%%%%%%%%%%%%%%%%%%%%%%%%%%%%%%%%%
\begin{frame}{Controlling the Size of the Tree}

The size of the tree can be chosen by a range of arguments

\code{tree.control( mincut = 5, minsize = 10, mindev = 0.01)}

\begin{Description}

\item[mincut]	 The minimum number of observations to include in either child node. This is a weighted quantity; the observational weights are used to compute the ‘number’. The default is 5.

\item[minsize] The smallest allowed node size: a weighted quantity. The default is 10.

\item[mindev]	 The within-node deviance must be at least this times that of the root node for the node to be split.
\end{Description}

To produce a tree that fits the (training) data perfectly, set \code{mindev = 0} and \code{minsize = 1}.

\end{frame}
%%%%%%%%%%%%%%%%%%%%%%%%%%%%%%%%%%%%%%%%%%%%%%%%%%%%%%%%%%


%%%%%%%%%%%%%%%%%%%%%%%%%%%%%%%%%%%%%%%%%%%%%%%%%%%%%%%%%%
\begin{frame}[fragile]{Bias- vs Variance Trade-Off}
\begin{knitrout}\tiny
\definecolor{shadecolor}{rgb}{0.969, 0.969, 0.969}\color{fgcolor}\begin{kframe}
\begin{alltt}
\hlcom{#superbushy}
\hlstd{res}\hlkwb{<-}\hlkwd{tree}\hlstd{(spam}\hlopt{~}\hlstd{.,} \hlkwc{data}\hlstd{=email[}\hlopt{-}\hlstd{validation],}\hlkwc{minsize}\hlstd{=}\hlnum{2}\hlstd{,}\hlkwc{mindev}\hlstd{=}\hlnum{0.0001}\hlstd{)}
\hlstd{testerror}\hlkwb{<-}\hlkwa{NULL}
\hlstd{trainerror}\hlkwb{<-}\hlkwa{NULL}
\hlkwa{for}\hlstd{(a} \hlkwa{in} \hlkwd{seq}\hlstd{(}\hlnum{0}\hlstd{,}\hlnum{20}\hlstd{,}\hlnum{0.25}\hlstd{)) \{}
\hlcom{#ONLY PLOT SOME AND NOT AH}
\hlstd{prune.k}\hlkwb{<-}\hlkwd{prune.tree}\hlstd{(res,} \hlkwc{newdata}\hlstd{=email[}\hlopt{-}\hlstd{validation,],} \hlkwc{method}\hlstd{=}\hlstr{"misclass"}\hlstd{,} \hlkwc{k}\hlstd{=a)}

\hlstd{pred}\hlkwb{<-}\hlkwd{predict}\hlstd{(prune.k, email[validation,],} \hlkwc{type}\hlstd{=}\hlstr{"class"}\hlstd{)}

\hlstd{testerror}\hlkwb{<-}\hlkwd{c}\hlstd{(testerror,}\hlnum{1}\hlopt{-}\hlkwd{sum}\hlstd{(}\hlkwd{diag}\hlstd{(}\hlnum{2}\hlstd{)} \hlopt{*} \hlkwd{table}\hlstd{(pred, email[validation,]}\hlopt{$}\hlstd{spam))}\hlopt{/}\hlnum{500}\hlstd{)}
\hlstd{pred}\hlkwb{<-}\hlkwd{predict}\hlstd{(prune.k, email[}\hlopt{-}\hlstd{validation,],} \hlkwc{type}\hlstd{=}\hlstr{"class"}\hlstd{)}

\hlstd{trainerror}\hlkwb{<-}\hlkwd{c}\hlstd{(trainerror,}\hlnum{1}\hlopt{-}\hlkwd{sum}\hlstd{(}\hlkwd{diag}\hlstd{(}\hlnum{2}\hlstd{)} \hlopt{*} \hlkwd{table}\hlstd{(pred, email[}\hlopt{-}\hlstd{validation,]}\hlopt{$}\hlstd{spam))}\hlopt{/}\hlnum{4101}\hlstd{)}

\hlstd{\}}
\end{alltt}
\end{kframe}
\end{knitrout}
\end{frame}
%%%%%%%%%%%%%%%%%%%%%%%%%%%%%%%%%%%%%%%%%%%%%%%%%%%%%%%%%%
    
 
%%%%%%%%%%%%%%%%%%%%%%%%%%%%%%%%%%%%%%%%%%%%%%%%%%%%%%%%%%
\begin{frame}[fragile]{Bias- vs Variance Trade-Off}
\begin{knitrout}\tiny
\definecolor{shadecolor}{rgb}{0.969, 0.969, 0.969}\color{fgcolor}\begin{kframe}
\begin{alltt}
\hlkwd{plot}\hlstd{(}\hlkwd{seq}\hlstd{(}\hlnum{0}\hlstd{,}\hlnum{20}\hlstd{,}\hlnum{0.25}\hlstd{),testerror,} \hlkwc{type}\hlstd{=}\hlstr{"l"}\hlstd{)}
\hlkwd{lines}\hlstd{(}\hlkwd{seq}\hlstd{(}\hlnum{0}\hlstd{,}\hlnum{20}\hlstd{,}\hlnum{0.25}\hlstd{),trainerror,} \hlkwc{col}\hlstd{=}\hlstr{"red"}\hlstd{)}
\end{alltt}
\end{kframe}
\end{knitrout}

\begin{center}
\includegraphics[scale=0.4]{figures/traintesterror.png}
\end{center}
\end{frame}
%%%%%%%%%%%%%%%%%%%%%%%%%%%%%%%%%%%%%%%%%%%%%%%%%%%%%%%%%%
   
    

  
 
\section{Bagging and Random Forests}

%%%%%%%%%%%%%%%%%%%%%%%%%%%%%%%%%%%%%%%%%%%%%%%%%%%%%%%%%%
\begin{frame}{Aggregation for Variance Reduction}

\begin{itemize}

\item General result: for some observations $Z_1,...,Z_n$ that are drawn \emph{independently} and identically from a distribution with the same variance, you know that 
$$Var(1/n \sum_{i=1}^{n} Z_i) = \frac{\sigma^2}{n}$$

\item Note this only holds for $Z_i, Z_j$ being independent draws.

\item Regression Trees/ Classification Trees tend to have high variance and low bias.

\item We can reduce the variance of statistical learning methods, by taking many (independently drawn) training sets from the population, build separate classification model and then average the results.

\item This is the core idea of bootstrapping, which is particularly useful for  trees and we study it in this context.
\end{itemize}

\end{frame}
%%%%%%%%%%%%%%%%%%%%%%%%%%%%%%%%%%%%%%%%%%%%%%%%%%%%%%%%%%


%%%%%%%%%%%%%%%%%%%%%%%%%%%%%%%%%%%%%%%%%%%%%%%%%%%%%%%%%%
\begin{frame}[shrink]{Bagged Trees}

\begin{itemize}

\item Suppose you calculate $\hat{f}^1(x),...,\hat{f}^B(x)$ from $B$ different randomly selected training sets, your prediction is given as

$$\hat{f}_{avg}(x) = \frac{1}{B} \sum_{b=1}^B \hat{f}^b(x)$$

\item We sample from our set of observatins $n$ with replacement, so some observations may randomly appear multiple times.

\item We compute $B$ regression trees using $B$ bootstrapped training sets, and average the resulting predictions.

\item \emph{No need to prune} these trees. An individual tree has high variance, but low bias.

\item Averaging across these $B$ trees \emph{reduces the variance}.

\item Intuition: Variance in prediction due to individual trees fitting (non existing) patterns in the error term $\epsilon$. Each individual bootstrapped tree fits a different pattern, which washes out as we average.
\end{itemize}

\end{frame}
%%%%%%%%%%%%%%%%%%%%%%%%%%%%%%%%%%%%%%%%%%%%%%%%%%%%%%%%%%



%%%%%%%%%%%%%%%%%%%%%%%%%%%%%%%%%%%%%%%%%%%%%%%%%%%%%%%%%%
\begin{frame}{Bagged Trees and Bias Variance Tradeoff}

$$E(f(x_0) - \hat{f(x_0)})^2 = Var(\hat{f}(x_0)) + [Bias(\hat{f}(x_0))]^2 + Var(\epsilon)$$

\begin{itemize}

\item Variance in prediction due to individual trees fitting (non existing) patterns in the error term $\epsilon$, that is  $Var(\hat{f}(x_0))$ captures $ Var(\epsilon)$

\item  Each individual bootstrapped tree fits a different pattern in the noise, which washes out as we average.

\item There is a second source of variance in $ Var(\hat{f}(x_0))$, coming from the fact that, mechanically, the predictions across bootstrapped samples are correlated.

\item Random forests deals with this issue in a clever way by \emph{decorrelate} individual $\hat{f}^b$.

\end{itemize}

\end{frame}
%%%%%%%%%%%%%%%%%%%%%%%%%%%%%%%%%%%%%%%%%%%%%%%%%%%%%%%%%%



%%%%%%%%%%%%%%%%%%%%%%%%%%%%%%%%%%%%%%%%%%%%%%%%%%%%%%%%%%
\begin{frame}{Random Forests}

\begin{itemize}

\item The problem: our different bootstrapped $\hat{f}^b$'s are correlated mechanically, since they are obtained using the same set of observations. This increases the variance of our prediction.

\item Correlation arises from using the same set of $X$'s in the construction of each bootstrapped tree.

\item One way to avoid this is to try to \emph{decorrelate} individual $\hat{f}^b$'s


\item To avoid this, when building these trees, each time a split in a tree is considered, a random sample of $m$ predictors is chosen as split candidates from the full set of $p$ predictors.

\item Usually $m \approx \sqrt{p}$ is chosen.

\item This mechanically decorrelates the fitted values. 


\end{itemize}

\end{frame}
%%%%%%%%%%%%%%%%%%%%%%%%%%%%%%%%%%%%%%%%%%%%%%%%%%%%%%%%%%



%%%%%%%%%%%%%%%%%%%%%%%%%%%%%%%%%%%%%%%%%%%%%%%%%%%%%%%%%%
\begin{frame}{Random Forests}

\begin{itemize}

\item The problem: our different bootstrapped $\hat{f}^b$'s are correlated mechanically, since they are obtained using the same set of observations. This increases the variance of our prediction.

\item Correlation arises from using the same set of $X$'s in the construction of each bootstrapped tree.

\item One way to avoid this is to try to \emph{decorrelate} individual $\hat{f}^b$'s


\item To avoid this, when building these trees, each time a split in a tree is considered, a random sample of $m$ predictors is chosen as split candidates from the full set of $p$ predictors.

\item Usually $m \approx \sqrt{p}$ is chosen.

\item This mechanically decorrelates the fitted values. 
\end{itemize}

\end{frame}
%%%%%%%%%%%%%%%%%%%%%%%%%%%%%%%%%%%%%%%%%%%%%%%%%%%%%%%%%%


%%%%%%%%%%%%%%%%%%%%%%%%%%%%%%%%%%%%%%%%%%%%%%%%%%%%%%%%%%
\begin{frame}[fragile]{Random Forest Illustration}
\begin{knitrout}\tiny
\definecolor{shadecolor}{rgb}{0.969, 0.969, 0.969}\color{fgcolor}\begin{kframe}
\begin{alltt}
\hlkwd{library}\hlstd{(randomForest)}
\end{alltt}


{\ttfamily\noindent\itshape\color{messagecolor}{\#\# randomForest 4.6-12}}

{\ttfamily\noindent\itshape\color{messagecolor}{\#\# Type rfNews() to see new features/changes/bug fixes.}}

{\ttfamily\noindent\itshape\color{messagecolor}{\#\# \\\#\# Attaching package: 'randomForest'}}

{\ttfamily\noindent\itshape\color{messagecolor}{\#\# The following object is masked from 'package:ggplot2':\\\#\# \\\#\#\ \ \ \  margin}}\begin{alltt}
\hlkwd{set.seed}\hlstd{(}\hlnum{12022017}\hlstd{)}
\hlkwd{dim}\hlstd{(email)}
\end{alltt}
\begin{verbatim}
## [1] 4601   58
\end{verbatim}
\begin{alltt}
\hlstd{fit} \hlkwb{<-} \hlkwd{randomForest}\hlstd{(}\hlkwd{as.factor}\hlstd{(spam)} \hlopt{~} \hlstd{.,}
                      \hlkwc{data}\hlstd{=email[}\hlopt{-}\hlstd{validation,],}
                      \hlkwc{importance}\hlstd{=}\hlnum{TRUE}\hlstd{,}
                      \hlkwc{ntree}\hlstd{=}\hlnum{500}\hlstd{)}
\hlstd{fit}
\end{alltt}
\begin{verbatim}
## 
## Call:
##  randomForest(formula = as.factor(spam) ~ ., data = email[-validation,      ], importance = TRUE, ntree = 500) 
##                Type of random forest: classification
##                      Number of trees: 500
## No. of variables tried at each split: 7
## 
##         OOB estimate of  error rate: 4.58%
## Confusion matrix:
##           important spam class.error
## important      2416   70      0.0282
## spam            118 1497      0.0731
\end{verbatim}
\end{kframe}
\end{knitrout}
\end{frame}
%%%%%%%%%%%%%%%%%%%%%%%%%%%%%%%%%%%%%%%%%%%%%%%%%%%%%%%%%%


%%%%%%%%%%%%%%%%%%%%%%%%%%%%%%%%%%%%%%%%%%%%%%%%%%%%%%%%%%
\begin{frame}[fragile]{Random Forest Variable Importance}
\begin{knitrout}\tiny
\definecolor{shadecolor}{rgb}{0.969, 0.969, 0.969}\color{fgcolor}\begin{kframe}
\begin{alltt}
\hlcom{##five most important variables}
\hlkwd{varImpPlot}\hlstd{(fit,} \hlkwc{sort}\hlstd{=}\hlnum{TRUE}\hlstd{,} \hlkwc{n.var}\hlstd{=}\hlnum{10}\hlstd{,} \hlkwc{type}\hlstd{=}\hlnum{1}\hlstd{)}
\end{alltt}
\end{kframe}

{\centering \includegraphics[width=3in]{figures/knitr-randomforestvarimp-1} 

}



\end{knitrout}
\end{frame}
%%%%%%%%%%%%%%%%%%%%%%%%%%%%%%%%%%%%%%%%%%%%%%%%%%%%%%%%%%
      
%%%%%%%%%%%%%%%%%%%%%%%%%%%%%%%%%%%%%%%%%%%%%%%%%%%%%%%%%%
\begin{frame}[fragile]{Random Forest Variable Importance (2)}
\begin{knitrout}\tiny
\definecolor{shadecolor}{rgb}{0.969, 0.969, 0.969}\color{fgcolor}\begin{kframe}
\begin{alltt}
\hlcom{##five most important variables}
\hlkwd{varImpPlot}\hlstd{(fit,} \hlkwc{sort}\hlstd{=}\hlnum{TRUE}\hlstd{,} \hlkwc{n.var}\hlstd{=}\hlnum{10}\hlstd{,} \hlkwc{type}\hlstd{=}\hlnum{2}\hlstd{)}
\end{alltt}
\end{kframe}

{\centering \includegraphics[width=3in]{figures/knitr-randomforestvarimp2-1} 

}



\end{knitrout}
\end{frame}
%%%%%%%%%%%%%%%%%%%%%%%%%%%%%%%%%%%%%%%%%%%%%%%%%%%%%%%%%%
      

%%%%%%%%%%%%%%%%%%%%%%%%%%%%%%%%%%%%%%%%%%%%%%%%%%%%%%%%%%
\begin{frame}[fragile]{Random Forest Confusion Matrix}
\begin{knitrout}\tiny
\definecolor{shadecolor}{rgb}{0.969, 0.969, 0.969}\color{fgcolor}\begin{kframe}
\begin{alltt}
\hlstd{pred}\hlkwb{<-}\hlkwd{predict}\hlstd{(fit,} \hlkwc{newdata} \hlstd{= email[validation,],} \hlkwc{type}\hlstd{=}\hlstr{"class"}\hlstd{)}
\hlkwd{table}\hlstd{(email[validation,]}\hlopt{$}\hlstd{spam,pred)}
\end{alltt}
\begin{verbatim}
##            pred
##             important spam
##   important       292   10
##   spam             12  186
\end{verbatim}
\end{kframe}
\end{knitrout}
\end{frame}
%%%%%%%%%%%%%%%%%%%%%%%%%%%%%%%%%%%%%%%%%%%%%%%%%%%%%%%%%%
   

%%%%%%%%%%%%%%%%%%%%%%%%%%%%%%%%%%%%%%%%%%%%%%%%%%%%%%%%%%
\begin{frame}[fragile]{Random Forest / Bagging Example (2)}
\begin{knitrout}\tiny
\definecolor{shadecolor}{rgb}{0.969, 0.969, 0.969}\color{fgcolor}\begin{kframe}
\begin{alltt}
\hlstd{TRAINING}\hlkwb{<-}\hlkwd{data.table}\hlstd{(}\hlkwd{read.csv}\hlstd{(}\hlkwc{file}\hlstd{=}\hlstr{"/Users/thiemo/Dropbox/Research/Matteo and Thiemo/senna/classification-tree.csv"}\hlstd{))}
\hlstd{PERSON}\hlkwb{<-}\hlstd{TRAINING[objecttype}\hlopt{==}\hlstr{"person"} \hlopt{&} \hlstd{label1}\hlopt{!=}\hlstr{""}\hlstd{]}
\hlstd{PERSON}\hlopt{$}\hlstd{label1} \hlkwb{<-} \hlkwd{factor}\hlstd{(PERSON}\hlopt{$}\hlstd{label1)}
\hlstd{PERSON}\hlopt{$}\hlstd{label1num} \hlkwb{<-} \hlkwd{as.numeric}\hlstd{(}\hlkwd{factor}\hlstd{(PERSON}\hlopt{$}\hlstd{label1))}
\hlkwd{library}\hlstd{(tm)}
\hlkwd{library}\hlstd{(RTextTools)}
\hlkwd{set.seed}\hlstd{(}\hlnum{30012017}\hlstd{)}
\hlcom{#a validation set}
\hlstd{valid}\hlkwb{<-}\hlkwd{sample}\hlstd{(}\hlnum{1}\hlopt{:}\hlkwd{nrow}\hlstd{(PERSON),} \hlnum{500}\hlstd{)}
\hlstd{PERSON}\hlopt{$}\hlstd{validation}\hlkwb{<-} \hlnum{0}
\hlstd{PERSON[valid]}\hlopt{$}\hlstd{validation}\hlkwb{<-}\hlnum{1}
\hlstd{PERSON}\hlkwb{<-}\hlstd{PERSON[}\hlkwd{order}\hlstd{(validation)]}
\hlstd{DOC}\hlkwb{<-}\hlkwd{create_matrix}\hlstd{(}\hlkwd{c}\hlstd{(PERSON[,}\hlkwd{paste}\hlstd{(objectcleanpp,}\hlkwc{sep}\hlstd{=}\hlstr{" "}\hlstd{)]),}\hlkwc{language}\hlstd{=}\hlstr{"english"}\hlstd{,}
                   \hlkwc{removeNumbers}\hlstd{=}\hlnum{TRUE}\hlstd{,}\hlkwc{stemWords}\hlstd{=}\hlnum{TRUE}\hlstd{,}\hlkwc{removePunctuation}\hlstd{=}\hlnum{TRUE}\hlstd{,}\hlkwc{removeSparseTerms}\hlstd{=}\hlnum{0.999}\hlstd{)}
\hlstd{DOCCONT}\hlkwb{<-}\hlkwd{create_container}\hlstd{(DOC,PERSON}\hlopt{$}\hlstd{label1num,} \hlkwc{trainSize}\hlstd{=}\hlnum{1}\hlopt{:}\hlnum{1200}\hlstd{,}
                          \hlkwc{testSize}\hlstd{=}\hlnum{1201}\hlopt{:}\hlkwd{nrow}\hlstd{(PERSON),} \hlkwc{virgin}\hlstd{=}\hlnum{TRUE}\hlstd{)}
\hlstd{MOD} \hlkwb{<-} \hlkwd{train_models}\hlstd{(DOCCONT,} \hlkwc{algorithms}\hlstd{=}\hlkwd{c}\hlstd{(}\hlstr{"MAXENT"}\hlstd{,}\hlstr{"TREE"}\hlstd{,}\hlstr{"BOOSTING"}\hlstd{,}\hlstr{"BAGGING"}\hlstd{,}\hlstr{"RF"}\hlstd{))}
\hlstd{RES} \hlkwb{<-} \hlkwd{classify_models}\hlstd{(DOCCONT, MOD)}
\end{alltt}
\end{kframe}
\end{knitrout}
\end{frame}
%%%%%%%%%%%%%%%%%%%%%%%%%%%%%%%%%%%%%%%%%%%%%%%%%%%%%%%%%%
   
  
%%%%%%%%%%%%%%%%%%%%%%%%%%%%%%%%%%%%%%%%%%%%%%%%%%%%%%%%%%
\begin{frame}[fragile]{Random Forest / Bagging Example (2)}
\begin{knitrout}\tiny
\definecolor{shadecolor}{rgb}{0.969, 0.969, 0.969}\color{fgcolor}\begin{kframe}
\begin{alltt}
\hlstd{analytics} \hlkwb{<-} \hlkwd{create_analytics}\hlstd{(DOCCONT, RES)}
\hlstd{res}\hlkwb{<-}\hlkwd{data.table}\hlstd{(analytics}\hlopt{@}\hlkwc{document_summary}\hlstd{)}
\hlstd{VALID}\hlkwb{<-}\hlkwd{cbind}\hlstd{(PERSON[validation}\hlopt{==}\hlnum{1}\hlstd{],res)}

\hlcom{#confusion matrix}
\hlkwd{sum}\hlstd{(}\hlkwd{diag}\hlstd{(}\hlnum{3}\hlstd{)} \hlopt{*}\hlkwd{table}\hlstd{(VALID}\hlopt{$}\hlstd{MAXENTROPY_LABEL,VALID}\hlopt{$}\hlstd{label1))}\hlopt{/}\hlnum{500}
\end{alltt}
\begin{verbatim}
## [1] 0.896
\end{verbatim}
\begin{alltt}
\hlkwd{sum}\hlstd{(}\hlkwd{diag}\hlstd{(}\hlnum{3}\hlstd{)} \hlopt{*}\hlkwd{table}\hlstd{(VALID}\hlopt{$}\hlstd{TREE_LABEL,VALID}\hlopt{$}\hlstd{label1))}\hlopt{/}\hlnum{500}
\end{alltt}
\begin{verbatim}
## [1] 0.876
\end{verbatim}
\begin{alltt}
\hlkwd{sum}\hlstd{(}\hlkwd{diag}\hlstd{(}\hlnum{3}\hlstd{)} \hlopt{*}\hlkwd{table}\hlstd{(VALID}\hlopt{$}\hlstd{BAGGING_LABEL,VALID}\hlopt{$}\hlstd{label1))}\hlopt{/}\hlnum{500}
\end{alltt}
\begin{verbatim}
## [1] 0.904
\end{verbatim}
\begin{alltt}
\hlkwd{sum}\hlstd{(}\hlkwd{diag}\hlstd{(}\hlnum{3}\hlstd{)} \hlopt{*}\hlkwd{table}\hlstd{(VALID}\hlopt{$}\hlstd{FORESTS_LABEL,VALID}\hlopt{$}\hlstd{label1))}\hlopt{/}\hlnum{500}
\end{alltt}
\begin{verbatim}
## [1] 0.912
\end{verbatim}
\end{kframe}
\end{knitrout}
\end{frame}
%%%%%%%%%%%%%%%%%%%%%%%%%%%%%%%%%%%%%%%%%%%%%%%%%%%%%%%%%%
   

%%%%%%%%%%%%%%%%%%%%%%%%%%%%%%%%%%%%%%%%%%%%%%%%%%%%%%%%%%
\begin{frame}{Tree's applications for Classification}

\begin{itemize}

\item We see that trees by themselves may yield relatively poor performance.

\item Extensions such as Bagging or Random Forests, produce superior classification performance.

\item Yet, many perceive the models as too "black boxy" and generally, tree based methods are said to have high variance.

\item Sometimes, simpler models may be preferred to more complex ones. 

\item In case you can fit multiple models to a training data, it is worth exploring the extent to which they yield similar results.

\item For word features, the fact that Trees explicitly model interactions is very useful as even for unigram word features, we can explicitly model negations etc.

\end{itemize}

\end{frame}
%%%%%%%%%%%%%%%%%%%%%%%%%%%%%%%%%%%%%%%%%%%%%%%%%%%%%%%%%%




%%%%%%%%%%%%%%%%%%%%%%%%%%%%%%%%%%%%%%%%%%%%%%%%%%%%%%%%%%%%%%%%%%%%%%%%%
\begin{frame}[fragile]{Going back to the Brexit example}

\begin{center}
\includegraphics[scale=0.5]{figures/brexit-plot.pdf}
\end{center}

\end{frame}
%%%%%%%%%%%%%%%%%%%%%%%%%%%%%%%%%%%%%%%%%%%%%%%%%%%%%%%%%%



%%%%%%%%%%%%%%%%%%%%%%%%%%%%%%%%%%%%%%%%%%%%%%%%%%%%%%%%%%
\begin{frame}[fragile]{Going back to the Brexit example}
\begin{knitrout}\tiny
\definecolor{shadecolor}{rgb}{0.969, 0.969, 0.969}\color{fgcolor}\begin{kframe}
\begin{alltt}
\hlcom{##just look at one word feature}
\hlkwd{library}\hlstd{(tm)}
\hlkwd{library}\hlstd{(RTextTools)}
\hlkwd{set.seed}\hlstd{(}\hlnum{17052017}\hlstd{)}
\hlcom{#a validation set}
\hlstd{DOC}\hlkwb{<-}\hlkwd{create_matrix}\hlstd{(DTA}\hlopt{$}\hlstd{A1,}\hlkwc{language}\hlstd{=}\hlstr{"english"}\hlstd{,}
                   \hlkwc{removeNumbers}\hlstd{=}\hlnum{TRUE}\hlstd{,}\hlkwc{stemWords}\hlstd{=}\hlnum{TRUE}\hlstd{,}\hlkwc{removePunctuation}\hlstd{=}\hlnum{TRUE}\hlstd{,} \hlkwc{removeSparseTerms} \hlstd{=} \hlnum{0.999}\hlstd{)}
\hlstd{DOCCONT}\hlkwb{<-}\hlkwd{create_container}\hlstd{(DOC,}\hlkwd{as.numeric}\hlstd{(DTA}\hlopt{$}\hlstd{leaveeu),} \hlkwc{trainSize}\hlstd{=}\hlnum{1}\hlopt{:}\hlnum{1869}\hlstd{,}
                          \hlkwc{testSize}\hlstd{=}\hlnum{1870}\hlopt{:}\hlkwd{nrow}\hlstd{(DTA),} \hlkwc{virgin}\hlstd{=}\hlnum{TRUE}\hlstd{)}
\hlstd{MOD} \hlkwb{<-} \hlkwd{train_models}\hlstd{(DOCCONT,} \hlkwc{algorithms}\hlstd{=}\hlkwd{c}\hlstd{(}\hlstr{"TREE"}\hlstd{,}\hlstr{"BOOSTING"}\hlstd{,}\hlstr{"BAGGING"}\hlstd{,}\hlstr{"RF"}\hlstd{))}
\hlstd{RES} \hlkwb{<-} \hlkwd{classify_models}\hlstd{(DOCCONT, MOD)}

\hlstd{analytics} \hlkwb{<-} \hlkwd{create_analytics}\hlstd{(DOCCONT, RES)}
\hlstd{res}\hlkwb{<-}\hlkwd{data.table}\hlstd{(analytics}\hlopt{@}\hlkwc{document_summary}\hlstd{)}
\hlstd{VALID}\hlkwb{<-}\hlkwd{cbind}\hlstd{(DTA[}\hlnum{1870}\hlopt{:}\hlkwd{nrow}\hlstd{(DTA)],res)}

\hlcom{#confusion matrix }
\hlkwd{table}\hlstd{(VALID}\hlopt{$}\hlstd{TREE_LABEL,VALID}\hlopt{$}\hlstd{leaveeu)}\hlopt{/}\hlnum{200}
\end{alltt}
\begin{verbatim}
##    
##     leave remain
##   1 0.225  0.125
##   2 0.240  0.410
\end{verbatim}
\begin{alltt}
\hlkwd{table}\hlstd{(VALID}\hlopt{$}\hlstd{BAGGING_LABEL,VALID}\hlopt{$}\hlstd{leaveeu)}\hlopt{/}\hlnum{200}
\end{alltt}
\begin{verbatim}
##    
##     leave remain
##   1 0.205  0.130
##   2 0.260  0.405
\end{verbatim}
\begin{alltt}
\hlkwd{table}\hlstd{(VALID}\hlopt{$}\hlstd{FORESTS_LABEL,VALID}\hlopt{$}\hlstd{leaveeu)}\hlopt{/}\hlnum{200}
\end{alltt}
\begin{verbatim}
##    
##     leave remain
##   1 0.250  0.170
##   2 0.215  0.365
\end{verbatim}
\end{kframe}
\end{knitrout}
\end{frame}
%%%%%%%%%%%%%%%%%%%%%%%%%%%%%%%%%%%%%%%%%%%%%%%%%%%%%%%%%%%%%%%%%%%%%%%%%



%%%%%%%%%%%%%%%%%%%%%%%%%%%%%%%%%%%%%%%%%%%%%%%%%%%%%%%%%%
\begin{frame}[fragile]{Course Projects}

\begin{itemize}

\item In total, I received 14 submissions covering 29 students out of a class of 31

\item Group size so average group size is 2

\item I will aim to combine a couple groups due to similar coverage.

\end{itemize}

\end{frame}
%%%%%%%%%%%%%%%%%%%%%%%%%%%%%%%%%%%%%%%%%%%%%%%%%%%%%%%%%%%%%%%%%%%%%%%%%




%%%%%%%%%%%%%%%%%%%%%%%%%%%%%%%%%%%%%%%%%%%%%%%%%%%%%%%%%%
\begin{frame}[fragile]{Course Projects}



\begin{knitrout}\tiny
\definecolor{shadecolor}{rgb}{0.969, 0.969, 0.969}\color{fgcolor}\begin{kframe}
\begin{alltt}
\hlcom{##just look at one word feature}
\hlkwd{library}\hlstd{(quanteda)}
\hlkwd{library}\hlstd{(readtext)}
\hlstd{CORP}\hlkwb{<-}\hlkwd{readtext}\hlstd{(}\hlstr{"/Users/thiemo/Dropbox/Teaching/QTA/Homeworks/Projects/txt"}\hlstd{)}
\hlstd{CORP}\hlkwb{<-}\hlkwd{corpus}\hlstd{(CORP)}

\hlkwd{summary}\hlstd{(CORP)}
\end{alltt}
\begin{verbatim}
## Corpus consisting of 14 documents.
## 
##        Text Types Tokens Sentences
##   Team1.txt   289    707        27
##  Team10.txt   329    750        34
##  Team11.txt   415   1143        79
##  Team12.txt   208    474        27
##  Team13.txt   293    802        36
##  Team14.txt   142    286        16
##   Team2.txt   329    733        33
##   Team3.txt   251    637        18
##   Team4.txt   202    367        25
##   Team5.txt   275    571        21
##   Team6.txt   220    392        27
##   Team7.txt   205    395        15
##   Team8.txt   355    693        24
##   Team9.txt   358    883        32
## 
## Source:  /Users/thiemo/Dropbox/Teaching/QTA/Lectures/Week 7/* on x86_64 by thiemo
## Created: Thu May 18 15:38:51 2017
## Notes:
\end{verbatim}
\end{kframe}
\end{knitrout}

\end{frame}
%%%%%%%%%%%%%%%%%%%%%%%%%%%%%%%%%%%%%%%%%%%%%%%%%%%%%%%%%%%%%%%%%%%%%%%%%


%%%%%%%%%%%%%%%%%%%%%%%%%%%%%%%%%%%%%%%%%%%%%%%%%%%%%%%%%%
\begin{frame}[fragile]{Course Projects}

\begin{knitrout}\tiny
\definecolor{shadecolor}{rgb}{0.969, 0.969, 0.969}\color{fgcolor}

{\centering \includegraphics[width=\maxwidth]{figures/knitr-courseproject2-1} 

}



\end{knitrout}

\end{frame}
%%%%%%%%%%%%%%%%%%%%%%%%%%%%%%%%%%%%%%%%%%%%%%%%%%%%%%%%%%%%%%%%%%%%%%%%%



\end{document}
