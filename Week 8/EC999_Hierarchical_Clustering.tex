\documentclass{beamer}\usepackage[]{graphicx}\usepackage[]{color}
%% maxwidth is the original width if it is less than linewidth
%% otherwise use linewidth (to make sure the graphics do not exceed the margin)
\makeatletter
\def\maxwidth{ %
  \ifdim\Gin@nat@width>\linewidth
    \linewidth
  \else
    \Gin@nat@width
  \fi
}
\makeatother

\definecolor{fgcolor}{rgb}{0.345, 0.345, 0.345}
\newcommand{\hlnum}[1]{\textcolor[rgb]{0.686,0.059,0.569}{#1}}%
\newcommand{\hlstr}[1]{\textcolor[rgb]{0.192,0.494,0.8}{#1}}%
\newcommand{\hlcom}[1]{\textcolor[rgb]{0.678,0.584,0.686}{\textit{#1}}}%
\newcommand{\hlopt}[1]{\textcolor[rgb]{0,0,0}{#1}}%
\newcommand{\hlstd}[1]{\textcolor[rgb]{0.345,0.345,0.345}{#1}}%
\newcommand{\hlkwa}[1]{\textcolor[rgb]{0.161,0.373,0.58}{\textbf{#1}}}%
\newcommand{\hlkwb}[1]{\textcolor[rgb]{0.69,0.353,0.396}{#1}}%
\newcommand{\hlkwc}[1]{\textcolor[rgb]{0.333,0.667,0.333}{#1}}%
\newcommand{\hlkwd}[1]{\textcolor[rgb]{0.737,0.353,0.396}{\textbf{#1}}}%
\let\hlipl\hlkwb

\usepackage{framed}
\makeatletter
\newenvironment{kframe}{%
 \def\at@end@of@kframe{}%
 \ifinner\ifhmode%
  \def\at@end@of@kframe{\end{minipage}}%
  \begin{minipage}{\columnwidth}%
 \fi\fi%
 \def\FrameCommand##1{\hskip\@totalleftmargin \hskip-\fboxsep
 \colorbox{shadecolor}{##1}\hskip-\fboxsep
     % There is no \\@totalrightmargin, so:
     \hskip-\linewidth \hskip-\@totalleftmargin \hskip\columnwidth}%
 \MakeFramed {\advance\hsize-\width
   \@totalleftmargin\z@ \linewidth\hsize
   \@setminipage}}%
 {\par\unskip\endMakeFramed%
 \at@end@of@kframe}
\makeatother

\definecolor{shadecolor}{rgb}{.97, .97, .97}
\definecolor{messagecolor}{rgb}{0, 0, 0}
\definecolor{warningcolor}{rgb}{1, 0, 1}
\definecolor{errorcolor}{rgb}{1, 0, 0}
\newenvironment{knitrout}{}{} % an empty environment to be redefined in TeX

\usepackage{alltt}
\usetheme{default}
%\usetheme{Malmoe}

\title[EC994: Applications of Data Science]{EC994: Hierarchical Clustering} \def\newblock{\hskip .11em plus .33em minus .07em}


\def\Tiny{\fontsize{10pt}{10pt}\selectfont}
\def\smaller{\fontsize{8pt}{8pt}\selectfont}

\institute[Warwick]{University of Warwick}
\author[Thiemo Fetzer]{Thiemo Fetzer}

 \date{\today}

\usepackage{natbib}
\usepackage{amsmath}
\usepackage{hyperref}
\usepackage{graphicx}
\usepackage{graphics}

\usepackage{amsfonts}
\usepackage{amssymb}
\usepackage{pdfpages}
\usepackage{natbib}
\usepackage{hyperref}
%\usepackage{enumitem}
 \usepackage{pgffor}
\usepackage{booktabs,caption,fixltx2e}
\usepackage[flushleft]{threeparttable}
\usepackage{verbatim} 
\usepackage{cancel}
\newcommand\xxcancel[1]{\xcancel{#1}\vphantom{#1}}

\usepackage{mathtools,xparse}
 

\setbeamersize{text margin left = 16pt, text margin right = 16pt}

\newenvironment<>{algorithm}[1][\undefined]{%
\begin{actionenv}#2%
\ifx#1\undefined%
   \def\insertblocktitle{Algorithm}%
\else%
   \def\insertblocktitle{Algorithm ({\em#1})}%
\fi%
\par%
\mode<presentation>{%
  \setbeamercolor{block title}{fg=white,bg=yellow!50!black}
  \setbeamercolor{block body}{fg=black,bg=yellow!20}
}%
\usebeamertemplate{block begin}\em}
{\par\usebeamertemplate{block end}\end{actionenv}}


\newenvironment<>{assumption}[1][\undefined]{%
\begin{actionenv}#2%
\ifx#1\undefined%
   \def\insertblocktitle{Assumption}%
\else%
   \def\insertblocktitle{Assumption ({\em#1})}%
\fi%
\par%
\mode<presentation>{%
  \setbeamercolor{block title}{fg=white,bg=blue!50!black}
  \setbeamercolor{block body}{fg=black,bg=blue!20}
}%
\usebeamertemplate{block begin}\em}
{\par\usebeamertemplate{block end}\end{actionenv}}
\IfFileExists{upquote.sty}{\usepackage{upquote}}{}
\begin{document}


\AtBeginSection[]
{
 \begin{frame}<beamer>
 \frametitle{Plan}
 \tableofcontents[currentsection]
 \end{frame}
}
\maketitle
 

%%%%%%%%%%%%%%%%%%%%%%%%%%%%


%%%%%%%%%%%%%%%%%%%%%%%%%%%%%%%%%%%%%%%%%%%%%%%%%%%%%%%%%%
\begin{frame}[fragile]{Hierarchical Clustering}
 
\begin{itemize}

\item k-Means clustering is a very powerful tool to detect patterns some data matrix $X$. However, it has some drawbacks.

\item \textbf{Numeric features}: In the construction, we compute Euclidian distances between observations and their distinct means - we implicitly assume that the data are \emph{numeric} $\rightarrow$ k-medoids can fix this!

\item \textbf{Choice of K}: We are required to make a choice over the number of clusters in the data.

\item Hierarchical clustering is a powerful approach that can build clusters based on data dissimilarity matrices and does not require a choice of clusters $K$.

\end{itemize}
\end{frame}
%%%%%%%%%%%%%%%%%%%%%%%%%%%%%%%%%%%%%%%%%%%%%%%%%%%%%%%%%%%%%%%%%%%%%%%%%



%%%%%%%%%%%%%%%%%%%%%%%%%%%%%%%%%%%%%%%%%%%%%%%%%%%%%%%%%%
\begin{frame}[fragile]{Hierarchical Clustering Idea}
 
 We will discuss one form of hierarchical clustering in this course: Agglomerative hierarchical clustering.

\begin{itemize}

\item Start with each point in its own cluster.
\item Identify the \emph{closest} two clusters and merge them.
\item Repeat.
\item Ends when all points are in a single cluster.

\end{itemize}


\end{frame}
%%%%%%%%%%%%%%%%%%%%%%%%%%%%%%%%%%%%%%%%%%%%%%%%%%%%%%%%%%%%%%%%%%%%%%%%%



%%%%%%%%%%%%%%%%%%%%%%%%%%%%%%%%%%%%%%%%%%%%%%%%%%%%%%%%%%
\begin{frame}[fragile]{Hierarchical Clustering Illustration}
 

\begin{figure}[h]
\begin{center}$
\begin{array}{c}
\includegraphics[scale=.6]<1>{figures/hierarchical-clustering-illustration-1.png} 
\includegraphics[scale=.6]<2>{figures/hierarchical-clustering-illustration-2.png} 
\includegraphics[scale=.6]<3>{figures/hierarchical-clustering-illustration-3.png} 
\includegraphics[scale=.6]<4>{figures/hierarchical-clustering-illustration-4.png} 
\includegraphics[scale=.6]<5>{figures/hierarchical-clustering-illustration-5.png} 
\end{array}$
%\caption{}
\end{center}
\end{figure}


\end{frame}
%%%%%%%%%%%%%%%%%%%%%%%%%%%%%%%%%%%%%%%%%%%%%%%%%%%%%%%%%%%%%%%%%%%%%%%%%



%%%%%%%%%%%%%%%%%%%%%%%%%%%%%%%%%%%%%%%%%%%%%%%%%%%%%%%%%%
\begin{frame}[fragile]{Representing Hierarchical Clustering as Tree}
 
We can not plot out hierarchical clustering results in two dimensions only if there are two dimensional features. With more dimensions, it easiest to display the clustering results in form of a dendogram.

\begin{figure}[h]
\begin{center}$
\begin{array}{c}
\includegraphics[scale=.47]<1>{figures/hierarchical-clustering-illustration-6.png} 
\end{array}$
%\caption{}
\end{center}
\end{figure}


\end{frame}
%%%%%%%%%%%%%%%%%%%%%%%%%%%%%%%%%%%%%%%%%%%%%%%%%%%%%%%%%%%%%%%%%%%%%%%%%


%%%%%%%%%%%%%%%%%%%%%%%%%%%%%%%%%%%%%%%%%%%%%%%%%%%%%%%%%%
\begin{frame}{Hierarchical Clustering Algorithm}

 \begin{algorithm}[Hierarchical Clustering]
   \begin{enumerate}
   
\item  Start by considering each item as its own cluster, for $n$ clusters and calculate the $n(n − 1)/2$ pairwise distances between each of the $n$ clusters.


For $i=n, n-1,...,2$

\item Examine all pair-wise inter-cluster dissimilarities among the $i$ clusters, identify the pair of clusters $k$, $k'$ that are least dissimilar and combine them. Store the distance betweent $k$ and $k'$. 

\item Recalculate distance matrix with the remaining $i-1$ cluster. 

\item Stop when only two clusters remain. 
   \end{enumerate}
\end{algorithm}
 
\end{frame}
%%%%%%%%%%%%%%%%%%%%%%%%%%%%%%%%%%%%%%%%%%%%%%%%%%%%%%%%%%



%%%%%%%%%%%%%%%%%%%%%%%%%%%%%%%%%%%%%%%%%%%%%%%%%%%%%%%%%%
\begin{frame}[shrink]{Another example performing hierarchical clustering}
\begin{knitrout}\tiny
\definecolor{shadecolor}{rgb}{0.969, 0.969, 0.969}\color{fgcolor}

{\centering \includegraphics[width=\maxwidth]{figures/knitr-hierarchicalclustexample-1} 

}



\end{knitrout}

\end{frame}
%%%%%%%%%%%%%%%%%%%%%%%%%%%%%%%%%%%%%%%%%%%%%%%%%%%%%%%%%%




%%%%%%%%%%%%%%%%%%%%%%%%%%%%%%%%%%%%%%%%%%%%%%%%%%%%%%%%%%
\begin{frame}[fragile]{Another example performing hierarchical clustering}
\begin{knitrout}
\definecolor{shadecolor}{rgb}{0.969, 0.969, 0.969}\color{fgcolor}\begin{kframe}
\begin{verbatim}
##      A    B    C     D     E
## A 0.00 3.88 1.57 2.700 1.964
## B 3.88 0.00 3.12 2.540 2.338
## C 1.57 3.12 0.00 3.145 2.254
## D 2.70 2.54 3.14 0.000 0.891
## E 1.96 2.34 2.25 0.891 0.000
\end{verbatim}
\end{kframe}
\end{knitrout}
\end{frame}
%%%%%%%%%%%%%%%%%%%%%%%%%%%%%%%%%%%%%%%%%%%%%%%%%%%%%%%%%%


%%%%%%%%%%%%%%%%%%%%%%%%%%%%%%%%%%%%%%%%%%%%%%%%%%%%%%%%%%
\begin{frame}[fragile]{Sequence of Steps}

\begin{itemize}

\item We start with every observation being its own cluster, i.e. we are at the bottom of the tree and there are  five clusters.

\item In the first iteration, we combine combine $D$ and $E$. Their distance is 0.891, so we record this distance. Now we are left with four clusters $DE$, $A$, $B$, $C$.

\item The second step in the algorithm asks us to recompute the distance matrix, but how should we do this? 

\item There are multiple choices.

\end{itemize}

\end{frame}
%%%%%%%%%%%%%%%%%%%%%%%%%%%%%%%%%%%%%%%%%%%%%%%%%%%%%%%%%%



%%%%%%%%%%%%%%%%%%%%%%%%%%%%%%%%%%%%%%%%%%%%%%%%%%%%%%%%%%
\begin{frame}[fragile]{Different Linkage choices: Complete Linkage}


\emph{Complete Linkage}: This maximizes intercluster dissimilarity, as we compute all pairwise distances between the observations in cluster A and the observations in cluster B and record the \emph{largest} in our reduced distance matrix.

\begin{figure}[h]
\begin{center}$
\begin{array}{cc}
\includegraphics[scale=.2]<1>{figures/clust-example-different-linkage-complete.pdf} &
\includegraphics[scale=.3]<1>{figures/linkage-distance.png}
\end{array}$
%\caption{}
\end{center}
\end{figure}
\end{frame}
%%%%%%%%%%%%%%%%%%%%%%%%%%%%%%%%%%%%%%%%%%%%%%%%%%%%%%%%%%


%%%%%%%%%%%%%%%%%%%%%%%%%%%%%%%%%%%%%%%%%%%%%%%%%%%%%%%%%%
\begin{frame}[fragile]{Different Linkage choices: Single Linkage}


\emph{Single Linkage}: This leads to minimal intercluster dissimilarity, as we compute all pairwise distances between the observations in cluster A and the observations in cluster B and record the \emph{minimal} in our reduced distance matrix.

\begin{figure}[h]
\begin{center}$
\begin{array}{cc}
\includegraphics[scale=.2]<1>{figures/clust-example-different-linkage-single} &
\includegraphics[scale=.3]<1>{figures/linkage-distance.png}
\end{array}$
%\caption{}
\end{center}
\end{figure}
\end{frame}
%%%%%%%%%%%%%%%%%%%%%%%%%%%%%%%%%%%%%%%%%%%%%%%%%%%%%%%%%%


%%%%%%%%%%%%%%%%%%%%%%%%%%%%%%%%%%%%%%%%%%%%%%%%%%%%%%%%%%
\begin{frame}[fragile]{Different Linkage choices: Average Linkage}


\emph{Average Linkage}: Compute all pairwise dissimilarities between observations in cluster A and observations in cluster B and record the average of these dissimilarities.

\begin{figure}[h]
\begin{center}$
\begin{array}{cc}
\includegraphics[scale=.2]<1>{figures/clust-example-different-linkage-average} &
\includegraphics[scale=.3]<1>{figures/linkage-distance.png}
\end{array}$
%\caption{}
\end{center}
\end{figure}
\end{frame}
%%%%%%%%%%%%%%%%%%%%%%%%%%%%%%%%%%%%%%%%%%%%%%%%%%%%%%%%%%


%%%%%%%%%%%%%%%%%%%%%%%%%%%%%%%%%%%%%%%%%%%%%%%%%%%%%%%%%%
\begin{frame}[fragile]{Different Linkage choices: Centroid Linkage}


\emph{Centroid Linkage}: Compute the centroid within each cluster. 

\begin{figure}[h]
\begin{center}$
\begin{array}{cc}
\includegraphics[scale=.2]<1>{figures/clust-example-different-linkage-centroid} &
\includegraphics[scale=.3]<1>{figures/linkage-distance.png}
\end{array}$
%\caption{}
\end{center}
\end{figure}
\end{frame}
%%%%%%%%%%%%%%%%%%%%%%%%%%%%%%%%%%%%%%%%%%%%%%%%%%%%%%%%%%
 
 

%%%%%%%%%%%%%%%%%%%%%%%%%%%%%%%%%%%%%%%%%%%%%%%%%%%%%%%%%%
\begin{frame}[fragile]{Number of Clusters}


By cutting at different heights of the tree, you can control the number of clusters to consider. 

\begin{figure}[h]
\begin{center}$
\begin{array}{c}
\includegraphics[scale=.5]<1>{figures/cutting-criterion.png} 
\end{array}$
%\caption{}
\end{center}
\end{figure}
\end{frame}
%%%%%%%%%%%%%%%%%%%%%%%%%%%%%%%%%%%%%%%%%%%%%%%%%%%%%%%%%%
 
 
  

%%%%%%%%%%%%%%%%%%%%%%%%%%%%%%%%%%%%%%%%%%%%%%%%%%%%%%%%%%
\begin{frame}[fragile]{Correlation based distance metrics}
 
\begin{itemize}

\item Rather than caring about levels of distance, we may be interested in correlation based distances; for example, a point described as $(1,1)$ may be quite far away from a point $(10,10)$, but they are actually perfectly correlated if we think of them as vectors.

\item A correlation measure would take this scaling factor out and indicate that these two vectors are identical!

\item In many instances, we care about pattern correlatedness and not necessarily pure distance.

\item Last week, we have already seen one example


\end{itemize}
\end{frame}
%%%%%%%%%%%%%%%%%%%%%%%%%%%%%%%%%%%%%%%%%%%%%%%%%%%%%%%%%%%%%%%%%%%%%%%%%
 
 


%%%%%%%%%%%%%%%%%%%%%%%%%%%%%%%%%%%%%%%%%%%%%%%%%%%%%%%%%%
\begin{frame}[fragile]{Cosine Similarity: Measuring Angle between two unit length vectors}

\begin{itemize}

\item What is the length of a vector $A$ and $B$? its simply the Euclidian distance from origin, i.e. $\lVert \mathbf{y_A} \rVert$,$\lVert \mathbf{y_B} \rVert$ 

\item So the vectors $\mathbf{y'_A}=\frac{\mathbf{y_A}}{\lVert \mathbf{y_A} \rVert}$ and $\mathbf{y'_B}=\frac{\mathbf{y_B}}{\lVert \mathbf{y_B} \rVert}$ both have length 1.

\item What is the angle between the vectors $\frac{\mathbf{y_A}}{\lVert \mathbf{y_A} \rVert}$ and $\frac{\mathbf{y_B}}{\lVert \mathbf{y_B} \rVert}$?

$$\cos{(\mathbf{y_A,y_B})} ={\mathbf {y_A} \cdot \mathbf {y_B}  \over \|\mathbf {y_A} \|\|\mathbf {y_B} \|}={\frac {\sum \limits _{i=1}^{n}{y_{iA}y_{iB}}}{{\sqrt {\sum \limits _{i=1}^{n}{y_{iA}^{2}}}}{\sqrt {\sum \limits _{i=1}^{n}{y_{iB}^{2}}}}}}$$

\end{itemize}

\end{frame}
%%%%%%%%%%%%%%%%%%%%%%%%%%%%%%%%%%%%%%%%%%%%%%%%%%%%%%%%%%%%%%%%%%%%%%%%%




%%%%%%%%%%%%%%%%%%%%%%%%%%%%%%%%%%%%%%%%%%%%%%%%%%%%%%%%%%
\begin{frame}[fragile]{Mapping Legislative Influence}
 
\begin{itemize} 

\item Most proposed bills do never make it into actual law. 

\item Consider the following examples for the US:

\item  H.R. 1060 (105th): Pharmacy Compounding Act

\item S. 830 (105th): Food and Drug Administration Modernization Act of 1997 
 
\end{itemize} 
\end{frame}
%%%%%%%%%%%%%%%%%%%%%%%%%%%%%%%%%%%%%%%%%%%%%%%%%%%%%%%%%%%%%%%%%%%%%%%%%

%%%%%%%%%%%%%%%%%%%%%%%%%%%%%%%%%%%%%%%%%%%%%%%%%%%%%%%%%%
\begin{frame}[fragile]{Mapping Legislative Influence}
\begin{figure}[h]
\begin{center}$
\begin{array}{c}
\includegraphics[scale=.35]<1>{figures/food-drug-example1.png} 
\includegraphics[scale=.35]<2>{figures/food-drug-example2.png} 
\includegraphics[scale=.25]<3>{figures/fda-drug-example.png} 
\end{array}$
%\caption{}
\end{center}
\end{figure}
\end{frame}
%%%%%%%%%%%%%%%%%%%%%%%%%%%%%%%%%%%%%%%%%%%%%%%%%%%%%%%%%%%%%%%%%%%%%%%%%

      


%%%%%%%%%%%%%%%%%%%%%%%%%%%%%%%%%%%%%%%%%%%%%%%%%%%%%%%%%%
\begin{frame}[fragile]{State of the Union Clustering}
 
 
\begin{figure}[h]
\begin{center}$
\begin{array}{c}
\includegraphics[scale=.55]<1>{figures/state-of-union-1.png} 
\end{array}$
%\caption{}
\end{center}
\end{figure}
 
 
\end{frame}
%%%%%%%%%%%%%%%%%%%%%%%%%%%%%%%%%%%%%%%%%%%%%%%%%%%%%%%%%%%%%%%%%%%%%%%%%
       

%%%%%%%%%%%%%%%%%%%%%%%%%%%%%%%%%%%%%%%%%%%%%%%%%%%%%%%%%%
\begin{frame}[fragile]{Word Feature Clustering to identify collocated words}

 
\begin{figure}[h]
\begin{center}$
\begin{array}{c}
\includegraphics[scale=.6]<1>{figures/state-of-union-2.png} 
\end{array}$
%\caption{}
\end{center}
\end{figure}
 
 
\end{frame}
%%%%%%%%%%%%%%%%%%%%%%%%%%%%%%%%%%%%%%%%%%%%%%%%%%%%%%%%%%%%%%%%%%%%%%%%%
         

%%%%%%%%%%%%%%%%%%%%%%%%%%%%%%%%%%%%%%%%%%%%%%%%%%%%%%%%%%
\begin{frame}[fragile]{Some hands-on R code on clustering}
 
\begin{knitrout}\tiny
\definecolor{shadecolor}{rgb}{0.969, 0.969, 0.969}\color{fgcolor}\begin{kframe}
\begin{alltt}
\hlkwd{library}\hlstd{(cluster)}
\hlkwd{library}\hlstd{(quanteda)}
\end{alltt}


{\ttfamily\noindent\itshape\color{messagecolor}{\#\# quanteda version 0.9.9.3}}

{\ttfamily\noindent\itshape\color{messagecolor}{\#\# \\\#\# Attaching package: 'quanteda'}}

{\ttfamily\noindent\itshape\color{messagecolor}{\#\# The following object is masked from 'package:utils':\\\#\# \\\#\#\ \ \ \  View}}

{\ttfamily\noindent\itshape\color{messagecolor}{\#\# The following object is masked from 'package:base':\\\#\# \\\#\#\ \ \ \  sample}}\begin{alltt}
\hlkwd{load}\hlstd{(}\hlkwc{file} \hlstd{=} \hlstr{"../../Data/trumpstweets.rdata"}\hlstd{)}
\hlcom{# remove retweet entities remove html links}
\hlstd{tw.user.df}\hlopt{$}\hlstd{text} \hlkwb{<-} \hlkwd{gsub}\hlstd{(}\hlstr{"http([^ ]*)"}\hlstd{,} \hlstr{""}\hlstd{, tw.user.df}\hlopt{$}\hlstd{text)}
\hlstd{tw.user.df[,} \hlkwd{`:=`}\hlstd{(firsthash,} \hlkwd{str_extract}\hlstd{(text,} \hlstr{"#([^ ]*)"}\hlstd{))]}
\hlstd{tw.user.df} \hlkwb{<-} \hlstd{tw.user.df[}\hlopt{!}\hlkwd{is.na}\hlstd{(firsthash)]}

\hlstd{tw.user.df}\hlopt{$}\hlstd{text} \hlkwb{<-} \hlkwd{gsub}\hlstd{(}\hlstr{"(RT|via)((?:\textbackslash{}\textbackslash{}b\textbackslash{}\textbackslash{}W*@\textbackslash{}\textbackslash{}w+)+)"}\hlstd{,} \hlstr{""}\hlstd{, tw.user.df}\hlopt{$}\hlstd{text)}
\hlcom{# remove at people}
\hlstd{tw.user.df}\hlopt{$}\hlstd{text} \hlkwb{<-} \hlkwd{gsub}\hlstd{(}\hlstr{"@\textbackslash{}\textbackslash{}w+"}\hlstd{,} \hlstr{""}\hlstd{, tw.user.df}\hlopt{$}\hlstd{text)}
\hlcom{# remove punctuation}
\hlstd{tw.user.df}\hlopt{$}\hlstd{text} \hlkwb{<-} \hlkwd{gsub}\hlstd{(}\hlstr{"[[:punct:]]"}\hlstd{,} \hlstr{""}\hlstd{, tw.user.df}\hlopt{$}\hlstd{text)}
\hlcom{# remove numbers}
\hlstd{tw.user.df}\hlopt{$}\hlstd{text} \hlkwb{<-} \hlkwd{gsub}\hlstd{(}\hlstr{"[[:digit:]]"}\hlstd{,} \hlstr{""}\hlstd{, tw.user.df}\hlopt{$}\hlstd{text)}
\hlcom{# remove unnecessary spaces}
\hlstd{tw.user.df}\hlopt{$}\hlstd{text} \hlkwb{<-} \hlkwd{gsub}\hlstd{(}\hlstr{"[ \textbackslash{}t]\{2,\}"}\hlstd{,} \hlstr{""}\hlstd{, tw.user.df}\hlopt{$}\hlstd{text)}
\hlstd{tw.user.df}\hlopt{$}\hlstd{text} \hlkwb{<-} \hlkwd{gsub}\hlstd{(}\hlstr{"^\textbackslash{}\textbackslash{}s+|\textbackslash{}\textbackslash{}s+$"}\hlstd{,} \hlstr{""}\hlstd{, tw.user.df}\hlopt{$}\hlstd{text)}
\hlstd{tw.user.df[,} \hlkwd{`:=`}\hlstd{(docname,} \hlkwd{paste}\hlstd{(}\hlstr{"text"}\hlstd{,} \hlnum{1}\hlopt{:}\hlkwd{nrow}\hlstd{(tw.user.df),} \hlkwc{sep} \hlstd{=} \hlstr{""}\hlstd{))]}
\hlstd{trump.dfm} \hlkwb{<-} \hlkwd{dfm}\hlstd{(tw.user.df}\hlopt{$}\hlstd{text,} \hlkwc{remove} \hlstd{=} \hlkwd{stopwords}\hlstd{(),} \hlkwc{stem} \hlstd{=} \hlnum{TRUE}\hlstd{)}
\end{alltt}
\end{kframe}
\end{knitrout}

\end{frame}
%%%%%%%%%%%%%%%%%%%%%%%%%%%%%%%%%%%%%%%%%%%%%%%%%%%%%%%%%%%%%%%%%%%%%%%%%
  
  
  

%%%%%%%%%%%%%%%%%%%%%%%%%%%%%%%%%%%%%%%%%%%%%%%%%%%%%%%%%%
\begin{frame}[fragile]{Hierarchical Clustering on Trump tweets}
 
\begin{knitrout}\tiny
\definecolor{shadecolor}{rgb}{0.969, 0.969, 0.969}\color{fgcolor}\begin{kframe}
\begin{alltt}
\hlcom{## k-means clustering}

\hlkwd{set.seed}\hlstd{(}\hlnum{18022017}\hlstd{)}
\hlstd{trump.dfm.trim} \hlkwb{<-} \hlkwd{dfm_trim}\hlstd{(trump.dfm,} \hlkwc{min_count} \hlstd{=} \hlnum{25}\hlstd{,} \hlkwc{min_docfreq} \hlstd{=} \hlnum{10}\hlstd{)}
\end{alltt}


{\ttfamily\noindent\itshape\color{messagecolor}{\#\# Removing features occurring:}}

{\ttfamily\noindent\itshape\color{messagecolor}{\#\#\ \  - fewer than 25 times: 1,376}}

{\ttfamily\noindent\itshape\color{messagecolor}{\#\#\ \  - in fewer than 10 documents: 1,328}}

{\ttfamily\noindent\itshape\color{messagecolor}{\#\#\ \  Total features removed: 1,376 (98.1\%).}}\begin{alltt}
\hlstd{trump.dfm.trim} \hlkwb{<-} \hlstd{trump.dfm.trim[}\hlnum{1}\hlopt{:}\hlnum{50}\hlstd{, ]}
\hlcom{# tf function converts word count dfm to share}
\hlstd{trump.dfm.dist} \hlkwb{<-} \hlkwd{dist}\hlstd{(}\hlkwd{as.matrix}\hlstd{(}\hlkwd{dfm_weight}\hlstd{(trump.dfm.trim,} \hlstr{"frequency"}\hlstd{)))}

\hlstd{trump.dfm.clust} \hlkwb{<-} \hlkwd{hclust}\hlstd{(trump.dfm.dist)}
\hlcom{# label with document names}
\hlstd{trump.dfm.clust}\hlopt{$}\hlstd{labels} \hlkwb{<-} \hlstd{tw.user.df[docname} \hlopt \hlkwd{docnames}\hlstd{(trump.dfm.trim)]}\hlopt{$}\hlstd{firsthash}
\end{alltt}
\end{kframe}
\end{knitrout}

\end{frame}
%%%%%%%%%%%%%%%%%%%%%%%%%%%%%%%%%%%%%%%%%%%%%%%%%%%%%%%%%%%%%%%%%%%%%%%%%
  

%%%%%%%%%%%%%%%%%%%%%%%%%%%%%%%%%%%%%%%%%%%%%%%%%%%%%%%%%%
\begin{frame}[fragile,shrink]{Hierarchical Clustering on Trump tweets}
 
\begin{knitrout}\tiny
\definecolor{shadecolor}{rgb}{0.969, 0.969, 0.969}\color{fgcolor}\begin{kframe}
\begin{alltt}
\hlcom{## k-means clustering plot as a dendrogram}
\hlkwd{plot}\hlstd{(trump.dfm.clust,} \hlkwc{xlab} \hlstd{=} \hlstr{""}\hlstd{,} \hlkwc{sub} \hlstd{=} \hlstr{""}\hlstd{)}
\end{alltt}
\end{kframe}

{\centering \includegraphics[width=\maxwidth]{figures/knitr-clusteringdistance2-1} 

}



\end{knitrout}

\end{frame}
%%%%%%%%%%%%%%%%%%%%%%%%%%%%%%%%%%%%%%%%%%%%%%%%%%%%%%%%%%%%%%%%%%%%%%%%%
  
  

%%%%%%%%%%%%%%%%%%%%%%%%%%%%%%%%%%%%%%%%%%%%%%%%%%%%%%%%%%
\begin{frame}[fragile,shrink]{Hierarchical Clusterong on SOTUs}
\begin{knitrout}\tiny
\definecolor{shadecolor}{rgb}{0.969, 0.969, 0.969}\color{fgcolor}\begin{kframe}
\begin{alltt}
\hlcom{## k-means clustering}
\hlkwd{data}\hlstd{(SOTUCorpus,} \hlkwc{package} \hlstd{=} \hlstr{"quantedaData"}\hlstd{)}
\hlstd{presDfm} \hlkwb{<-} \hlkwd{dfm}\hlstd{(}\hlkwd{corpus_subset}\hlstd{(SOTUCorpus, Date} \hlopt{>} \hlkwd{as.Date}\hlstd{(}\hlstr{"1960-01-01"}\hlstd{)),} \hlkwc{verbose} \hlstd{=} \hlnum{FALSE}\hlstd{,} \hlkwc{stem} \hlstd{=} \hlnum{TRUE}\hlstd{,}
    \hlkwc{remove} \hlstd{=} \hlkwd{stopwords}\hlstd{(}\hlstr{"english"}\hlstd{),} \hlkwc{removePunct} \hlstd{=} \hlnum{TRUE}\hlstd{)}
\hlstd{presDfm} \hlkwb{<-} \hlkwd{dfm_trim}\hlstd{(presDfm,} \hlkwc{min_count} \hlstd{=} \hlnum{5}\hlstd{,} \hlkwc{min_docfreq} \hlstd{=} \hlnum{3}\hlstd{)}
\end{alltt}


{\ttfamily\noindent\itshape\color{messagecolor}{\#\# Removing features occurring:}}

{\ttfamily\noindent\itshape\color{messagecolor}{\#\#\ \  - fewer than 5 times: 5,857}}

{\ttfamily\noindent\itshape\color{messagecolor}{\#\#\ \  - in fewer than 3 documents: 5,115}}

{\ttfamily\noindent\itshape\color{messagecolor}{\#\#\ \  Total features removed: 5,908 (64.7\%).}}\begin{alltt}
\hlcom{# hierarchical clustering - get distances on normalized dfm}
\hlstd{presDistMat} \hlkwb{<-} \hlkwd{dist}\hlstd{(}\hlkwd{as.matrix}\hlstd{(}\hlkwd{dfm_weight}\hlstd{(presDfm,} \hlstr{"relFreq"}\hlstd{)))}
\hlcom{# hiarchical clustering the distance object}
\hlstd{presCluster} \hlkwb{<-} \hlkwd{hclust}\hlstd{(presDistMat)}
\hlcom{# label with document names}
\hlstd{presCluster}\hlopt{$}\hlstd{labels} \hlkwb{<-} \hlkwd{docnames}\hlstd{(presDfm)}
\hlcom{# plot as a dendrogram}
\end{alltt}
\end{kframe}
\end{knitrout}
\end{frame}
%%%%%%%%%%%%%%%%%%%%%%%%%%%%%%%%%%%%%%%%%%%%%%%%%%%%%%%%%%%%%%%%%%%%%%%%%
  
    
    

%%%%%%%%%%%%%%%%%%%%%%%%%%%%%%%%%%%%%%%%%%%%%%%%%%%%%%%%%%
\begin{frame}[fragile,shrink]{Hierarchical Clusterong on SOTUs}
\begin{knitrout}\tiny
\definecolor{shadecolor}{rgb}{0.969, 0.969, 0.969}\color{fgcolor}\begin{kframe}
\begin{alltt}
\hlkwd{plot}\hlstd{(presCluster,} \hlkwc{xlab} \hlstd{=} \hlstr{""}\hlstd{,} \hlkwc{sub} \hlstd{=} \hlstr{""}\hlstd{)}
\end{alltt}
\end{kframe}

{\centering \includegraphics[width=5in]{figures/knitr-sotuclustering2-1} 

}



\end{knitrout}
\end{frame}
%%%%%%%%%%%%%%%%%%%%%%%%%%%%%%%%%%%%%%%%%%%%%%%%%%%%%%%%%%%%%%%%%%%%%%%%%
  

%%%%%%%%%%%%%%%%%%%%%%%%%%%%%%%%%%%%%%%%%%%%%%%%%%%%%%%%%%
\begin{frame}[fragile,shrink]{Identify Bigrams}
\begin{knitrout}\tiny
\definecolor{shadecolor}{rgb}{0.969, 0.969, 0.969}\color{fgcolor}\begin{kframe}
\begin{alltt}
\hlstd{wordDfm} \hlkwb{<-} \hlkwd{dfm_sort}\hlstd{(}\hlkwd{dfm_weight}\hlstd{(presDfm,} \hlstr{"relFreq"}\hlstd{))}  \hlcom{# sort in decreasing order of total word freq}
\hlstd{wordDfm} \hlkwb{<-} \hlkwd{t}\hlstd{(wordDfm)[}\hlnum{1}\hlopt{:}\hlnum{50}\hlstd{, ]}  \hlcom{# because transposed}
\hlstd{wordDistMat} \hlkwb{<-} \hlkwd{dist}\hlstd{(wordDfm)}
\hlstd{wordCluster} \hlkwb{<-} \hlkwd{hclust}\hlstd{(wordDistMat)}
\hlkwd{plot}\hlstd{(wordCluster,} \hlkwc{xlab} \hlstd{=} \hlstr{""}\hlstd{)}
\end{alltt}
\end{kframe}

{\centering \includegraphics[width=5in]{figures/knitr-sotuclustering3-1} 

}



\end{knitrout}
\end{frame}
%%%%%%%%%%%%%%%%%%%%%%%%%%%%%%%%%%%%%%%%%%%%%%%%%%%%%%%%%%%%%%%%%%%%%%%%%
  

 

\end{document}
