\documentclass{beamer}\usepackage[]{graphicx}\usepackage[]{color}
%% maxwidth is the original width if it is less than linewidth
%% otherwise use linewidth (to make sure the graphics do not exceed the margin)
\makeatletter
\def\maxwidth{ %
  \ifdim\Gin@nat@width>\linewidth
    \linewidth
  \else
    \Gin@nat@width
  \fi
}
\makeatother

\definecolor{fgcolor}{rgb}{0.345, 0.345, 0.345}
\newcommand{\hlnum}[1]{\textcolor[rgb]{0.686,0.059,0.569}{#1}}%
\newcommand{\hlstr}[1]{\textcolor[rgb]{0.192,0.494,0.8}{#1}}%
\newcommand{\hlcom}[1]{\textcolor[rgb]{0.678,0.584,0.686}{\textit{#1}}}%
\newcommand{\hlopt}[1]{\textcolor[rgb]{0,0,0}{#1}}%
\newcommand{\hlstd}[1]{\textcolor[rgb]{0.345,0.345,0.345}{#1}}%
\newcommand{\hlkwa}[1]{\textcolor[rgb]{0.161,0.373,0.58}{\textbf{#1}}}%
\newcommand{\hlkwb}[1]{\textcolor[rgb]{0.69,0.353,0.396}{#1}}%
\newcommand{\hlkwc}[1]{\textcolor[rgb]{0.333,0.667,0.333}{#1}}%
\newcommand{\hlkwd}[1]{\textcolor[rgb]{0.737,0.353,0.396}{\textbf{#1}}}%
\let\hlipl\hlkwb

\usepackage{framed}
\makeatletter
\newenvironment{kframe}{%
 \def\at@end@of@kframe{}%
 \ifinner\ifhmode%
  \def\at@end@of@kframe{\end{minipage}}%
  \begin{minipage}{\columnwidth}%
 \fi\fi%
 \def\FrameCommand##1{\hskip\@totalleftmargin \hskip-\fboxsep
 \colorbox{shadecolor}{##1}\hskip-\fboxsep
     % There is no \\@totalrightmargin, so:
     \hskip-\linewidth \hskip-\@totalleftmargin \hskip\columnwidth}%
 \MakeFramed {\advance\hsize-\width
   \@totalleftmargin\z@ \linewidth\hsize
   \@setminipage}}%
 {\par\unskip\endMakeFramed%
 \at@end@of@kframe}
\makeatother

\definecolor{shadecolor}{rgb}{.97, .97, .97}
\definecolor{messagecolor}{rgb}{0, 0, 0}
\definecolor{warningcolor}{rgb}{1, 0, 1}
\definecolor{errorcolor}{rgb}{1, 0, 0}
\newenvironment{knitrout}{}{} % an empty environment to be redefined in TeX

\usepackage{alltt}
\usetheme{default}
%\usetheme{Malmoe}

\title[EC999: Applications of Data Science]{Principal Component Analysis} \def\newblock{\hskip .11em plus .33em minus .07em}


\def\Tiny{\fontsize{10pt}{10pt}\selectfont}
\def\smaller{\fontsize{8pt}{8pt}\selectfont}

\institute[Warwick]{University of Warwick}
\author[Thiemo Fetzer]{Thiemo Fetzer}

 \date{\today}

\usepackage{natbib}
\usepackage{amsmath}
\usepackage{hyperref}
\usepackage{graphicx}
\usepackage{graphics}

\usepackage{amsfonts}
\usepackage{amssymb}
\usepackage{pdfpages}
\usepackage{natbib}
\usepackage{hyperref}
%\usepackage{enumitem}
 \usepackage{pgffor}
\usepackage{booktabs,caption,fixltx2e}
\usepackage[flushleft]{threeparttable}
\usepackage{verbatim} 
\usepackage{cancel}
\newcommand\xxcancel[1]{\xcancel{#1}\vphantom{#1}}

\usepackage{mathtools,xparse}
 

\setbeamersize{text margin left = 16pt, text margin right = 16pt}

\newenvironment<>{algorithm}[1][\undefined]{%
\begin{actionenv}#2%
\ifx#1\undefined%
   \def\insertblocktitle{Algorithm}%
\else%
   \def\insertblocktitle{Algorithm ({\em#1})}%
\fi%
\par%
\mode<presentation>{%
  \setbeamercolor{block title}{fg=white,bg=yellow!50!black}
  \setbeamercolor{block body}{fg=black,bg=yellow!20}
}%
\usebeamertemplate{block begin}\em}
{\par\usebeamertemplate{block end}\end{actionenv}}


\newenvironment<>{assumption}[1][\undefined]{%
\begin{actionenv}#2%
\ifx#1\undefined%
   \def\insertblocktitle{Assumption}%
\else%
   \def\insertblocktitle{Assumption ({\em#1})}%
\fi%
\par%
\mode<presentation>{%
  \setbeamercolor{block title}{fg=white,bg=blue!50!black}
  \setbeamercolor{block body}{fg=black,bg=blue!20}
}%
\usebeamertemplate{block begin}\em}
{\par\usebeamertemplate{block end}\end{actionenv}}
\IfFileExists{upquote.sty}{\usepackage{upquote}}{}
\begin{document}


\AtBeginSection[]
{
 \begin{frame}<beamer>
 \frametitle{Plan}
 \tableofcontents[currentsection]
 \end{frame}
}
\maketitle
 

%%%%%%%%%%%%%%%%%%%%%%%%%%%%







%%%%%%%%%%%%%%%%%%%%%%%%%%%%%%%%%%%%%%%%%%%%%%%%%%%%%%%%%%
\begin{frame}[fragile]{Dimensionality Reduction}

The goal of dimensionality reduction or unsupervised machine learning is to summarize or simplify the data structure in a setting, where you do not have a dependent variable $y$ to run any prediction on. 


\begin{enumerate}

\item In the section on \emph{Clustering}, we talked about methods and ways to find subgroups in the data that stand out by having common attributes in the feature space $X$.

\item This last lecture, we are briefly introduction \emph{Principal Component Analysis}, which is a method of summarzing, visualizing and uncovering underlying relationships between covariates.  


\end{enumerate}


\end{frame}
%%%%%%%%%%%%%%%%%%%%%%%%%%%%%%%%%%%%%%%%%%%%%%%%%%%%%%%%%%%%%%%%%%%%%%%%%




%%%%%%%%%%%%%%%%%%%%%%%%%%%%%%%%%%%%%%%%%%%%%%%%%%%%%%%%%%
\begin{frame}[fragile]{Information overflow}

\begin{figure}[h]
\begin{center}$
\begin{array}{c}
\includegraphics[scale=.45]<1>{figures/knitr-correlogram-1.png} 
\end{array}$
%\caption{}
\end{center}
\end{figure}


\end{frame}
%%%%%%%%%%%%%%%%%%%%%%%%%%%%%%%%%%%%%%%%%%%%%%%%%%%%%%%%%%%%%%%%%%%%%%%%%



%%%%%%%%%%%%%%%%%%%%%%%%%%%%%%%%%%%%%%%%%%%%%%%%%%%%%%%%%%
\begin{frame}[fragile]{Dimensionality reduction via PCA}

\begin{itemize}

\item The way we think of information in machine learning or data science is, that a variable $X$ and a variable $Z$ carry information about one another due to their correlatedness or their covariance structure.

\item In many situations, you have information overflow. Suppose you have a data matrix with $p= 10$, then there is $\binom{p}{2}$, or 45 different pairs.

\item Principal component analysis is a method to combine variables $X_1,...,X_p$ into a set of variables $Z_1, ..., Z_m$ with $m<p$, without sacrificing much of the information contained in the original $p$ variables.

\end{itemize}

\end{frame}
%%%%%%%%%%%%%%%%%%%%%%%%%%%%%%%%%%%%%%%%%%%%%%%%%%%%%%%%%%%%%%%%%%%%%%%%%


%%%%%%%%%%%%%%%%%%%%%%%%%%%%%%%%%%%%%%%%%%%%%%%%%%%%%%%%%%
\begin{frame}[fragile]{Dimensionality reduction via PCA}

\begin{itemize}

\item The goal of PCA is to produce a low-dimensional representation of a dataset by finding a sequence of linear combbinations of the variables that have \emph{maximal variance} and are \emph{mutually uncorrelated}.

\item In addition, PCA allows you to visualize the data in different forms. We will illustrate this using our housing price data.

\end{itemize}

\end{frame}
%%%%%%%%%%%%%%%%%%%%%%%%%%%%%%%%%%%%%%%%%%%%%%%%%%%%%%%%%%%%%%%%%%%%%%%%%



%%%%%%%%%%%%%%%%%%%%%%%%%%%%%%%%%%%%%%%%%%%%%%%%%%%%%%%%%%
\begin{frame}[fragile]{Principal Component Analysis}
\begin{figure}[h]
\begin{center}$
\begin{array}{c}
\includegraphics[scale=.4]<1>{figures/pca-scatter-scaled1.png} 
\includegraphics[scale=.4]<2>{figures/pca-scatter-scaled2.png} 
\includegraphics[scale=.4]<3>{figures/pca-scatter-scaled3.png} 
\includegraphics[scale=.4]<4>{figures/pca-scatter-scaled4.png} 
\includegraphics[scale=.4]<5>{figures/pca-scatter-scaled5.png} 
\end{array}$
%\caption{}
\end{center}
\end{figure}
\end{frame}
%%%%%%%%%%%%%%%%%%%%%%%%%%%%%%%%%%%%%%%%%%%%%%%%%%%%%%%%%%%%%%%%%%%%%%%%%


%%%%%%%%%%%%%%%%%%%%%%%%%%%%%%%%%%%%%%%%%%%%%%%%%%%%%%%%%%
\begin{frame}[fragile]{Dimensionality reduction via PCA}

\begin{itemize}

\item The idea of PCA is, roughly speaking, to find the hyperplane - which is defined by a set of spanning vectors - along which the data \emph{varies the most}.

\item The requirement of \emph{varying the most} means, higher order principal components will contain less information.

\item So a bit more formally:

\end{itemize}

\end{frame}
%%%%%%%%%%%%%%%%%%%%%%%%%%%%%%%%%%%%%%%%%%%%%%%%%%%%%%%%%%%%%%%%%%%%%%%%%



%%%%%%%%%%%%%%%%%%%%%%%%%%%%%%%%%%%%%%%%%%%%%%%%%%%%%%%%%%
\begin{frame}[fragile]{Principal Component Analysis}

\begin{itemize}

\item The first principal component of a set of features
$X_1, X_2,..., X_p$ is the normalized linear combination of the features

$$Z_1 =\phi_{11} X_1 + \phi_{21} X_2 +...+ \phi_{p1}X_p$$

\item The normalization refers to the requirement that 

$$\sum_{j=1}^p \phi_{j1}^2 = 1$$.

\item So the above expression is an expression of a hyperplane (remember we saw that in the SVM section).

\item We refer to the elements $\phi_{11},..., \phi_{p1}$ as the loadings of the first principal component; together, the loadings make up the principal component loading vector.

\end{itemize}
\end{frame}
%%%%%%%%%%%%%%%%%%%%%%%%%%%%%%%%%%%%%%%%%%%%%%%%%%%%%%%%%%%%%%%%%%%%%%%%%





%%%%%%%%%%%%%%%%%%%%%%%%%%%%%%%%%%%%%%%%%%%%%%%%%%%%%%%%%%
\begin{frame}[fragile]{Finding Hyerplane along which data has maximal variance}

We can write the optimization problem as:

$$\frac{1}{n} \sum_{i=1}^n z_{i1}^2 \quad \text{subject to} \quad \sum_{j=1}^p \phi_{j1}^2 =1 $$

where

$$z_{i1} = \phi_{11} x_{i1} + \phi_{21} x_{i2} + ... + \phi_{p1} x_{ip}$$

We thus choose a vector $\phi_{1}$ to maximize the estimated sample variance $Var(Z_1)$

As with SVM, the normalization $\sum_{j=1}^p \phi_{j1}^2 =1$ ensures that we obtain a hyerplane, where the function value, as we plug in the $x_i$'s measures the orthogonal distance to that hyperplane.

\end{frame}
%%%%%%%%%%%%%%%%%%%%%%%%%%%%%%%%%%%%%%%%%%%%%%%%%%%%%%%%%%%%%%%%%%%%%%%%%


%%%%%%%%%%%%%%%%%%%%%%%%%%%%%%%%%%%%%%%%%%%%%%%%%%%%%%%%%%
\begin{frame}[fragile]{Finding higher order Principal Components}

\begin{itemize}
\item Higher order principal components need to satisfy the further constraint, that they be uncorrelated with lower order principal components.


\item The second principal component of a set of features
$X_1, X_2,..., X_p$ is the linear combination that has maximal variance among all linear combinations that are uncorrelated with $Z_1$.

\item This is identical to requiring that the loadings vector $\phi_2$ is orthogonal to the loadings vector $\phi_1$.

\item There is a degree of arbiraryness: how many principal components to compute? 

\item What is the maximum number of principal components? $\min(n-1,p)$
\end{itemize}

\end{frame}
%%%%%%%%%%%%%%%%%%%%%%%%%%%%%%%%%%%%%%%%%%%%%%%%%%%%%%%%%%%%%%%%%%%%%%%%%



%%%%%%%%%%%%%%%%%%%%%%%%%%%%%%%%%%%%%%%%%%%%%%%%%%%%%%%%%%
\begin{frame}[fragile]{Centering is important...}
 
\begin{figure}[h]
\begin{center}$
\begin{array}{c}
\includegraphics[scale=.4]<1>{figures/centering.jpg} 
\end{array}$
%\caption{}
\end{center}
\end{figure}


Typically, variables are standardized as well - i.e. the variances of individual variables are normalized to 1. Is this a problem?

No! Since we dont want to attribute high variability of some variable $X$ to the fact that some variable is measured e.g. in millions of dollars, and another is measured in cents. 
\end{frame}
%%%%%%%%%%%%%%%%%%%%%%%%%%%%%%%%%%%%%%%%%%%%%%%%%%%%%%%%%%%%%%%%%%%%%%%%%



%%%%%%%%%%%%%%%%%%%%%%%%%%%%%%%%%%%%%%%%%%%%%%%%%%%%%%%%%%
\begin{frame}[fragile]{Applying PCA to the Housing Data}
 
 
\begin{knitrout}\tiny
\definecolor{shadecolor}{rgb}{0.969, 0.969, 0.969}\color{fgcolor}\begin{kframe}
\begin{alltt}
\hlkwd{prcomp}\hlstd{(HOUSES[}\hlnum{1}\hlopt{:}\hlnum{1000}\hlstd{,}\hlkwd{c}\hlstd{(}\hlstr{"soldprice"}\hlstd{,}\hlstr{"builtyryr"}\hlstd{,}\hlstr{"bedrooms"}\hlstd{,}\hlstr{"baths"}\hlstd{,}\hlstr{"sqft"}\hlstd{),}\hlkwc{with}\hlstd{=F],} \hlkwc{scale}\hlstd{=}\hlnum{TRUE}\hlstd{)}
\end{alltt}
\begin{verbatim}
## Standard deviations:
## [1] 1.726 1.001 0.639 0.557 0.549
## 
## Rotation:
##             PC1     PC2     PC3    PC4     PC5
## soldprice 0.472 -0.3198  0.4984  0.649  0.0699
## builtyryr 0.380 -0.6389 -0.5024 -0.161 -0.4118
## bedrooms  0.396  0.5916 -0.5516  0.432 -0.0486
## baths     0.510 -0.0275 -0.0585 -0.420  0.7479
## sqft      0.465  0.3727  0.4376 -0.435 -0.5136
\end{verbatim}
\end{kframe}
\end{knitrout}
  

\end{frame}
%%%%%%%%%%%%%%%%%%%%%%%%%%%%%%%%%%%%%%%%%%%%%%%%%%%%%%%%%%%%%%%%%%%%%%%%%


%%%%%%%%%%%%%%%%%%%%%%%%%%%%%%%%%%%%%%%%%%%%%%%%%%%%%%%%%%
\begin{frame}[fragile]{Loading Vectors}
\begin{figure}[h]
\begin{center}$
\begin{array}{c}
\includegraphics[scale=.4]<1>{figures/loadingvectors.png} 
\end{array}$
%\caption{}
\end{center}
\end{figure}
\end{frame}
%%%%%%%%%%%%%%%%%%%%%%%%%%%%%%%%%%%%%%%%%%%%%%%%%%%%%%%%%%%%%%%%%%%%%%%%%


%%%%%%%%%%%%%%%%%%%%%%%%%%%%%%%%%%%%%%%%%%%%%%%%%%%%%%%%%%
\begin{frame}[fragile,shrink]{Verify whether the constraints are satisfied...}
\begin{knitrout}\tiny
\definecolor{shadecolor}{rgb}{0.969, 0.969, 0.969}\color{fgcolor}\begin{kframe}
\begin{alltt}
\hlstd{TEMP}\hlkwb{<-}\hlkwd{prcomp}\hlstd{(HOUSES[}\hlnum{1}\hlopt{:}\hlnum{1000}\hlstd{,}\hlkwd{c}\hlstd{(}\hlstr{"soldprice"}\hlstd{,}\hlstr{"builtyryr"}\hlstd{,}\hlstr{"bedrooms"}\hlstd{,}\hlstr{"baths"}\hlstd{,}\hlstr{"sqft"}\hlstd{),}\hlkwc{with}\hlstd{=F],} \hlkwc{scale}\hlstd{=}\hlnum{TRUE}\hlstd{)}
\hlcom{#loadings phi1 and phi2}
\hlstd{TEMP}\hlopt{$}\hlstd{rotation[,}\hlnum{1}\hlstd{]}
\end{alltt}
\begin{verbatim}
## soldprice builtyryr  bedrooms     baths      sqft 
##     0.472     0.380     0.396     0.510     0.465
\end{verbatim}
\begin{alltt}
\hlstd{TEMP}\hlopt{$}\hlstd{rotation[,}\hlnum{2}\hlstd{]}
\end{alltt}
\begin{verbatim}
## soldprice builtyryr  bedrooms     baths      sqft 
##   -0.3198   -0.6389    0.5916   -0.0275    0.3727
\end{verbatim}
\begin{alltt}
\hlcom{##do squared loadings add to 1?}
\hlkwd{sum}\hlstd{(TEMP}\hlopt{$}\hlstd{rotation[,}\hlnum{1}\hlstd{]}\hlopt{^}\hlnum{2}\hlstd{)}
\end{alltt}
\begin{verbatim}
## [1] 1
\end{verbatim}
\begin{alltt}
\hlcom{##what is the angle between loading vectors?}
\hlkwd{cos}\hlstd{(TEMP}\hlopt{$}\hlstd{rotation[,}\hlnum{1}\hlstd{]} \hlopt \hlstd{TEMP}\hlopt{$}\hlstd{rotation[,}\hlnum{2}\hlstd{])}
\end{alltt}
\begin{verbatim}
##      [,1]
## [1,]    1
\end{verbatim}
\end{kframe}
\end{knitrout}

Where we remember that 
$$ \theta =\cos^{-1}(\frac{\langle \phi_1,\phi_2 \rangle}{||\phi_1||_2 ||\phi_2||_2} )$$

\end{frame}
%%%%%%%%%%%%%%%%%%%%%%%%%%%%%%%%%%%%%%%%%%%%%%%%%%%%%%%%%%%%%%%%%%%%%%%%%



%%%%%%%%%%%%%%%%%%%%%%%%%%%%%%%%%%%%%%%%%%%%%%%%%%%%%%%%%%
\begin{frame}[fragile,shrink]{Visualizing Principal Components}
\begin{knitrout}\tiny
\definecolor{shadecolor}{rgb}{0.969, 0.969, 0.969}\color{fgcolor}\begin{kframe}
\begin{alltt}
\hlkwd{biplot}\hlstd{(TEMP)}
\end{alltt}
\end{kframe}

{\centering \includegraphics[width=\maxwidth]{figures/knitr-prcvisualized-1} 

}



\end{knitrout}

\end{frame}
%%%%%%%%%%%%%%%%%%%%%%%%%%%%%%%%%%%%%%%%%%%%%%%%%%%%%%%%%%%%%%%%%%%%%%%%%




%%%%%%%%%%%%%%%%%%%%%%%%%%%%%%%%%%%%%%%%%%%%%%%%%%%%%%%%%%
\begin{frame}[fragile,shrink]{Visualizing Principal Components}

\begin{itemize}

\item This is a bi-plot, which is plotting out the first two principal components.

\item The points numbered in the background are the principal component scores for individual houses, i.e. its plotting out $(z_{i1}, z_{i2})$.

\item The arrows indicate the loading vectors $(\phi_1,\phi_2)$. 

\item The length of the loading tells you about the importance of that variable for a particular principal component.

\begin{itemize}
\item The first PC puts very similar weight on the individual variables.

\item While PC2 has an almost zero loading for baths for the second principal component. 

\end{itemize}

\item Houses with large scores on the first PCA will correspond to high quality high price houses, compared to houses with negative scores. 


\end{itemize}

\end{frame}
%%%%%%%%%%%%%%%%%%%%%%%%%%%%%%%%%%%%%%%%%%%%%%%%%%%%%%%%%%%%%%%%%%%%%%%%%


%%%%%%%%%%%%%%%%%%%%%%%%%%%%%%%%%%%%%%%%%%%%%%%%%%%%%%%%%%
\begin{frame}[fragile,shrink]{Applying PCA to the Housing Data: Without Scaling.}
 
 
\begin{knitrout}\tiny
\definecolor{shadecolor}{rgb}{0.969, 0.969, 0.969}\color{fgcolor}\begin{kframe}
\begin{alltt}
\hlkwd{biplot}\hlstd{(}\hlkwd{prcomp}\hlstd{(HOUSES[}\hlnum{1}\hlopt{:}\hlnum{1000}\hlstd{,}\hlkwd{c}\hlstd{(}\hlstr{"soldprice"}\hlstd{,}\hlstr{"builtyryr"}\hlstd{,}\hlstr{"bedrooms"}\hlstd{,}\hlstr{"baths"}\hlstd{,}\hlstr{"sqft"}\hlstd{),}\hlkwc{with}\hlstd{=F],} \hlkwc{scale}\hlstd{=}\hlnum{FALSE}\hlstd{))}
\end{alltt}
\end{kframe}

{\centering \includegraphics[width=\maxwidth]{figures/knitr-prcompexamplenoscale-1} 

}



\end{knitrout}
  

\end{frame}
%%%%%%%%%%%%%%%%%%%%%%%%%%%%%%%%%%%%%%%%%%%%%%%%%%%%%%%%%%%%%%%%%%%%%%%%%




%%%%%%%%%%%%%%%%%%%%%%%%%%%%%%%%%%%%%%%%%%%%%%%%%%%%%%%%%%
\begin{frame}[fragile]{Proportion of Variance Explained}
 
The total variation in our data can be estimated as:
 
$$ \sum_{j=1}^p Var(X_j) = \sum_{j=1}^p \frac{1}{n} \sum_{i=1}^{n} x_{ij}^2$$

The m-th principal component has a total variation of

$$\frac{1}{n} \sum_{i=1}^{n} z_{im}^2$$

Remember that $z_{im} = \phi_{1m} x_{i1} + ... + \phi_{pm} x_{ip}$.

One can show that for $M$ principal components

$$  \sum_{j=1}^p Var(X_j) =  \sum_{m=1}^M Var(Z_m)$$
 
\end{frame}
%%%%%%%%%%%%%%%%%%%%%%%%%%%%%%%%%%%%%%%%%%%%%%%%%%%%%%%%%%%%%%%%%%%%%%%%%


%%%%%%%%%%%%%%%%%%%%%%%%%%%%%%%%%%%%%%%%%%%%%%%%%%%%%%%%%%
\begin{frame}[fragile]{Proportion of Variance Explained}
 
So a measure of fit may be given as:

$$\frac{\frac{1}{n} \sum_{i=1}^{n} z_{im}^2}{\sum_{j=1}^p \frac{1}{n} \sum_{i=1}^{n} x_{ij}^2} $$

Since 
$$  \sum_{j=1}^p Var(X_j) =  \sum_{m=1}^M Var(Z_m)$$

The PVEs sum to one. We sometimes display the cumulative PVEs. 

You can plot out the function value as a share that is explained by each principal component m.

\end{frame}
%%%%%%%%%%%%%%%%%%%%%%%%%%%%%%%%%%%%%%%%%%%%%%%%%%%%%%%%%%%%%%%%%%%%%%%%%



%%%%%%%%%%%%%%%%%%%%%%%%%%%%%%%%%%%%%%%%%%%%%%%%%%%%%%%%%%
\begin{frame}[fragile]{Proportion of Variance Explained}
\begin{knitrout}\tiny
\definecolor{shadecolor}{rgb}{0.969, 0.969, 0.969}\color{fgcolor}\begin{kframe}
\begin{alltt}
\hlkwd{summary}\hlstd{(TEMP)}
\end{alltt}
\begin{verbatim}
## Importance of components:
##                          PC1   PC2    PC3   PC4    PC5
## Standard deviation     1.726 1.001 0.6385 0.557 0.5488
## Proportion of Variance 0.596 0.200 0.0815 0.062 0.0602
## Cumulative Proportion  0.596 0.796 0.8778 0.940 1.0000
\end{verbatim}
\end{kframe}
\end{knitrout}
\end{frame}
%%%%%%%%%%%%%%%%%%%%%%%%%%%%%%%%%%%%%%%%%%%%%%%%%%%%%%%%%%%%%%%%%%%%%%%%%


%%%%%%%%%%%%%%%%%%%%%%%%%%%%%%%%%%%%%%%%%%%%%%%%%%%%%%%%%%
\begin{frame}[fragile, shrink]{Scree Plot of Proportion of Variance Explained}
\begin{knitrout}\tiny
\definecolor{shadecolor}{rgb}{0.969, 0.969, 0.969}\color{fgcolor}\begin{kframe}
\begin{alltt}
\hlkwd{plot}\hlstd{(}\hlkwd{summary}\hlstd{(TEMP)}\hlopt{$}\hlstd{importance[}\hlnum{2}\hlstd{,])}
\end{alltt}
\end{kframe}

{\centering \includegraphics[width=\maxwidth]{figures/knitr-prcompplotscree-1} 

}



\end{knitrout}
\end{frame}
%%%%%%%%%%%%%%%%%%%%%%%%%%%%%%%%%%%%%%%%%%%%%%%%%%%%%%%%%%%%%%%%%%%%%%%%%

%%%%%%%%%%%%%%%%%%%%%%%%%%%%%%%%%%%%%%%%%%%%%%%%%%%%%%%%%%
\begin{frame}[fragile]{Some Features of PCA}
 
 \begin{itemize}
 
 \item We already said, scaling matters  / centering is important.
 
 \item Principal Components are unique, so as opposed to K-Means clustering, you will get an identical result on any computer as long as you have the same dataset.
 
 \item The total number of principal components is bounded as $min(n-1, p)$.
 
 \item There is no simple answer to the question as to how many principal components are adequate - typically, we look for a type of scree plot.
 
 \end{itemize}
 
\end{frame}
%%%%%%%%%%%%%%%%%%%%%%%%%%%%%%%%%%%%%%%%%%%%%%%%%%%%%%%%%%%%%%%%%%%%%%%%%




%%%%%%%%%%%%%%%%%%%%%%%%%%%%%%%%%%%%%%%%%%%%%%%%%%%%%%%%%%
\begin{frame}[fragile]{Latent Semantic Analysis is PCA for term-document matrices}
 
 \begin{itemize}
 
 \item For term document matrices, singular value decomposition (SVD) is applied to any $m x n$ matrix shape.
 
 \item Latent Semantic Analysis uses signular value decomposition to factorizes a term document matrix $X$ into three parts
 
 ${\displaystyle {\begin{matrix}X=U\Sigma V^{T}\end{matrix}}} {\begin{matrix}X=U\Sigma V^{T}\end{matrix}}$

\item where $X$ is $k x n$ ($k$ terms in $n$ documents).


 \end{itemize}
 
\end{frame}
%%%%%%%%%%%%%%%%%%%%%%%%%%%%%%%%%%%%%%%%%%%%%%%%%%%%%%%%%%%%%%%%%%%%%%%%%


\end{document}

%%%%%%%%%%%%%%%%%%%%%%%%%%%%%%%%%%%%%%%%%%%%%%%%%%%%%%%%%%
\begin{frame}[fragile]{Proportion of Variance Explained}

One can show that for $M$ principal components

$$  \sum_{j=1}^p Var(X_j) =  \sum_{m=1}^M Var(Z_m)$$
 
For two principal components and two variables:

$$Var(X_1) + Var(X_2) = Var( \phi_{11} X_1 + \phi_{21} X_2) +  Var( \phi_{12} X_1 + \phi_{22} X_2) $$
$$Var(  \phi_{11} X_1 + \phi_{21} X_2) = \phi_{11}^2 Var(X_1) + \phi_{21}^2  Var(X_2) + 2 \phi_{11} \phi_{21} Cov(X_1,X_2)$$
$$Var( \phi_{12} X_1 + \phi_{22} X_2) = \phi_{12}^2 Var(X_1) + \phi_{22}^2  Var(X_2) + 2 \phi_{12} \phi_{22} Cov(X_1,X_2)$$

$$(\phi_{11}^2 + \phi_{12}^2) Var(X_1) +  (\phi_{22}^2 + \phi_{21}^2)  Var(X_2) + 2 [\phi_{11} \phi_{21} + \phi_{12} \phi_{22}] Cov(X_1,X_2) $$

\end{frame}
%%%%%%%%%%%%%%%%%%%%%%%%%%%%%%%%%%%%%%%%%%%%%%%%%%%%%%%%%%%%%%%%%%%%%%%%%


