\documentclass{beamer}\usepackage[]{graphicx}\usepackage[]{color}
%% maxwidth is the original width if it is less than linewidth
%% otherwise use linewidth (to make sure the graphics do not exceed the margin)
\makeatletter
\def\maxwidth{ %
  \ifdim\Gin@nat@width>\linewidth
    \linewidth
  \else
    \Gin@nat@width
  \fi
}
\makeatother

\definecolor{fgcolor}{rgb}{0.345, 0.345, 0.345}
\newcommand{\hlnum}[1]{\textcolor[rgb]{0.686,0.059,0.569}{#1}}%
\newcommand{\hlstr}[1]{\textcolor[rgb]{0.192,0.494,0.8}{#1}}%
\newcommand{\hlcom}[1]{\textcolor[rgb]{0.678,0.584,0.686}{\textit{#1}}}%
\newcommand{\hlopt}[1]{\textcolor[rgb]{0,0,0}{#1}}%
\newcommand{\hlstd}[1]{\textcolor[rgb]{0.345,0.345,0.345}{#1}}%
\newcommand{\hlkwa}[1]{\textcolor[rgb]{0.161,0.373,0.58}{\textbf{#1}}}%
\newcommand{\hlkwb}[1]{\textcolor[rgb]{0.69,0.353,0.396}{#1}}%
\newcommand{\hlkwc}[1]{\textcolor[rgb]{0.333,0.667,0.333}{#1}}%
\newcommand{\hlkwd}[1]{\textcolor[rgb]{0.737,0.353,0.396}{\textbf{#1}}}%
\let\hlipl\hlkwb

\usepackage{framed}
\makeatletter
\newenvironment{kframe}{%
 \def\at@end@of@kframe{}%
 \ifinner\ifhmode%
  \def\at@end@of@kframe{\end{minipage}}%
  \begin{minipage}{\columnwidth}%
 \fi\fi%
 \def\FrameCommand##1{\hskip\@totalleftmargin \hskip-\fboxsep
 \colorbox{shadecolor}{##1}\hskip-\fboxsep
     % There is no \\@totalrightmargin, so:
     \hskip-\linewidth \hskip-\@totalleftmargin \hskip\columnwidth}%
 \MakeFramed {\advance\hsize-\width
   \@totalleftmargin\z@ \linewidth\hsize
   \@setminipage}}%
 {\par\unskip\endMakeFramed%
 \at@end@of@kframe}
\makeatother

\definecolor{shadecolor}{rgb}{.97, .97, .97}
\definecolor{messagecolor}{rgb}{0, 0, 0}
\definecolor{warningcolor}{rgb}{1, 0, 1}
\definecolor{errorcolor}{rgb}{1, 0, 0}
\newenvironment{knitrout}{}{} % an empty environment to be redefined in TeX

\usepackage{alltt}
\usetheme{default}
%\usetheme{Malmoe}

\title[EC999: Quantitative Text Analysis]{EC999: K-Nearest Neighbours (KNN)} \def\newblock{\hskip .11em plus .33em minus .07em}


\def\Tiny{\fontsize{10pt}{10pt}\selectfont}
\def\smaller{\fontsize{8pt}{8pt}\selectfont}

\institute[Warwick]{University of Chicago \& University of Warwick}
\author[Thiemo Fetzer]{Thiemo Fetzer}

 \date{\today}

\usepackage{natbib}
\usepackage{amsmath}
\usepackage{hyperref}
\usepackage{graphicx}
\usepackage{graphics}

\usepackage{amsfonts}
\usepackage{amssymb}
\usepackage{pdfpages}
\usepackage{natbib}
\usepackage{hyperref}
%\usepackage{enumitem}
 \usepackage{pgffor}
\usepackage{booktabs,caption,fixltx2e}
\usepackage[flushleft]{threeparttable}
\usepackage{verbatim} 
\usepackage{cancel}
\newcommand\xxcancel[1]{\xcancel{#1}\vphantom{#1}}

\usepackage{mathtools,xparse}
 

\setbeamersize{text margin left = 16pt, text margin right = 16pt}

\newenvironment<>{algorithm}[1][\undefined]{%
\begin{actionenv}#2%
\ifx#1\undefined%
   \def\insertblocktitle{Algorithm}%
\else%
   \def\insertblocktitle{Algorithm ({\em#1})}%
\fi%
\par%
\mode<presentation>{%
  \setbeamercolor{block title}{fg=white,bg=yellow!50!black}
  \setbeamercolor{block body}{fg=black,bg=yellow!20}
}%
\usebeamertemplate{block begin}\em}
{\par\usebeamertemplate{block end}\end{actionenv}}


\newenvironment<>{assumption}[1][\undefined]{%
\begin{actionenv}#2%
\ifx#1\undefined%
   \def\insertblocktitle{Assumption}%
\else%
   \def\insertblocktitle{Assumption ({\em#1})}%
\fi%
\par%
\mode<presentation>{%
  \setbeamercolor{block title}{fg=white,bg=blue!50!black}
  \setbeamercolor{block body}{fg=black,bg=blue!20}
}%
\usebeamertemplate{block begin}\em}
{\par\usebeamertemplate{block end}\end{actionenv}}
\IfFileExists{upquote.sty}{\usepackage{upquote}}{}
\begin{document}


\AtBeginSection[]
{
 \begin{frame}<beamer>
 \frametitle{Plan}
 \tableofcontents[currentsection]
 \end{frame}
}
\maketitle
 

%%%%%%%%%%%%%%%%%%%%%%%%%%%%


%%%%%%%%%%%%%%%%%%%%%%%%%%%%%%%%%%%%%%%%%%%%%%%%%%%%%%%%%%
\begin{frame}{K Nearest Neighbours}

\begin{itemize}

\item The first classification algorithm we present is called \textbf{K nearest neighbors} or KNN.

\item KNN assigns a label to a new observation $y_i$ based on ``what is the most common class around the vector $x_i$''?

\item The idea is that, if you have similar $X's$ values, then also the label $y$ should be similar.

\item One dimensional space [draw example]


\end{itemize}

\end{frame}
%%%%%%%%%%%%%%%%%%%%%%%%%%%%%%%%%%%%%%%%%%%%%%%%%%%%%%%%%%

%%%%%%%%%%%%%%%%%%%%%%%%%%%%%%%%%%%%%%%%%%%%%%%%%%%%%%%%%%
\begin{frame}{K Nearest Neighbours}

\begin{figure}[h]
\begin{center}$
\begin{array}{c}
\includegraphics[scale=.5]<1>{figures/Data3classes-star-0.png} 
\includegraphics[scale=.375]<2>{figures/Data3classes-star-00.png} 
\includegraphics[scale=.5]<3>{figures/Data3classes-star-1.png} 
\includegraphics[scale=.5]<4>{figures/Data3classes-star-2.png} 
\includegraphics[scale=.5]<5>{figures/Data3classes-star-3.png} 
\includegraphics[scale=.4]<6>{figures/Data3classes-star-4.png} 
\end{array}$
\caption{A three class nearest neighbours visual example with two features $X_1, X_2$}
\end{center}
\end{figure}


\end{frame}
%%%%%%%%%%%%%%%%%%%%%%%%%%%%%%%%%%%%%%%%%%%%%%%%%%%%%%%%%%



%%%%%%%%%%%%%%%%%%%%%%%%%%%%%%%%%%%%%%%%%%%%%%%%%%%%%%%%%%
\begin{frame}{K Nearest Neighbours}

 
 \begin{algorithm}[K Nearest Neighbours]
   \begin{enumerate}
   
   \item Store all the training data $(y_i, x_{i1},...,x_{ip})$
   
   \item For each new point $x_q = (x_{q1},...x_{qp})$, find the $K$ \textbf{nearest} neighbours in the training set, indexed by $\mathcal{N}_0$.
   
   \item For each possible value  $j \in \mathcal{C}$, compute 
   $$\hat{P}(Y=j|X=x_q) = \frac{1}{K} \sum_{i \in \mathcal{N}_0} I(y_j=j)$$
   
   \item Assign the class $j$ for which $\hat{P}(Y=j|X=x_q)$ is maximal.
   
   \end{enumerate}
\end{algorithm}
 
\end{frame}
%%%%%%%%%%%%%%%%%%%%%%%%%%%%%%%%%%%%%%%%%%%%%%%%%%%%%%%%%%


%%%%%%%%%%%%%%%%%%%%%%%%%%%%%%%%%%%%%%%%%%%%%%%%%%%%%%%%%%
\begin{frame}{K Nearest Neighbours Estimates Complicated Functions}


\begin{figure}[h]
\begin{center}$
\begin{array}{c}
\includegraphics[scale=.5]<1>{figures/Map1NN.png} 
\end{array}$
\caption{A three class nearest neighbours visual example with two features $X_1, X_2$}
\end{center}
\end{figure}
  
\end{frame}
%%%%%%%%%%%%%%%%%%%%%%%%%%%%%%%%%%%%%%%%%%%%%%%%%%%%%%%%%%




%%%%%%%%%%%%%%%%%%%%%%%%%%%%%%%%%%%%%%%%%%%%%%%%%%%%%%%%%%
\begin{frame}{Some thoughts on KNN}
 
\begin{itemize}
\item KNN is among the simplest of all machine learning algorithms.

\item Its explicitly non-parametric (you dont estimate any coefficients)

\item KNN is referred to as a lazy learning algorithm: the function is only approximated locally and all computation is deferred until classification. 

\item So there is actually no explicit training step.

\item Results from KNN are sensitive to the local structure of the data (example highly skewed, ``majority voting'' distorted, weights).

\item Its (potentially) too slow, as it may be impossible to hold all the training data in memory.

\end{itemize}
 
\end{frame}
%%%%%%%%%%%%%%%%%%%%%%%%%%%%%%%%%%%%%%%%%%%%%%%%%%%%%%%%%%





%%%%%%%%%%%%%%%%%%%%%%%%%%%%%%%%%%%%%%%%%%%%%%%%%%%%%%%%%%
\begin{frame}{Applications in Research: Fetzer and Marden, 2014}
 
 
\begin{figure}[h]
\begin{center}$
\begin{array}{c}
\includegraphics[scale=.4]<1>{figures/landuse-knn1.png} 
\includegraphics[scale=.4]<2>{figures/landuse-knn2.png} 
\includegraphics[scale=.4]<3>{figures/landuse-knn3.png} 
\includegraphics[scale=.4]<4>{figures/landuse-knn4.png} 
\end{array}$
\caption{Classifiying historical landsat imagery into Forest or Non Forest Status.}
\end{center}
\end{figure}

[See forested.R]
 
 
\end{frame}
%%%%%%%%%%%%%%%%%%%%%%%%%%%%%%%%%%%%%%%%%%%%%%%%%%%%%%%%%%



%%%%%%%%%%%%%%%%%%%%%%%%%%%%%%%%%%%%%%%%%%%%%%%%%%%%%%%%%%
\begin{frame}[fragile, shrink]{Plotting Out Band Values for Red, Near Infrared}
 
\begin{knitrout}\tiny
\definecolor{shadecolor}{rgb}{0.969, 0.969, 0.969}\color{fgcolor}\begin{kframe}
\begin{alltt}
\hlkwd{options}\hlstd{(}\hlkwc{stringsAsFactors}\hlstd{=}\hlnum{FALSE}\hlstd{)}
\hlkwd{library}\hlstd{(data.table)}
\hlkwd{library}\hlstd{(plyr)}
\hlkwd{library}\hlstd{(class)}
\hlstd{FORESTED} \hlkwb{<-} \hlkwd{data.table}\hlstd{(}\hlkwd{read.csv}\hlstd{(}\hlstr{"R/forested.csv"}\hlstd{))}
\hlstd{COMPOSITE} \hlkwb{<-} \hlkwd{data.table}\hlstd{(}\hlkwd{read.csv}\hlstd{(}\hlstr{"R/composite.csv"}\hlstd{))}
\hlstd{MODIS} \hlkwb{<-} \hlkwd{data.table}\hlstd{(}\hlkwd{read.csv}\hlstd{(}\hlstr{"R/modis.csv"}\hlstd{))}
\hlkwd{setnames}\hlstd{(MODIS,} \hlstr{"mean"}\hlstd{,} \hlstr{"landcover"}\hlstd{)}
\hlkwd{setnames}\hlstd{(FORESTED,} \hlstr{"mean"}\hlstd{,} \hlstr{"forestcover"}\hlstd{)}
\hlstd{DF}\hlkwb{<-} \hlkwd{join}\hlstd{(}\hlkwd{join}\hlstd{(FORESTED[,} \hlkwd{c}\hlstd{(}\hlstr{"system.index"}\hlstd{,} \hlstr{"forestcover"}\hlstd{),} \hlkwc{with}\hlstd{=F],}
               \hlstd{MODIS[,} \hlkwd{c}\hlstd{(}\hlstr{"system.index"}\hlstd{,} \hlstr{"landcover"}\hlstd{),} \hlkwc{with}\hlstd{=F]), COMPOSITE)}
\hlstd{DF[, Forested} \hlkwb{:=} \hlstd{forestcover}\hlopt{>}\hlnum{.8}\hlstd{]}
\hlstd{DF[, MODISforested} \hlkwb{:=} \hlstd{(landcover}\hlopt{>}\hlnum{0} \hlopt{&} \hlstd{landcover}\hlopt{<=}\hlnum{5}\hlstd{)]}
\hlstd{DF[, B3}\hlkwb{:=}\hlkwd{scale}\hlstd{(B3)]}
\hlstd{DF[, B4}\hlkwb{:=}\hlkwd{scale}\hlstd{(B4)]}
\hlstd{df.xlim} \hlkwb{<-} \hlkwd{range}\hlstd{(DF}\hlopt{$}\hlstd{B3)}
\hlstd{df.ylim}\hlkwb{<-} \hlkwd{range}\hlstd{(DF}\hlopt{$}\hlstd{B4)}
\hlkwd{plot}\hlstd{(DF[Forested}\hlopt{==}\hlnum{TRUE}\hlstd{][}\hlnum{1}\hlopt{:}\hlnum{120}\hlstd{,} \hlkwd{c}\hlstd{(}\hlstr{"B3"}\hlstd{,} \hlstr{"B4"}\hlstd{),} \hlkwc{with}\hlstd{=F],} \hlkwc{col}\hlstd{=}\hlstr{"green"}\hlstd{,} \hlkwc{xlim}\hlstd{=df.xlim,} \hlkwc{ylim}\hlstd{=df.ylim)}
\hlkwd{points}\hlstd{(DF[Forested}\hlopt{==}\hlnum{FALSE}\hlstd{][}\hlnum{1}\hlopt{:}\hlnum{80}\hlstd{,} \hlkwd{c}\hlstd{(}\hlstr{"B3"}\hlstd{,} \hlstr{"B4"}\hlstd{),} \hlkwc{with}\hlstd{=F],} \hlkwc{col}\hlstd{=}\hlstr{"red"}\hlstd{,} \hlkwc{xlim}\hlstd{=df.xlim,} \hlkwc{ylim}\hlstd{=df.ylim)}
\end{alltt}
\end{kframe}

{\centering \includegraphics[width=\maxwidth]{figures/knitr-knnforestedexample-1} 

}



\end{knitrout}
\end{frame}
%%%%%%%%%%%%%%%%%%%%%%%%%%%%%%%%%%%%%%%%%%%%%%%%%%%%%%%%%%


 %%%%%%%%%%%%%%%%%%%%%%%%%%%%%%%%%%%%%%%%%%%%%%%%%%%%%%%%%%
\begin{frame}[fragile, shrink]{Results for KNN with K=10}
\begin{knitrout}\tiny
\definecolor{shadecolor}{rgb}{0.969, 0.969, 0.969}\color{fgcolor}\begin{kframe}
\begin{alltt}
\hlkwd{set.seed}\hlstd{(}\hlnum{1151}\hlstd{)}
\hlstd{train}\hlkwb{<-}\hlkwd{sample}\hlstd{(}\hlnum{1}\hlopt{:}\hlnum{1000}\hlstd{,} \hlnum{800}\hlstd{)}
\hlcom{# get the contour map}
\hlstd{px1} \hlkwb{<-} \hlkwd{range}\hlstd{(DF[}\hlopt{-}\hlstd{train]}\hlopt{$}\hlstd{B3)}
\hlstd{px1}\hlkwb{<-}\hlkwd{seq}\hlstd{(px1[}\hlnum{1}\hlstd{], px1[}\hlnum{2}\hlstd{],} \hlnum{0.05}\hlstd{)}
\hlstd{px2} \hlkwb{<-} \hlkwd{range}\hlstd{(DF[}\hlopt{-}\hlstd{train]}\hlopt{$}\hlstd{B4)}
\hlstd{px2}\hlkwb{<-}\hlkwd{seq}\hlstd{(px2[}\hlnum{1}\hlstd{], px2[}\hlnum{2}\hlstd{],}\hlnum{0.05}\hlstd{)}
\hlstd{xnew} \hlkwb{<-} \hlkwd{expand.grid}\hlstd{(}\hlkwc{x1} \hlstd{= px1,} \hlkwc{x2} \hlstd{= px2)}
\hlstd{knn10} \hlkwb{<-} \hlkwd{knn}\hlstd{(DF[train,} \hlkwd{c}\hlstd{(}\hlstr{"B3"}\hlstd{,}\hlstr{"B4"}\hlstd{),} \hlkwc{with}\hlstd{=F],} \hlkwc{test} \hlstd{= xnew,} \hlkwc{cl} \hlstd{= DF[train]}\hlopt{$}\hlstd{Forested,} \hlkwc{k} \hlstd{=} \hlnum{10}\hlstd{,} \hlkwc{prob} \hlstd{=} \hlnum{TRUE}\hlstd{)}
\hlstd{prob} \hlkwb{<-} \hlkwd{attr}\hlstd{(knn10,} \hlstr{"prob"}\hlstd{)}
\hlstd{prob} \hlkwb{<-} \hlkwd{ifelse}\hlstd{(knn10}\hlopt{==}\hlnum{TRUE}\hlstd{, prob,} \hlnum{1}\hlopt{-}\hlstd{prob)}
\hlstd{prob10} \hlkwb{<-} \hlkwd{matrix}\hlstd{(prob,} \hlkwc{nrow} \hlstd{=} \hlkwd{length}\hlstd{(px1),} \hlkwc{ncol} \hlstd{=} \hlkwd{length}\hlstd{(px2))}
\hlkwd{par}\hlstd{(}\hlkwc{mar} \hlstd{=} \hlkwd{rep}\hlstd{(}\hlnum{2}\hlstd{,}\hlnum{4}\hlstd{))}
\hlkwd{contour}\hlstd{(px1, px2, prob10,} \hlkwc{levels}\hlstd{=}\hlnum{.5}\hlstd{,} \hlkwc{labels}\hlstd{=}\hlstr{""}\hlstd{,} \hlkwc{xlab}\hlstd{=}\hlstr{""}\hlstd{,} \hlkwc{ylab}\hlstd{=}\hlstr{""}\hlstd{,} \hlkwc{main}\hlstd{=} \hlstr{"10-nearest neighbour"}\hlstd{,} \hlkwc{axes}\hlstd{=}\hlnum{FALSE}\hlstd{)}
\hlkwd{points}\hlstd{(DF[train,}\hlkwd{c}\hlstd{(}\hlstr{"B3"}\hlstd{,}\hlstr{"B4"}\hlstd{),} \hlkwc{with}\hlstd{=F],} \hlkwc{col}\hlstd{=}\hlkwd{ifelse}\hlstd{(DF[train]}\hlopt{$}\hlstd{Forested}\hlopt{==}\hlnum{TRUE}\hlstd{,} \hlstr{"green"}\hlstd{,} \hlstr{"red"}\hlstd{))}
\hlkwd{points}\hlstd{(xnew,} \hlkwc{pch}\hlstd{=}\hlstr{"."}\hlstd{,} \hlkwc{cex}\hlstd{=}\hlnum{3.5}\hlstd{,} \hlkwc{col}\hlstd{=}\hlkwd{ifelse}\hlstd{(prob10}\hlopt{>}\hlnum{.5}\hlstd{,} \hlstr{"green"}\hlstd{,} \hlstr{"red"}\hlstd{))}
\hlkwd{box}\hlstd{()}
\end{alltt}
\end{kframe}

{\centering \includegraphics[width=\maxwidth]{figures/knitr-knnforestedexampleknn10-1} 

}



\end{knitrout}
\end{frame}
%%%%%%%%%%%%%%%%%%%%%%%%%%%%%%%%%%%%%%%%%%%%%%%%%%%%%%%%%%

 %%%%%%%%%%%%%%%%%%%%%%%%%%%%%%%%%%%%%%%%%%%%%%%%%%%%%%%%%%
\begin{frame}[fragile, shrink]{How do we perform in terms of classification?}

\begin{knitrout}\tiny
\definecolor{shadecolor}{rgb}{0.969, 0.969, 0.969}\color{fgcolor}\begin{kframe}
\begin{alltt}
\hlstd{knn10} \hlkwb{<-} \hlkwd{knn}\hlstd{(DF[train,} \hlkwd{c}\hlstd{(}\hlstr{"B3"}\hlstd{,}\hlstr{"B4"}\hlstd{),} \hlkwc{with}\hlstd{=F],} \hlkwc{test} \hlstd{= DF[}\hlopt{-}\hlstd{train,} \hlkwd{c}\hlstd{(}\hlstr{"B3"}\hlstd{,}\hlstr{"B4"}\hlstd{),} \hlkwc{with}\hlstd{=F],}
             \hlkwc{cl} \hlstd{= DF[train]}\hlopt{$}\hlstd{Forested,} \hlkwc{k} \hlstd{=} \hlnum{10}\hlstd{,} \hlkwc{prob} \hlstd{=} \hlnum{TRUE}\hlstd{)}
\hlstd{prob} \hlkwb{<-} \hlkwd{attr}\hlstd{(knn10,} \hlstr{"prob"}\hlstd{)}
\hlstd{prob} \hlkwb{<-} \hlkwd{ifelse}\hlstd{(knn10}\hlopt{==}\hlnum{TRUE}\hlstd{, prob,} \hlnum{1}\hlopt{-}\hlstd{prob)}
\hlkwd{table}\hlstd{(prob}\hlopt{>}\hlnum{0.5}\hlstd{,DF[}\hlopt{-}\hlstd{train]}\hlopt{$}\hlstd{Forested)}
\end{alltt}
\begin{verbatim}
##        
##         FALSE TRUE
##   FALSE    52   10
##   TRUE     15  123
\end{verbatim}
\end{kframe}
\end{knitrout}

Overall error rate is quite low: 25/200 = 12.5\%. 

\begin{itemize}

\item Pixels are wrongly classified as forested (type 1 error) 10/200

\item Pixels are wrongly classified as non forested 15/200

\item Suppose you are sending an enforcement agent to fine people who have illegally deforested land (predicted value = FALSE), i.e. for 62 pixels.

\item It costs a lot of money to send police out there, but in 16\% (10/(52+10)) of the cases you find the forest is actually standing... 

\end{itemize}

\end{frame}
%%%%%%%%%%%%%%%%%%%%%%%%%%%%%%%%%%%%%%%%%%%%%%%%%%%%%%%%%%


 %%%%%%%%%%%%%%%%%%%%%%%%%%%%%%%%%%%%%%%%%%%%%%%%%%%%%%%%%%
\begin{frame}[fragile]{Changing $\bar{c}$...}

Make it less likely, that we have pixels wrongly classified as being forested...reduce the threshold $\bar{c}$.

\begin{knitrout}\tiny
\definecolor{shadecolor}{rgb}{0.969, 0.969, 0.969}\color{fgcolor}\begin{kframe}
\begin{alltt}
\hlkwd{table}\hlstd{(prob}\hlopt{>}\hlnum{0.4}\hlstd{,DF[}\hlopt{-}\hlstd{train]}\hlopt{$}\hlstd{Forested)}
\end{alltt}
\begin{verbatim}
##        
##         FALSE TRUE
##   FALSE    51    5
##   TRUE     16  128
\end{verbatim}
\end{kframe}
\end{knitrout}
\begin{itemize}

\item Now, we will send enforcement teams out 56 times, but only in 8\% of the cases, will the forest still be standing...

\item While the threshold $\bar{c}=0.5$, theoretically minimizes overall prediction error, you may still prefer different $\bar{c}$, in case there are different costs associated with type 1 or type 2 errors.

\end{itemize}
\end{frame}
%%%%%%%%%%%%%%%%%%%%%%%%%%%%%%%%%%%%%%%%%%%%%%%%%%%%%%%%%%



 %%%%%%%%%%%%%%%%%%%%%%%%%%%%%%%%%%%%%%%%%%%%%%%%%%%%%%%%%%
\begin{frame}{Exercise: Plot an ROC Curve...}

Can you try to compute ROC curves and plot them out?

\end{frame}
%%%%%%%%%%%%%%%%%%%%%%%%%%%%%%%%%%%%%%%%%%%%%%%%%%%%%%%%%%



 %%%%%%%%%%%%%%%%%%%%%%%%%%%%%%%%%%%%%%%%%%%%%%%%%%%%%%%%%%
\begin{frame}{How to Choose $K$?}

\begin{itemize}

\item As you will have guessed, it turns out that $K$ is a tuning parameter that we need to select.

\item In binary (two class) classification problems, it is helpful to choose $K$ to be an odd number -- why?

\item Cross validation is our preferred method, i.e. for each different value of $K$, we perform cross validation and plot out the overall average error rates.

\item Why does the choice of $K$ matter?

\end{itemize}
\end{frame}
%%%%%%%%%%%%%%%%%%%%%%%%%%%%%%%%%%%%%%%%%%%%%%%%%%%%%%%%%%



 %%%%%%%%%%%%%%%%%%%%%%%%%%%%%%%%%%%%%%%%%%%%%%%%%%%%%%%%%%
\begin{frame}[fragile, shrink]{Results for KNN with K=15}
\begin{knitrout}\tiny
\definecolor{shadecolor}{rgb}{0.969, 0.969, 0.969}\color{fgcolor}\begin{kframe}
\begin{alltt}
\hlkwd{set.seed}\hlstd{(}\hlnum{1151}\hlstd{)}
\hlstd{train}\hlkwb{<-}\hlkwd{sample}\hlstd{(}\hlnum{1}\hlopt{:}\hlnum{1000}\hlstd{,} \hlnum{800}\hlstd{)}
\hlcom{# get the contour map}
\hlstd{px1} \hlkwb{<-} \hlkwd{range}\hlstd{(DF[}\hlopt{-}\hlstd{train]}\hlopt{$}\hlstd{B3)}
\hlstd{px1}\hlkwb{<-}\hlkwd{seq}\hlstd{(px1[}\hlnum{1}\hlstd{], px1[}\hlnum{2}\hlstd{],} \hlnum{0.05}\hlstd{)}
\hlstd{px2} \hlkwb{<-} \hlkwd{range}\hlstd{(DF[}\hlopt{-}\hlstd{train]}\hlopt{$}\hlstd{B4)}
\hlstd{px2}\hlkwb{<-}\hlkwd{seq}\hlstd{(px2[}\hlnum{1}\hlstd{], px2[}\hlnum{2}\hlstd{],}\hlnum{0.05}\hlstd{)}
\hlstd{xnew} \hlkwb{<-} \hlkwd{expand.grid}\hlstd{(}\hlkwc{x1} \hlstd{= px1,} \hlkwc{x2} \hlstd{= px2)}
\hlstd{knn15} \hlkwb{<-} \hlkwd{knn}\hlstd{(DF[train,} \hlkwd{c}\hlstd{(}\hlstr{"B3"}\hlstd{,}\hlstr{"B4"}\hlstd{),} \hlkwc{with}\hlstd{=F],} \hlkwc{test} \hlstd{= xnew,} \hlkwc{cl} \hlstd{= DF[train]}\hlopt{$}\hlstd{Forested,} \hlkwc{k} \hlstd{=} \hlnum{15}\hlstd{,} \hlkwc{prob} \hlstd{=} \hlnum{TRUE}\hlstd{)}
\hlstd{prob} \hlkwb{<-} \hlkwd{attr}\hlstd{(knn15,} \hlstr{"prob"}\hlstd{)}
\hlstd{prob} \hlkwb{<-} \hlkwd{ifelse}\hlstd{(knn15}\hlopt{==}\hlnum{TRUE}\hlstd{, prob,} \hlnum{1}\hlopt{-}\hlstd{prob)}
\hlstd{prob15} \hlkwb{<-} \hlkwd{matrix}\hlstd{(prob,} \hlkwc{nrow} \hlstd{=} \hlkwd{length}\hlstd{(px1),} \hlkwc{ncol} \hlstd{=} \hlkwd{length}\hlstd{(px2))}
\hlkwd{par}\hlstd{(}\hlkwc{mar} \hlstd{=} \hlkwd{rep}\hlstd{(}\hlnum{2}\hlstd{,}\hlnum{4}\hlstd{))}
\hlkwd{contour}\hlstd{(px1, px2, prob15,} \hlkwc{levels}\hlstd{=}\hlnum{.5}\hlstd{,} \hlkwc{labels}\hlstd{=}\hlstr{""}\hlstd{,} \hlkwc{xlab}\hlstd{=}\hlstr{""}\hlstd{,} \hlkwc{ylab}\hlstd{=}\hlstr{""}\hlstd{,} \hlkwc{main}\hlstd{=} \hlstr{"15-nearest neighbour"}\hlstd{,} \hlkwc{axes}\hlstd{=}\hlnum{FALSE}\hlstd{)}
\hlkwd{points}\hlstd{(DF[train,}\hlkwd{c}\hlstd{(}\hlstr{"B3"}\hlstd{,}\hlstr{"B4"}\hlstd{),} \hlkwc{with}\hlstd{=F],} \hlkwc{col}\hlstd{=}\hlkwd{ifelse}\hlstd{(DF[train]}\hlopt{$}\hlstd{Forested}\hlopt{==}\hlnum{TRUE}\hlstd{,} \hlstr{"green"}\hlstd{,} \hlstr{"red"}\hlstd{))}
\hlkwd{points}\hlstd{(xnew,} \hlkwc{pch}\hlstd{=}\hlstr{"."}\hlstd{,} \hlkwc{cex}\hlstd{=}\hlnum{3.5}\hlstd{,} \hlkwc{col}\hlstd{=}\hlkwd{ifelse}\hlstd{(prob15}\hlopt{>}\hlnum{.5}\hlstd{,} \hlstr{"green"}\hlstd{,} \hlstr{"red"}\hlstd{))}
\hlkwd{box}\hlstd{()}
\end{alltt}
\end{kframe}

{\centering \includegraphics[width=\maxwidth]{figures/knitr-knnforestedexampleknn15-1} 

}



\end{knitrout}
\end{frame}
%%%%%%%%%%%%%%%%%%%%%%%%%%%%%%%%%%%%%%%%%%%%%%%%%%%%%%%%%%



 %%%%%%%%%%%%%%%%%%%%%%%%%%%%%%%%%%%%%%%%%%%%%%%%%%%%%%%%%%
\begin{frame}[fragile, shrink]{Results for KNN with K=1}
\begin{knitrout}\tiny
\definecolor{shadecolor}{rgb}{0.969, 0.969, 0.969}\color{fgcolor}\begin{kframe}
\begin{alltt}
\hlkwd{set.seed}\hlstd{(}\hlnum{1151}\hlstd{)}
\hlstd{train}\hlkwb{<-}\hlkwd{sample}\hlstd{(}\hlnum{1}\hlopt{:}\hlnum{1000}\hlstd{,} \hlnum{800}\hlstd{)}
\hlcom{# get the contour map}
\hlstd{px1} \hlkwb{<-} \hlkwd{range}\hlstd{(DF[}\hlopt{-}\hlstd{train]}\hlopt{$}\hlstd{B3)}
\hlstd{px1}\hlkwb{<-}\hlkwd{seq}\hlstd{(px1[}\hlnum{1}\hlstd{], px1[}\hlnum{2}\hlstd{],} \hlnum{0.05}\hlstd{)}
\hlstd{px2} \hlkwb{<-} \hlkwd{range}\hlstd{(DF[}\hlopt{-}\hlstd{train]}\hlopt{$}\hlstd{B4)}
\hlstd{px2}\hlkwb{<-}\hlkwd{seq}\hlstd{(px2[}\hlnum{1}\hlstd{], px2[}\hlnum{2}\hlstd{],}\hlnum{0.05}\hlstd{)}
\hlstd{xnew} \hlkwb{<-} \hlkwd{expand.grid}\hlstd{(}\hlkwc{x1} \hlstd{= px1,} \hlkwc{x2} \hlstd{= px2)}
\hlstd{knn1} \hlkwb{<-} \hlkwd{knn}\hlstd{(DF[train,} \hlkwd{c}\hlstd{(}\hlstr{"B3"}\hlstd{,}\hlstr{"B4"}\hlstd{),} \hlkwc{with}\hlstd{=F],} \hlkwc{test} \hlstd{= xnew,} \hlkwc{cl} \hlstd{= DF[train]}\hlopt{$}\hlstd{Forested,} \hlkwc{k} \hlstd{=} \hlnum{1}\hlstd{,} \hlkwc{prob} \hlstd{=} \hlnum{TRUE}\hlstd{)}
\hlstd{prob} \hlkwb{<-} \hlkwd{attr}\hlstd{(knn1,} \hlstr{"prob"}\hlstd{)}
\hlstd{prob} \hlkwb{<-} \hlkwd{ifelse}\hlstd{(knn1}\hlopt{==}\hlnum{TRUE}\hlstd{, prob,} \hlnum{1}\hlopt{-}\hlstd{prob)}
\hlstd{prob1} \hlkwb{<-} \hlkwd{matrix}\hlstd{(prob,} \hlkwc{nrow} \hlstd{=} \hlkwd{length}\hlstd{(px1),} \hlkwc{ncol} \hlstd{=} \hlkwd{length}\hlstd{(px2))}
\hlkwd{par}\hlstd{(}\hlkwc{mar} \hlstd{=} \hlkwd{rep}\hlstd{(}\hlnum{2}\hlstd{,}\hlnum{4}\hlstd{))}
\hlkwd{contour}\hlstd{(px1, px2, prob1,} \hlkwc{levels}\hlstd{=}\hlnum{.5}\hlstd{,} \hlkwc{labels}\hlstd{=}\hlstr{""}\hlstd{,} \hlkwc{xlab}\hlstd{=}\hlstr{""}\hlstd{,} \hlkwc{ylab}\hlstd{=}\hlstr{""}\hlstd{,} \hlkwc{main}\hlstd{=} \hlstr{"1-nearest neighbour"}\hlstd{,} \hlkwc{axes}\hlstd{=}\hlnum{FALSE}\hlstd{)}
\hlkwd{points}\hlstd{(DF[train,}\hlkwd{c}\hlstd{(}\hlstr{"B3"}\hlstd{,}\hlstr{"B4"}\hlstd{),} \hlkwc{with}\hlstd{=F],} \hlkwc{col}\hlstd{=}\hlkwd{ifelse}\hlstd{(DF[train]}\hlopt{$}\hlstd{Forested}\hlopt{==}\hlnum{TRUE}\hlstd{,} \hlstr{"green"}\hlstd{,} \hlstr{"red"}\hlstd{))}
\hlkwd{points}\hlstd{(xnew,} \hlkwc{pch}\hlstd{=}\hlstr{"."}\hlstd{,} \hlkwc{cex}\hlstd{=}\hlnum{3.5}\hlstd{,} \hlkwc{col}\hlstd{=}\hlkwd{ifelse}\hlstd{(prob1}\hlopt{>}\hlnum{.5}\hlstd{,} \hlstr{"green"}\hlstd{,} \hlstr{"red"}\hlstd{))}
\hlkwd{box}\hlstd{()}
\end{alltt}
\end{kframe}

{\centering \includegraphics[width=\maxwidth]{figures/knitr-knnforestedexampleknn1-1} 

}



\end{knitrout}
\end{frame}
%%%%%%%%%%%%%%%%%%%%%%%%%%%%%%%%%%%%%%%%%%%%%%%%%%%%%%%%%%



 %%%%%%%%%%%%%%%%%%%%%%%%%%%%%%%%%%%%%%%%%%%%%%%%%%%%%%%%%%
\begin{frame}[fragile,shrink]{Cross Validation and choice of $K$}

\begin{knitrout}\tiny
\definecolor{shadecolor}{rgb}{0.969, 0.969, 0.969}\color{fgcolor}\begin{kframe}
\begin{alltt}
\hlstd{RES}\hlkwb{<-}\hlkwa{NULL}
\hlkwa{for}\hlstd{(K} \hlkwa{in} \hlnum{1}\hlopt{:}\hlnum{50}\hlstd{) \{}
\hlstd{knncv} \hlkwb{<-} \hlkwd{knn.cv}\hlstd{(DF[,} \hlkwd{c}\hlstd{(}\hlstr{"B3"}\hlstd{,}\hlstr{"B4"}\hlstd{),} \hlkwc{with}\hlstd{=F],}
                \hlkwc{cl} \hlstd{= DF}\hlopt{$}\hlstd{Forested,} \hlkwc{k} \hlstd{= K,} \hlkwc{prob} \hlstd{=} \hlnum{TRUE}\hlstd{)}
\hlstd{RES}\hlkwb{<-}\hlkwd{rbind}\hlstd{(RES,}\hlkwd{cbind}\hlstd{(K,}\hlnum{1}\hlopt{-}\hlstd{(}\hlkwd{table}\hlstd{(knncv,DF}\hlopt{$}\hlstd{Forested)[}\hlnum{1}\hlstd{,}\hlnum{1}\hlstd{]}\hlopt{+}\hlkwd{table}\hlstd{(knncv,DF}\hlopt{$}\hlstd{Forested)[}\hlnum{2}\hlstd{,}\hlnum{2}\hlstd{])}\hlopt{/}\hlnum{1000}\hlstd{))}
\hlstd{\}}
\hlkwd{plot}\hlstd{(RES)}
\end{alltt}
\end{kframe}

{\centering \includegraphics[width=\maxwidth]{figures/knitr-knn10crossvalidation-1} 

}



\end{knitrout}

\end{frame}
%%%%%%%%%%%%%%%%%%%%%%%%%%%%%%%%%%%%%%%%%%%%%%%%%%%%%%%%%%


 %%%%%%%%%%%%%%%%%%%%%%%%%%%%%%%%%%%%%%%%%%%%%%%%%%%%%%%%%%
\begin{frame}[fragile]{Comparing Logistic Regression Versus kNN Classification}

\begin{itemize}

\item Cross Validation suggested that we should use $k \approx 10$ for kNN classification.

\item Lets compare the performance of logistic regression versus 10NN for this example.

\begin{knitrout}\tiny
\definecolor{shadecolor}{rgb}{0.969, 0.969, 0.969}\color{fgcolor}\begin{kframe}
\begin{alltt}
\hlkwd{set.seed}\hlstd{(}\hlnum{1151}\hlstd{)}
\hlstd{train}\hlkwb{<-}\hlkwd{sample}\hlstd{(}\hlnum{1}\hlopt{:}\hlnum{1000}\hlstd{,} \hlnum{800}\hlstd{)}

\hlstd{glm.fit}\hlkwb{<-}\hlkwd{glm}\hlstd{(Forested} \hlopt{~}  \hlstd{B3} \hlopt{+} \hlstd{B4,}             \hlkwc{data}\hlstd{=DF[train],} \hlkwc{family}\hlstd{=}\hlkwd{binomial}\hlstd{(}\hlkwc{link}\hlstd{=logit))}

\hlstd{glm.predict} \hlkwb{<-} \hlkwd{predict.glm}\hlstd{(glm.fit, DF[}\hlopt{-}\hlstd{train],} \hlkwc{type}\hlstd{=}\hlstr{"response"}\hlstd{)}

\hlstd{knn10} \hlkwb{<-} \hlkwd{knn}\hlstd{(DF[train,} \hlkwd{c}\hlstd{(}\hlstr{"B3"}\hlstd{,}\hlstr{"B4"}\hlstd{),} \hlkwc{with}\hlstd{=F],} \hlkwc{test} \hlstd{= DF[}\hlopt{-}\hlstd{train,} \hlkwd{c}\hlstd{(}\hlstr{"B3"}\hlstd{,}\hlstr{"B4"}\hlstd{),} \hlkwc{with}\hlstd{=F],}
             \hlkwc{cl} \hlstd{= DF[train]}\hlopt{$}\hlstd{Forested,} \hlkwc{k} \hlstd{=} \hlnum{10}\hlstd{,} \hlkwc{prob} \hlstd{=} \hlnum{TRUE}\hlstd{)}
\hlstd{prob} \hlkwb{<-} \hlkwd{attr}\hlstd{(knn10,} \hlstr{"prob"}\hlstd{)}
\hlstd{prob} \hlkwb{<-} \hlkwd{ifelse}\hlstd{(knn10}\hlopt{==}\hlnum{TRUE}\hlstd{, prob,} \hlnum{1}\hlopt{-}\hlstd{prob)}
\hlkwd{table}\hlstd{(prob}\hlopt{>}\hlnum{0.5}\hlstd{,DF[}\hlopt{-}\hlstd{train]}\hlopt{$}\hlstd{Forested)}
\end{alltt}
\begin{verbatim}
##        
##         FALSE TRUE
##   FALSE    52   10
##   TRUE     15  123
\end{verbatim}
\begin{alltt}
\hlkwd{table}\hlstd{(glm.predict}\hlopt{>}\hlnum{0.5}\hlstd{,DF[}\hlopt{-}\hlstd{train]}\hlopt{$}\hlstd{Forested)}
\end{alltt}
\begin{verbatim}
##        
##         FALSE TRUE
##   FALSE    48    6
##   TRUE     19  127
\end{verbatim}
\end{kframe}
\end{knitrout}
\end{itemize}
\end{frame}
%%%%%%%%%%%%%%%%%%%%%%%%%%%%%%%%%%%%%%%%%%%%%%%%%%%%%%%%%%

 %%%%%%%%%%%%%%%%%%%%%%%%%%%%%%%%%%%%%%%%%%%%%%%%%%%%%%%%%%
\begin{frame}[fragile]{Comparing Logistic Regression Versus kNN Classification}

\begin{itemize}

\item The result is not surprising. The data suggests a near linear decision boundary, both kNN as well as logistic regression estimate that linear boundary reasonably well.

\item You may still decide that you prefer logistic regression, since it is much more interpretable and less of a black box.

\item With logistic regression, you get interpretable coefficients; here we do not care much for inference, so it doesnt really matter. 

\item But in general, a (reasonable) rule of thumb is that you should prefer a simple parametric interpretable method over a non parametric method in case they yield similar results.



\end{itemize}
\end{frame}
%%%%%%%%%%%%%%%%%%%%%%%%%%%%%%%%%%%%%%%%%%%%%%%%%%%%%%%%%%



 %%%%%%%%%%%%%%%%%%%%%%%%%%%%%%%%%%%%%%%%%%%%%%%%%%%%%%%%%%
\begin{frame}[fragile]{When does kNN outperform?}


\begin{figure}[h]
\begin{center}$
\begin{array}{c}
\includegraphics[scale=.15]<1>{figures/quadratic.png} 
\includegraphics[scale=.7]<2>{figures/knn-outperform.png} 
\end{array}$
\caption{Example of Non-Linear Classification Problems}
\end{center}
\end{figure}


\end{frame}
%%%%%%%%%%%%%%%%%%%%%%%%%%%%%%%%%%%%%%%%%%%%%%%%%%%%%%%%%%


%%%%%%%%%%%%%%%%%%%%%%%%%%%%%%%%%%%%%%%%%%%%%%%%%%%%%%%%%%
\begin{frame}{KNN for text classification}
 
\begin{itemize}
\item As you can see from the specification, KNN uses Euclidian distances, which may be problematic if you want to classify documents of different lengths

\item In text classsification, KNN is also called 
\item Its explicitly non-parametric (you dont estimate any coefficients)

\item KNN is referred to as a lazy learning algorithm: the function is only approximated locally and all computation is deferred until classification. 

\item So there is actually no explicit training step.

\item Results from KNN are sensitive to the local structure of the data (example highly skewed, ``majority voting'' distorted, weights).

\item Its (potentially) too slow, as it may be impossible to hold all the training data in memory.

\end{itemize}
 
\end{frame}
%%%%%%%%%%%%%%%%%%%%%%%%%%%%%%%%%%%%%%%%%%%%%%%%%%%%%%%%%%







%%%%%%%%%%%%%%%%%%%%%%%%%%%%%%%%%%%%%%%%%%%%%%%%%%%%%%%%%%
\begin{frame}{KNN for text classification}
 
 \begin{center}
 \includegraphics[scale=0.45]{figures/knitr-knntextksensitivity-1.pdf}
 \end{center}
 
\end{frame}
%%%%%%%%%%%%%%%%%%%%%%%%%%%%%%%%%%%%%%%%%%%%%%%%%%%%%%%%%%





%%%%%%%%%%%%%%%%%%%%%%%%%%%%%%%%%%%%%%%%%%%%%%%%%%%%%%%%%%
\begin{frame}[fragile]{KNN for text classification}

\begin{knitrout}\tiny
\definecolor{shadecolor}{rgb}{0.969, 0.969, 0.969}\color{fgcolor}\begin{kframe}
\begin{alltt}
\hlkwd{library}\hlstd{(RTextTools)}
\hlcom{##knn function}
\hlkwd{library}\hlstd{(class)}
\hlstd{TTIM}\hlkwb{<-}\hlstd{TTIM[}\hlkwd{order}\hlstd{(sid)]}
\hlkwd{set.seed}\hlstd{(}\hlnum{06022017}\hlstd{)}
\hlstd{TTIM}\hlkwb{<-}\hlstd{TTIM[}\hlkwd{sample}\hlstd{(}\hlnum{1}\hlopt{:}\hlkwd{nrow}\hlstd{(TTIM),} \hlkwd{nrow}\hlstd{(TTIM))]}
\hlstd{L1} \hlkwb{<-} \hlkwd{create_matrix}\hlstd{(TTIM[,}\hlkwd{paste}\hlstd{(objectcleanpp,}\hlkwc{sep}\hlstd{=}\hlstr{" "}\hlstd{)],}
                    \hlkwc{language}\hlstd{=}\hlstr{"english"}\hlstd{,}\hlkwc{stemWords}\hlstd{=}\hlnum{FALSE}\hlstd{)}
\hlcom{##CREATION OF NON SPARSE MATRIX}
\hlstd{DTM}\hlkwb{<-}\hlkwd{as.matrix}\hlstd{(L1)}

\hlkwd{dim}\hlstd{(DTM)}
\end{alltt}
\begin{verbatim}
## [1] 1397 2460
\end{verbatim}
\begin{alltt}
\hlcom{#knn(train, test, labelsfortrainingset, k=1)}
\hlstd{res}\hlkwb{<-}\hlkwd{knn}\hlstd{(DTM[}\hlnum{201}\hlopt{:}\hlnum{1200}\hlstd{,], DTM[}\hlnum{1}\hlopt{:}\hlnum{200}\hlstd{,],} \hlkwc{cl}\hlstd{=}\hlkwd{as.factor}\hlstd{(TTIM[}\hlnum{201}\hlopt{:}\hlnum{1200}\hlstd{]}\hlopt{$}\hlstd{label1))}

\hlkwd{table}\hlstd{(res, TTIM[}\hlnum{1}\hlopt{:}\hlnum{200}\hlstd{]}\hlopt{$}\hlstd{label1)}
\end{alltt}
\begin{verbatim}
##            
## res         civilian security terrorist
##   civilian        40        5        17
##   security         7       31         1
##   terrorist        7        2        90
\end{verbatim}
\end{kframe}
\end{knitrout}

\end{frame}
%%%%%%%%%%%%%%%%%%%%%%%%%%%%%%%%%%%%%%%%%%%%%%%%%%%%%%%%%%

%%%%%%%%%%%%%%%%%%%%%%%%%%%%%%%%%%%%%%%%%%%%%%%%%%%%%%%%%%
\begin{frame}[fragile]{KNN for text classification}
\begin{knitrout}\tiny
\definecolor{shadecolor}{rgb}{0.969, 0.969, 0.969}\color{fgcolor}\begin{kframe}
\begin{alltt}
\hlstd{ACC}\hlkwb{<-}\hlkwa{NULL}
\hlkwa{for}\hlstd{(k} \hlkwa{in} \hlnum{1}\hlopt{:}\hlnum{30}\hlstd{) \{}
\hlcom{#knn(train, test, labelsfortrainingset, k=1)}
\hlstd{res}\hlkwb{<-}\hlkwd{knn}\hlstd{(DTM[}\hlnum{201}\hlopt{:}\hlnum{1200}\hlstd{,], DTM[}\hlnum{1}\hlopt{:}\hlnum{200}\hlstd{,],} \hlkwc{cl}\hlstd{=}\hlkwd{as.factor}\hlstd{(TTIM[}\hlnum{201}\hlopt{:}\hlnum{1200}\hlstd{]}\hlopt{$}\hlstd{label1),} \hlkwc{k}\hlstd{=k)}

\hlstd{ACC}\hlkwb{<-}\hlkwd{rbind}\hlstd{(ACC,} \hlkwd{data.frame}\hlstd{(}\hlkwc{k}\hlstd{=k,} \hlkwc{accuracy}\hlstd{=}\hlkwd{sum}\hlstd{(}\hlkwd{diag}\hlstd{(}\hlnum{3}\hlstd{)} \hlopt{*} \hlkwd{table}\hlstd{(res, TTIM[}\hlnum{1}\hlopt{:}\hlnum{200}\hlstd{]}\hlopt{$}\hlstd{label1))}\hlopt{/}\hlnum{200}\hlstd{))}

\hlstd{\}}
\end{alltt}
\end{kframe}
\end{knitrout}
 
\end{frame}
%%%%%%%%%%%%%%%%%%%%%%%%%%%%%%%%%%%%%%%%%%%%%%%%%%%%%%%%%%


\begin{knitrout}\tiny
\definecolor{shadecolor}{rgb}{0.969, 0.969, 0.969}\color{fgcolor}

{\centering \includegraphics[width=\maxwidth]{figures/knitr-knntextksensitivity2-1} 

}



\end{knitrout}

%%%%%%%%%%%%%%%%%%%%%%%%%%%%%%%%%%%%%%%%%%%%%%%%%%%%%%%%%%
\begin{frame}[fragile]{Accuracy for different values of $k$}
 
\begin{center}
\includegraphics[scale=0.4]{figures/knitr-knntextksensitivity2-1.pdf}
\end{center}

\end{frame}
%%%%%%%%%%%%%%%%%%%%%%%%%%%%%%%%%%%%%%%%%%%%%%%%%%%%%%%%%%




%%%%%%%%%%%%%%%%%%%%%%%%%%%%%%%%%%%%%%%%%%%%%%%%%%%%%%%%%%
\begin{frame}[shrink]{KNN is a discriminative classifier [non examinable]}



KNN is a discriminative algorithm since it models the conditional probability of a sample belonging to a given class. To see this just consider how one gets to the decision rule of kNNs.

A class label corresponds to a set of points which belong to some region in the feature space R. If you draw sample points from the actual probability distribution, p(x), independently, then the probability of drawing a sample from that class is,
 
 
$$P = \int_{R} p(x) dx$$

What if you have N points? The probability that K points of those N points fall in the region R follows the binomial distribution,


$$Prob(K) = {{N} \choose {K}}P^{K}(1-P)^{N-K}$$

As $N\rightarrow \infty$ this distribution is sharply peaked, so that the probability can be approximated by its mean value KN. 
\end{frame}
%%%%%%%%%%%%%%%%%%%%%%%%%%%%%%%%%%%%%%%%%%%%%%%%%%%%%%%%%%




%%%%%%%%%%%%%%%%%%%%%%%%%%%%%%%%%%%%%%%%%%%%%%%%%%%%%%%%%%
\begin{frame}[shrink]{KNN is a discriminative classifier[non examinable]}
An additional approximation is that the probability distribution over R remains approximately constant, so that one can approximate the integral by,

$$P = \int_{R} p(x) dx \approx p(x)V$$

where V is the total volume of the region. Under this approximations $p(x) \approx \frac{K}{NV}$  

Now, if we had several classes, we could repeat the same analysis for each one, which would give us,
$$p(x|C_{k}) = \frac{K_{k}}{N_{k}V}$$

where $K_k$ is the amount of points from class k which falls within that region and Nk is the total number of points which fall in that region. Notice $\sum_{k}N_{k}=N$.

Repeating the analysis with the binomial distribution, it is easy to see that we can estimate the prior $P(C_{k}) = \frac{N_{k}}{N}$

Using Bayes rule,
$$P(C_{k}|x)  = \frac{p(x|C_{k})p(C_{k})}{p(x)} = \frac{K_{k}}{K}$$
which is the rule for kNNs.

\end{frame}
%%%%%%%%%%%%%%%%%%%%%%%%%%%%%%%%%%%%%%%%%%%%%%%%%%%%%%%%%%




\end{document}
